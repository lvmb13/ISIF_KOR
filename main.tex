\documentclass[11pt,a4paper]{article}
\usepackage[utf8]{inputenc}
\usepackage{kotex}
\usepackage{textcomp}
\usepackage{quotes}
\usepackage{multicol}
\usepackage{geometry}
 \geometry{
 a4paper,
 total={170mm,257mm},
 left=20mm,
 top=20mm,
 }
\usepackage{hyperref}
\usepackage{sectsty}
\usepackage{indentfirst}
\allsectionsfont{\centering}


\title{산업사회와 그 미래}
\author{시어도어 존 카진스키}
\date{1995}

\begin{document}

\begin{multicols}{2}

\maketitle



\section*{서문}


1. 산업혁명과 그 결과는 인류에게 재앙이었다. 산업혁명 덕분에 “선진국”에 살고 있는 우리들의 평균  수명이 대폭 늘어난 것은 사실이다. 그러나 동시에 사회는 불안정해졌고, 삶은 무의미해졌으며, 인간은  비천한 존재로 전락했다. (제3세계의 경우에는 육체적 고통과 함께)심리적 고통은 광범위하게  확산되었으며, 자연은 돌이킬 수 없이 파괴되었다. 앞으로 기술이 계속 발전할 때 상황은 더욱 악화될  것이다. 인간의 존엄성은 아예 사라져 버릴 것이고, 자연은 더욱 극심하게 파괴될 것이다. 또한 추측컨대  사회적 혼란과 심리적 고통\hyperlink{1}{$^{1}$}도 훨씬 더 극심해질 것이며, “선진국”에서도 역시 육체적 고통에 시달리는  사람이 크게 늘어날 것이다.  


2. 이 산업-기술 체제는 살아남을 수도 있고 붕괴될 수도 있다. 이 체제가 살아남을 경우, 어쩌면 마침내  육체적, 심리적 고통을 낮은 수준으로 줄일 수 있을지도 모른다. 하지만 그것은 길고 몹시 고통스러운  적응기를 거친 후의 일일 것이며, 그 과정에서 인류와 수많은 생물들은 기계적 생산품 또는 사회라는  기계의 톱니바퀴에 불과한 존재로 격하되는 값비싼 대가를 치러야 할 것이다. 더 나아가 만일 이 체제가  살아남는다면, 그 결과는 불을 보듯 뻔하다. 체제를 개혁 또는 수정할 수 있는 방법은 없으며, 따라서  인간의 존엄성과 자율성이 박탈당하는 것을 막을 수도 없다.  


3. 이 체제가 붕괴될 경우에도 그 결과는 여전히 매우 고통스러울 것이다. 체제가 거대해질수록 그 붕괴로 인한 결과도 더욱 참혹해진다. 그러니 이 체제가 어차피 붕괴될 것이라면, 그 시기는 빠르면 빠를수록  좋다. 


4. 그런 이유에서 우리는 산업 체제에 항거하는 혁명을 주장한다. 이 혁명에선 폭력을 사용할 수도 있고  사용하지 않을 수도 있다. 혁명은 어느날 갑자기 일어날 수도 있고, 수십 년에 걸쳐 점진적으로 일어날  수도 있다. 우리가 예측할 수 있는 것은 아무 것도 없다. 다만 우리는 산업 체제를 증오하는 사람들이 체제에 항거하는 혁명의 길을 준비하는데 필요한 수단들을 아주 대략적인 수준에서 제시할 수는 있다. 이 혁명은 결코 정치적 혁명이 되어서는 안 된다. 혁명의 목표는 정부를 제거하는 것이 아니라, 현존 사회의  경제적, 기술적 토대를 제거하는 것이어야 한다. 


5. 이 글에서 우리는 산업기술사회에서 비롯된 부정적인 측면들의 일부에 대해서만 관심을 둘 것이다.  다른 측면들은 간단히 언급만 하거나 아니면 전부 생략한다. 이것은 우리가 그런 측면들을 중요하지  않다고 여기는 것이 아니다. 현실적인 이유 때문에 우리는 논의의 범위를 공공의 관심을 충분히 받지 못한 부분이나, 아니면 새롭게 논의되어지는 부분으로 한정시켜야한다. 예를 들면, 이미 잘 조직된 환경보호 운동들이 있기 때문에 우리는 환경, 야생자연의 파괴에 대해서는 거의 논하지 않을 것이다. 심지어 그것들이 매우 중요하다고 생각하면서도 말이다. 


\section*{현대 좌파의 심리} 
6. 우리 사회가 심각하게 병들었다는 사실에 대해서는 거의 모든 사람들이 동의할 것이다. 우리 세계가  품고 있는 광기 중에서 가장 광범위하게 드러난 광기가 바로 좌파 이념(leftism)이다. 좌파의 심리를 먼저  검토하는 것은, 현대 사회가 지닌 문제를 개괄하는데 그것이 길잡이 역할을 해 주리라고 생각하기  때문이다.


7. 그런데 좌파란 무엇인가? 20세기 전반에 좌파는 보통 사회주의로 여겨졌다. 오늘날의 좌파는  분산되어 있어 누굴 좌파라 해야할지 분명하지 않다. 우리가 좌파라고 할 때는 보통 사회주의자,  집단주의자, 정치적 올바름, 페미니스트, 동성애자와 장애인 권리 운동가, 동물보호 운동가 등을  의미한다. 하지만 저런 운동을 한다고 해서 반드시 좌파가 되는건 아니다. 우리가 좌파에 대해서 논할  
때는 운동이나 이념, 또는 연관된 집단에 대해 말하는 것이 아니다. 따라서, "좌파"가 무엇인지는 좌파의  심리에 대해 논하는 과정에서 더 분명하게 드러날 것이다.(문단 227${\sim}$230을 참고할 것.)  


8. 그래도 여전히 좌파에 대한 우리의 개념은 아주 모호한 상태로 남게 될 것이다. 하지만 여기에  대해서는 더 이상 어떤 해결책이 있을 것 같지도 않다. 우리가 여기서 시도할 수 있는 것은, 현대 좌파를  이끌어 가는 주된 원동력이라고 생각되는 두 가지 심리적 경향을 대강이나마 밝혀내는 정도다. 물론 우리가 좌파의 심리에 관한 모든 진실을 밝혀 낼 수는 없다. 또한 우리의 논의는 오로지 현대 좌파에만  국한된다. 19세기와 20세기 초기 좌파에 까지 우리의 논의가 적용될 수 있을지는 좀 더 생각해 보아야 할 문제이다. 


9. 현대 좌파의 저변에 깔려있는 두 가지 심리는 `열등감(feelings of inferiority)'과 `과잉 사회화 (oversocialization)'이다. 열등감은 현대 좌파 전반에 깔려있는 특성인 반면, 과잉 사회화는 일부의  특성이다. 하지만 그 일부는 대단히 영향력 있다. 


\section*{열등감} 
10. 우리가 `열등감'이라고 할 때 그것은 문자 그대로 열등감만을 뜻하는 것이 아니라, 자기 비하, 무력감, 비관주의, 패배주의, 죄책감, 자기 혐오 등과 같이 열등감과 관계 있는 모든 속성을 포괄적으로 뜻하는  것이다. 우리가 볼 때 현대의 좌파들은 (억압된 정도의 차이는 있겠지만)그런 감정들을 지니고 있으며,  그런 감정들이 현대 좌파의 방향을 결정한다.  


11. 누군가가 자신 또는 자신이 속한 집단을 칭하는 거의 모든 단어를 비하적 표현으로 여긴다면, 우린 그 사람이 열등감이나 낮은 자존감을 갖고 있다고 본다. 이 경향은 (본인이 자신이 지키고자 하는 소수자  집단에 포함되있던 아니던 간에)소수자 권리 운동가들 사이에 만연하다. 그들은 소수자를 지칭하는  단어에 대단히 민감하다. 아프리카인, 동양인, 장애인, 여성을 뜻하던 "흑인종(negro)", "동방인 (oriental)", "불구자(handicapped)", "계집(chick)"이라는 단어에는 원래 비하적 의미가 없었다. `계집'이라는 단어는 그저 남성의 `사내'에 해당하는 여성형 단어였을 뿐이다. 이런 표현에 비하적 의미를  부여한 것은 다른 사람이 아니라 운동가 그 자신들이다. 어떤 동물 보호 운동가들은 지나치게 멀리간  나머지 "애완동물(pet)"이라는 단어를 거부하고 "반려동물(animal companion)"을 대신 써야한다고  주장한다. 좌성향 인류학자들은 원시인들이 부정적으로 여겨질 수 있는 표현을 피하기 위해 대단히 많은  노력을 기울인다. 그들은 "원시(primitive)"를 "비문자(nonliterate)"로 교체하려고 한다. 그들은  원시문화가 자칫 우리의 문화보다 열등한 것으로 여겨질까봐 편집증적 행태를 보인다.(우리가 이런 말을  했다고 해서, 정말로 원시문화가 열등한 것이라고 주장하려는 것은 아니다. 우린 그저 좌성향 인류학자들이 얼마나 예민한지 지적했을 뿐이다.)  


12. “정치적으로 올바르지 못한” 단어에 민감하게 반응하는 이들은 흑인 빈민, 동양인 이민자, 학대당한  여성, 장애인들이 아니라 활동가들이다. 이 활동가들은 대부분 그들이 `억압당했다'고 주장하는 집단에  속해있지도 않다. 이 활동가들의 대부분은 사회적 특권층에 속해있다. 정치적 올바름은 안정된 직장과  충분한 월급이 보장된 대학교수, 그 중에서도 백인-이성애자-남성-중상류층에 속한 사람들 사이에서  확고하게 자리잡았다. 


13. 많은 좌파들은 약하고(여성), 패배했고(아메리카 원주민), 미움당한(동성애자) 이미지를 갖고 있는  집단에 스스로를 동일시한다. 좌파들 스스로가 이 집단들이 열등하다고 느낀다. 좌파 스스로는 절대  자신들이 그런 감정을 갖고 있지 않다고 주장할 테지만, 분명히 이런 집단을 열등하다고 보고 있기에  자신의 문제점을 이런 집단에 투영하는 것이다.(우리가 이런 말을 했다고 해서 정말로 여성, 인디언들이  열등하다고 주장하려는 것은 아니다. 우린 그저 좌파들의 심리상태를 지적했을 뿐이다.)


14. 페미니스트들은 온 힘을 다해 여성이 남성만큼이나 강하고 유능하다는 것을 증명하려 한다. 분명,  그들은 어쩌면 정말로 여성이 남성보다 못할지도 모른다는 공포감에 사로잡혀있다. 


15. 좌파들은 강하고, 좋고, 성공한 이미지를 지닌 것이라면 무엇이든 증오하는 경향을 보인다. 그들은  미국을 증오하고, 서구 문명을 증오하고, 백인 남성을 증오하고, 합리성을 증오한다. 좌파들이 말하는  자신들이 서구와 그에 비롯된 것들을 증오하는 이유와, 그들이 실제로 그것을 증오하는 동기는 분명히  다르다. 그들은 자신들이 서구가 호전적이고, 제국주의적이고, 성차별적이고, 자국민 중심적이기 때문에  서구를 증오한다고 말한다. 하지만 사회주의 국가나 원시 문화권에서 똑같은 결함이 나타날 경우, 좌파는  그들을 위한 변명거리를 찾아내고, 기껏해야 그런 결함이 존재한다는 것을 마지못해 인정할 뿐이다. 서구 문명에서 그런 결함이 보일 때는 열광적으로 (때로는 엄청나게 과장하며)지적하면서 말이다. 그러니  좌파가 미국과 서구를 증오하는 진짜 동기는 그 같은 결함 때문이 아니라는 것이 분명하다. 좌파가 미국과 서구를 증오하는 이유는, 바로 미국과 서구가 강하고 성공했기 때문이다. 


16. `자신감', `독립성', `자주성', `진취성', `낙관주의' 등의 단어들은 진보주의자, 좌파의 사전에서는  거의 쓰이지 않는다. 좌파는 반-개인적이고, 친-집단적이다. 좌파는 사회가 모든 사람들의 문제를 해결해 주고, 모든 사람들의 욕구를 채워 주고, 모든 사람들을 보살펴 주기를 바란다. 그는 자신의 문제를 스스로  해결할 수 있으며 자신의 욕구를 스스로 채울 수 있다는, 내적 자신감이 없는 인간이다. 좌파가  경쟁이라는 개념을 거부하는 것은, 그가 마음 속 깊이 스스로를 패배자로 여기고 있기 때문이다. 


17. 좌파 지식인들 사이에서 호응을 얻고 있는 예술 작품들은 흔히 더러움, 패배, 절망 등에 초점을  맞추고 있다. 또는, 합리적 계산에 의해서는 아무 것도 이룰 수 없으며, 그저 할 수 있는 것이라곤 순간적인 감각에 자신을 던져버리는 것뿐이라는 식으로, 이성적 통제의 가능성을 아예 포기하고 온통  미친 짓거리로 가득 찬 난장판만을 보여줄 뿐이다. 


18. 현대의 좌파 철학자들은 이성, 과학, 객관적 현실을 부정하고, 모든 것이 문화상대적이라고 주장한다. 물론 누구나 과학적 지식의 토대에 대해, 그리고 객관적 현실이라는 개념을 어떻게 정의할 것이냐고  진지한 질문을 던질 수 있다. 그러나 현대의 좌파 철학자들이 지식의 토대를 체계적으로 분석하고자 하는  냉철한 논리학자들이 아니라는 것만은 분명하다. 진리와 현실을 공격할 때, 그들은 감정적으로 심한 흥분  상태에 빠져 있다. 그들이 이 개념들을 공격하는 것은 자신들의 심리적 욕구 때문이다. 그들의 공격은  그들이 지닌 적개심의 표출이며, 공격이 성공할 때 그들의 권력욕도 충족된다. 더욱이 좌파들은 과학과  합리성을 증오한다. 과학과 합리성에 의해 참된 신념들(즉 성공한 것, 우월한 것)과 거짓 신념(즉 실패한  것, 열등한 것)이 구분되기 때문이다. 좌파의 열등감은 점점 깊어져 어떤 것을 성공한 것이나 우월한  것으로, 그리고 나머지를 실패한 것이나 열등한 것으로 구분하는 것조차도 참을 수 없게 된다. 많은  좌파들이 정신질환이라는 개념을 거부하고, IQ측정의 유용성을 거부하는하는 배경에는 이러한 심리가  깔려 있다. 좌파들은 인간의 능력, 행동을 유전적으로 설명하는 것에 대해서도 반대한다. 그런 설명이  어떤 사람들을 다른 사람들에 비해 우월하거나, 열등한 것으로 보이도록 만들기 때문이다. 좌파들은  개인의 유능함이나 무능함을 사회의 탓으로 돌리는 설명을 선호한다. 따라서 어떤 사람이 `열등'하다면,  그것은 그의 잘못이 아니라 사회의 잘못이다. 사회가 그를 올바르게 양육하지 않은 것이다.  


19. 좌파는 열등감 때문에 거만을 떨거나, 이기주의자가 되거나, 잘난척하거나, 깡패, 무자비한 경쟁자가  되지 않는다. 이 사람들은 스스로에 대한 믿음을 완전히 잃어버리지는 않았다. 자신이 나약함을 알고,  자부심도 약하지만, 그는 여전히 자신에게 강자가 될 능력이 있다고 믿는다. 그래서 스스로를 강자로  만들기 위해 노력하다 보니 그런 불쾌한 행동들이 나오는 것이다.\hyperlink{2}{$^{2}$} 하지만 좌파는 거기에서 한참을 더  나간다. 그의 열등감은 너무나 깊어서 그는 스스로가 강하고 가치 있다는 생각을 할 수 없다. 여기서  좌파의 집단주의가 생겨난다. 그는 자신을 동일화시킬 수 있는 거대 조직, 또는 대규모 사회 운동의  일원이 되었을 때에야 비로소 자신이 강하다고 느낄 수 있다.  


20. 좌파의 전략이 지닌 피학(被虐)적 성향에 주목하라. 좌파들은 자동차 앞에 누워서 저항하는가 하면,  경찰이나 인종주의자들을 일부러 자극해 자신들을 학대하도록 유도한다. 가끔씩 그런 전략이 효과적일 
수도 있다. 하지만 대부분의 좌파는 어떤 목적을 이루기 위한 수단으로서가 아니라, 그저 그것이 `좋아서' 그런 피학적 전략을 사용한다. 자기혐오는 좌파의 특징이다. 


21. 좌파들은 자신들의 운동이 동정심 또는 윤리적 원칙이라는 동기에 의한 것이라고 주장할 수도 있다.  그리고 과잉 사회화된 좌파의 운동에서 윤리적 원칙이 일정한 역할을 수행하는 것도 사실이다. 하지만  동정심과 윤리적 원칙은 결코 좌파 운동의 주된 동기가 아니다. 좌파의 행동에서는 호전성이 너무나  뚜렷하게 드러난다. 권력에 대한 욕망 역시 그렇다. 게다가 좌파의 행동 상당 부분은, 좌파들이 돕고자  하는 사람들의 이익을 위해 합리적으로 계산된 것도 아니다. 예를 들어보자. 흑인들을 위한 적극적  우대조치(affirmative action)가 흑인을 위한 것이라고 믿는다면, 호전적이고 교조적인 용어들을  사용하며 요구하는게 적절한가? 적극적 우대조치를 얻어내려면, 적극적 우대조치가 자신들을 차별한다고 여기는 백인들로부터 그저 빈말로라도, 또 상징적으로나마 양보를 얻어낼 수 있도록 조금 더 외교적이고  타협적인 접근 방법을 취해야 할 것이다. 하지만 좌파 활동가들은 그런 접근 방법을 취하지 않는다. 그런  접근 방법으로는 자신들의 감정적 욕구를 충족시킬 수 없기 때문이다. 그들의 진정한 목표는 흑인을 돕는  것이 아니다. 반대로 인종 문제는 좌파가 자신들의 적대감과 좌절된 권력 욕구를 표출하기 위한 좋은  핑계거리다. 그렇게 좌파는 오히려 흑인들에게 피해를 입힌다. 활동가들이 다수 백인에게 보이는 적대적  태도로 인해 인종 간의 증오가 증폭되기 때문이다. 


22. 만일 우리 사회에 문제가 없다면, 좌파들은 분쟁을 일으킬 핑계를 얻기 위해 문제들을 발명해 낼 것이다. 


23. 강조하지만, 앞에서 한 설명은 결코 좌파에 대한 정확한 설명이 아니다. 다만 좌파의 일반적 성향이  그렇다는 것을 포괄적으로 지적했을 뿐이다. 


\section*{과잉 사회화} 
24. 심리학자들은 `사회화'라는 단어를 어린이들이 사회의 요구에 부합하는 방향으로 말하고 생각하도록  훈련시키는 과정을 지칭하기 위해 사용한다. 누군가가 사회의 도덕률을 잘 따르고 사회의 일부로서  적절히 기능할 때, 그가 잘 사회화되었다고 한다. 보통 좌파들은 반항아로 여겨지기에, 많은 좌파들이  과잉 사회화되었다고 말하는건 언뜻 보기에 말이 안되는 것 같지만, 충분히 근거있는 이야기다. 대부분의  좌파들은 반항아가 아니다. 


25. 우리 사회의 도덕률은 너무나 많은 것을 요구하기에, 거의 어느 누구도 완벽하게 도덕적인 방식으로  생각하고, 느끼고, 활동할 수 없다. 예를들어, 사회는 우리에게 누군가를 미워해선 안된다고 가르친다.  그러나 당사자가 그걸 인정하는지와는 상관없이, 거의 대부분의 사람들이 어느 순간, 어느 장소에서  누군가를 미워한다. 어떤 사람들은 너무 심하게 사회화되어 있어, 이런 방향으로 생각하고, 말하고,  행동하는 것 자체가 그들에게 심각한 짐이 된다. 죄책감을 피하기 위해, 그들은 끊임없이 자신의 동기를  속여야하며, 자신의 비도덕적 동기에서 비롯된 감정이나 행동을 도덕적인 것으로 포장하기 위한 핑계를  찾는다. 우리는 이런 사람들을 두고 "과잉 사회화 되었다"고 말한다.\hyperlink{3}{$^{3}$} 


26. 과잉 사회화는 자기비하, 무력감, 패배주의, 죄책감 등을 유발할 수 있다. 우리 사회가 어린이를  사회화할 때 사용하는 가장 중요한 수단은, 어린이가 사회의 기대에 맞지 않는 말과 행동을 보일 때  창피를 주는 것이다. 이것이 지나칠 경우, 또는 어떤 어린이가 수치심에 유난히 예민한 성격을 지니고  있을 경우, 그 어린이는 스스로를 수치스러워하게 된다. 게다가 과잉 사회화된 사람은 항상 사회의 시선을 의식하기 때문에, 생각하고 행동하는 데 있어서 가볍게 사회화된 사람들보다 훨씬 심한 제약을 받는다.  대부분의 사람들은 못된 행동을 밥먹듯이 저지른다. 거짓말, 좀도둑질, 교통법규 위반, 근무 태만, 타인에  대한 증오, 악담, 남을 앞지르기위한 교묘한 속임수... 과잉 사회화된 사람은 그런 행동을 할 수가 없다.  설령 어쩌다 그런 행동을 저지른다 해도, 이번엔 스스로 만들어 낸 수치심과 자기혐오에 빠지게 된다.  과잉 사회화된 사람은 기존의 윤리에 어긋나는 것을 현실에서가 아니라, 단지 생각과 감정으로 경험하는  경우에조차 죄책감에 빠진다. 그는 “더러운” 생각을 해서는 안 되는 것이다. 그리고 윤리가 사회화의  전부는 아니다. 우리는 갖가지 비윤리적인 행동들에 대해서도 순응하도록 사회화된다. 결국 과잉
사회화된 사람은 끝까지 윤리라는 심리적 쇠사슬을 벗어날 수 없으며, 사회가 그에게 제시한 윤리적 삶을  따라 평생을 지낸다. 과잉 사회화된 사람들 대부분은 그 결과, 구속감과 무력감이라는 심각한 고통 속에서 살아가게 된다. 우리는 과잉 사회화야 말로 인간이 타인에게 저지를 수 있는 그 어떤 잔혹 행위보다도  무서운 잔혹 행위라고 생각한다. 


27. 현대 좌파 내에서 대단히 중요하고 큰 영향력을 지닌 한 분파가 과잉 사회화되어 있으며, 그들의 과잉 사회화에 의해 현대 좌파의 방향이 결정된다고 우리는 주장한다. 과잉 사회화된 좌파들은 흔히 지식인  또는 중상류층 출신들이다. 대학 지식인들\hyperlink{4}{$^{4}$}이 우리 사회에서 가장 고도로 사회화된 분파를 구성함과  아울러 대부분의 좌익세력을 이루고 있음을 주목하라. 


28. 과잉 사회화된 좌파는 자신의 심리적 구속감을 벗어 던지려 애쓰며, 저항을 통해 자신의 자율성을  내세우려 한다. 그러나 대부분의 경우, 그는 사회의 가장 기본적인 가치에 저항할만한 힘을 갖고 있지  않다. 다시 말해, 현대 좌파의 목표는 기존 윤리와 충돌하지 않는다. 오히려 좌파는 기존의 윤리적  원칙들을 받아들이고, 그것을 자신의 것으로 수용한 다음, 주류 사회가 그 원칙을 위반한다고 비난한다.  인종 평등, 성 평등, 가난한 사람들을 돕는 것, 전쟁반대, 비폭력 운동, 언론의 자유, 동물에 대한 사랑 등이 그런 윤리적 원칙들이다. 보다 근본적인 원칙으로는 사회를 위해 봉사해야하는 개인의 의무와,  개인을 보호해야하는 사회의 의무도 있다. 이 모든 원칙들은 우리 사회에서(아니면 적어도 중상류층\hyperlink{5}{$^{5}$} 사이에서) 오랫동안 깊게 뿌리내려 온 가치들이다. 주류 언론과 교육 시스템에 의해 우리에게 전달되는  자료들 대부분에는 그런 가치들이 명시적으로 표현되어 있거나 암묵적으로 전제되어 있다. 좌파, 특히  과잉 사회화된 좌파들은 그 원칙들에 대해 저항하지 않는다. 대신에 사회가 그 원칙들을 지키지 않는다고  주장함으로써(이는 어느 정도 사실이다) 사회에 대한 자신들의 적대감을 정당화한다. 


29. 과잉 사회화된 좌파들이 실제로는 우리 사회의 전통적인 가치를 따르면서도, 마치 사회에 저항하는  것처럼 포장하는 사례를 보이겠다. 많은 좌파들이 흑인들이 좋은 직업을 주고, 흑인들의 교육에 더 많은  돈을 투자하도록 하는 적극적 우대조치(affirmative action)을 요구한다. 그렇게 흑인들이 좌파들  스스로가 수치스럽게 여기는 "하층"으로부터 벗어날 수 있게 하려고 말이다. 좌파들은 흑인들을 체제의  일부로 만들고자 한다. 흑인들을 마치 중상류층 백인처럼 기업가, 변호사, 과학자로 만들려고한다. 여기에 좌파들은 자신들은 결코 흑인을 백인의 복사본으로 만들려고 하는게 아니라고 반박한다. 그 증거로  자신들은 아프리카계 미국인의 문화를 보존하고자 한다고 말한다. 그러나 아프리카계 미국인의 문화를  보존한다는게 무엇인가? 그저 흑인들의 음식, 흑인들의 음악, 흑인들의 패션, 흑인들의 교회와 모스크를  보존하는 것에 불과하다. 다시 말해서, 흑인 문화의 보존이란건 결국 이렇게 피상적일 수 밖에 없는  것이다. 모든 중요한 측면에서 봤을 때, 과잉 사회화된 좌파들은 흑인들이 백인들의 이상향에 복종하도록  만들려고 한다. 좌파들은 흑인이 이공계 과목을 전공하고, 기업의 오너 또는 과학자가 되어, 평생을 계층 사다리를 올라 흑인들도 백인만큼이나 뛰어나다는것을 증명하길 원한다. 좌파들은 흑인 아버지가 "책임감" 있고, 흑인 깡패들이 비폭력적으로 변하길 원한다. 하지만 이런 것들은 정확히 산업-기술 체제가 원하는 가치이다. 이 체제는 열심히 공부하고, 좋은 직업을 갖고, 계층 사다리를 오르고, "책임감" 있는 부모가 되기만 한다면 그 사람이 어떤 음악을 듣든, 어떤 옷을 입고 어떤 종교를 갖든 신경 쓰지 않는다.  사실상, 과잉 사회화된 좌파들은 그들이 얼마나 부정하던, 결국엔 흑인들을 체제에 통합시키고 체제의 가치에 복종시키려는 것이다. 


30. 우리는 과잉 사회화된 좌파를 포함해서, 좌파들이 우리 사회의 근본적 가치들에 대해 절대로  저항하지 않는다고는 주장하지는 않는다. 분명히 그들도 때로는 저항한다. 일부 과잉 사회화된 좌파는  심지어 물리적 폭력을 사용함으로써 현대 사회의 가장 중요한 원칙들에 저항하기도 한다. 그들의 설명에  따르면, 그들에게 있어 폭력은 `해방'의 한 형태다. 다시 말하자면, 폭력을 저지름으로써 자신들에게  내재화된 심리적 구속을 깨뜨리고 있다는 것이다. 과잉 사회화된 탓에 그들은 다른 사람들보다 훤씬 심한  심리적 구속에 얽매여 있다. 이 지점에서 구속에서 벗어나려는 그들의 욕구가 생겨난다. 그런데 좌파들은  흔히 주류 가치를 담고 있는 용어들을 이용해 자신들의 저항을 정당화한다. 가령 폭력을 행사할 경우,  그들은 인종차별 같은 것들에 저항해 싸우고 있노라고 주장하는 것이다.


31. 우리는 엉성한 스케치 수준에 불과한 좌파의 심리에 관한 우리의 설명에 반론의 여지가 많다는  사실을 잘 알고 있다. 실제의 상황은 매우 복합적이며, 그것을 완벽하게 설명하려면, 설령 필요한 자료를  다 구할 수 있다고 해도, 몇권의 책이 필요할 것이다. 우리는 다만 현대 좌파의 심리 안에 두 가지 매우  중요한 성향이 자리잡고 있음을 극히 대략적이나마 지적하려 했을 뿐이다. 


32. 좌파의 문제는 우리 사회의 문제를 고스란히 보여 주고 있다. 낮은 자존감, 비관적, 패배적 성향은 은  좌파에만 한정된 것이 아니다. 좌파에게서 유난히 눈에 띈다는 것뿐이지, 이런 성향은 우리 사회에  광범위하게 퍼져 있다. 그리고 오늘날의 사회는 과거 어느 사회보다도 더욱 철저하게 우리를  사회화시키려 한다. 심지어 사회는 무얼 먹을지, 어떻게 운동할지, 어떻게 연애할지, 어떻게 자녀를  키울지에 대해서까지 전문가들을 통해 지시하려 든다. 


\section*{권력 과정\hyperlink{6}{$^{6}$}} 
33. 인간은 우리가 앞으로 `권력 과정'이라 부르는 어떤 것에 대한 (아마 생물학적인)욕구를 갖고 있다. 이 욕구는 (널리 알려진 대로)권력에의 욕구와 밀접히 관련되어 있기는 하지만, 그것과 동일한 욕구는  아니다. 권력 과정은 네 개의 요소들로 이루어진다. 그 중에서 명확하게 구분할 수 있는 세 구성 요소를  우리는 목표, 노력, 목표 달성이라고 부른다. (모든 사람에겐 노력을 통해 달성할 수 있는 목표가  필요하다. 목표를 최소한 부분적으로라도 달성해야 한다.) 네 번째 요소는 정확히 규정하기 어렵고, 모든 사람에게 필수적인 것이 아닐 수도 있다. 우리는 그것을 자율성이라고 부르며, 이에 대해서는 나중에  거론할 것이다.(문단 42${\sim}$44) 


34. 원하기만 하면 무엇이든 가질 수 있는 사람이 있다고 가정해 보자. 그 사람은 권력을 갖고 있음에도  불구하고 점차 심각한 심리적 문제들을 키워가게 될 것이다. 처음에야 모든 것이 신나겠지만, 그것도  잠깐, 그는 급속히 권태에 빠지고 타락해 간다. 그러다 마침내는 병적인 우울증에 빠지게 될 것이다.  역사는 유한(有閑) 귀족들이 대체로 퇴폐하는 경향이 있음을 보여준다. 물론 권력을 유지하기 위해  투쟁해야 하는 전투적 귀족들에겐 해당되지 않는다. 하지만 열심히 노력해야 할 일이 아무 것도 없는  안정된 유한 귀족들은 권력을 쥐고 있어도 보통 권태에 빠지고, 쾌락에 탐닉하다가 결국 타락해 버린다.  여기서 우리가 알 수 있는 사실은 권력만으로는 충분치 않다는 것이다. 사람에겐 권력을 행사하기 위한  목표가 필수적인 것이다. 


35. 누구에게나 목표가 있다. 다른 목표가 전혀 없다 해도, 음식과 물, 그리고 기후에 따라 필요한 옷과  주거지 등 생필품을 구한다는 목표는 남아 있다. 그런데 유한 귀족은 이런 것들을 아무런 노력없이  획득한다. 여기에서 그의 권태와 타락이 생겨난다. 


36. 중요한 목표를 달성하지 못했을 때, 만약 그 목표에 목숨이 걸려있다면 그 실패의 결과는 죽음이다.  목숨과 상관없는 목표라면 그 결과는 좌절이다. 평생을 계속해서 목표 달성에 실패하면, 패배주의, 자기  비하, 우울증에 빠지게 된다. 


37. 따라서 심각한 심리적 문제를 피하기 위해, 인간에게는 노력을 쏟아 달성할 수 있는 목표들이  필요하며, 그 목표들을 납득 가능한 비율로 달성할 수 있어야 한다. 


\section*{대리 활동} 
38. 하지만 모든 유한 귀족이 권태에 빠지고 타락하는 것은 아니다. 한 예로 일본 천황 히로히토는  퇴폐적인 쾌락에 빠지는 대신, 해양생물학 연구에 전념했고 그 분야에서 이름을 남기기도 했다. 신체적  욕구를 채우기 위해 자신을 내던질 필요가 없을 때, 사람들은 흔히 인위적 목표를 세운다. 대개의 경우  사람들은 생필품을 구하기 위해 했던 것과 똑같은 에너지를 쏟아 가며, 자신의 전부를 바쳐 목표 달성을  추구한다. 그런 식으로 로마 제국의 귀족들은 문학에 열정을 쏟았고, 몇 세기 전의 유럽 귀족들은 고기에  대한 욕구가 전혀 없었음에도 사냥에 엄청난 시간과 에너지를 쏟아 부었던 것이다. 어떤 귀족들은 정교한 
방식으로 부를 과시함으로써 신분 경쟁을 계속했다. 그리고 히로히토 같은 소수의 귀족들은 과학으로  눈을 돌리기도 했다. 


39. 우리가 “대리 활동”이라고 할 때, 그것은 사람들이 단순히 어떤 목표를 갖기 위해, 또는 목표를  추구하는 과정에서 `충족감'을 얻기 위해 만들어낸 인위적 목표를 지향하는 활동을 뜻한다. 어떤 활동이  대리 활동인지 알아보는 간단한 기준은 다음과 같다. ‘X’라는 목표를 추구하는 데 상당한 시간과 에너지를 모두 쏟아 붓는 사람이 있다고 치고, 다음과 같은 질문을 생각해 보자. 만약 그 사람이 대부분의 시간과  에너지를 자신의 생물학적 욕구를 충족시키는 데 바쳐야만 했다면, 그리고 그런 생물학적 욕구를  충족시키기 위해 노력하는 과정에서 그가 자신의 신체적, 정신적 능력을 여러 가지 다양하고 흥미로운  방법으로 활용했다면, 목표 X를 달성하는데 실패했을 때 그가 심각한 박탈감을 느끼겠는가? 만약 대답이  ‘아니오’라면, 그 때 목표 X를 추구하는 것은 대리 활동이 된다. 히로히토의 해양생물학 연구는 분명히  하나의 대리 활동이 된다. 히로히토가 생필품을 얻기 위해 과학과 무관한 일에 모든 시간을 바쳤다 해도,  그가 그 일에서 재미를 느꼈다면, 그는 자신이 해양생물의 해부학과 습성을 모른다는 이유로 박탈감을  느끼지는 않으리라는 것은 틀림없다. 반면에 (예를 들어) 섹스나 사랑을 추구하는 것은 대리 활동이  아니다. 대부분의 사람들은, 이성과의 성적 관계를 전혀 가지지 않고 평생을 보내야 한다면, 설령 그  문제만 제외하고는 다른 모든 것이 만족스럽다 해도, 여전히 박탈감을 느낄 것이기 때문이다. (그러나  자신에게 정말로 필요한 정도를 넘어서 과도하게 섹스를 추구하는 것은 역시 대리 활동이 될 수 있다.) 


40. 현대 산업사회에서 신체적 욕구를 충족시키는 데에는 최소한의 노력만 있으면 된다. 간단한 기술을  습득하기 위해서는 일련의 훈련 과정을 거치는 것으로 충분하며, 직업을 계속 가지고 있으려면 그저 제  시간에 출근하고 적당히 열심히 일하면 되는 것이다. 필요한 것은 적당한 지능, 그리고 무엇보다도,  단순한 복종이다. 사회는 그런 사람들을 요람에서 무덤까지 보살펴 준다. (물론 생필품조차도 보장받지  못하는 최하층 사람들이 있다. 그러나 여기서 우리가 말하고 있는 것은 주류 사회이다.) 그러니 현대  사회가 대리 활동으로 가득차 있다고 해도 전혀 놀랄 일이 아니다. 여기에는 과학적 연구작업, 스포츠 기록 경쟁, 인도주의 활동, 예술 및 문학 창작, 회사 내에서의 직위 상승, 신체적 필요를 한참 넘어서는  돈과 물질적 재화의 획득, 그리고 비백인 소수자들의 권리를 위하는 백인 활동가처럼 활동가 자신에게는  별로 중요하지 않은 이슈를 떠들어대는 사회 운동이 포함된다. 이런 활동들이 언제나 순수한 대리  활동으로 머물지는 않는다. 많은 사람들에게 있어 단순히 추구해야 할 목표가 필요하다는 욕구 이외의  다른 욕구들이 활동의 동기가 될 수 있는 것이다. 과학 연구는 부분적으로 특권층이 되고자 하는 욕망에  의해 유발될 수도 있고, 예술 창작은 감정을 표현하려는 욕구에 의해서, 민병대 운동은 적대감에 의해서  유발될 수도 있다. 그러나 그런 활동을 추구하는 대부분의 사람들에게 그 활동의 상당 부분은 대리  활동이다. 예를 들어, 대부분의 과학자들은 아마 자신들의 연구를 통해 얻는 “만족감”이 돈이나 특권보다  더 중요하다는 것에 동의할 것이다. 


41. 대부분은 아니더라도, 많은 사람들에게 있어 대리 활동은 진짜 목표(즉, 권력 과정에 대한 욕구가  이미 채워졌을 경우에도 여전히 달성하기를 원하는 목표)를 추구하는 것에 비해 만족감이 떨어진다. 이는  대리 활동에 깊이 몰두해 있는 대부분의 사람들이 결코 만족할줄 모르며, 결코 멈추지 않는다는 사실을  통해 분명히 드러난다. 그래서 돈에 눈이 먼 사람은 끝없이 더 많은 부를 향해 질주하는 것이다. 과학자는  문제를 해결하자마자 곧바로 다음 문제로 달려간다. 장거리 달리기 선수는 언제나 보다 멀리, 보다 빨리  달리기 위해 자신을 몰아친다. 대리 활동을 추구하는 사람들은 자신들이 생물학적 욕구를 충족시키기  위한 “지루한” 활동에서 얻는 것보다 훨씬 더 큰 만족감을 그런 활동에서 얻는다고 말할 것이다. 그러나  그것은 우리 사회가 생물학적 욕구를 채우는 데 필요한 노력을 아주 미미한 수준으로 축소했기 때문에  가능해진 일이다. 보다 중요한 사실은 우리 사회에서 사람들이 자신의 생물학적 욕구를 자율적으로  만족시키지 못하고, 그 대신 거대한 사회적 기계의 부품으로서 기능함으로써 만족시킨다는 사실이다.  반대로 사람들은 자신의 대리 활동을 추구하는 데 있어서는 엄청난 자율권을 누리고 있다. 


\section*{자율성} 
42. 권력 과정의 한 부분으로서의 자율성은 모든 사람에게 필수적인 것이 아닐 수도 있다. 그러나  대부분의 사람들은 목표를 향해 노력할 때, 정도의 차이는 있을지라도 자율성이 필요하다. 사람들의 
노력은 스스로가 주도권을 쥐고 있는 가운데 이루어져야 하며, 또한 스스로 정한 방향과 통제를 따라  이루어져야 한다. 하지만 모든 사람이 한 사람의 개인으로서 이 같은 주도권과 방향 결정권, 통제권을  반드시 행사해야 할 필요는 없다. 대부분의 경우 작은 집단의 구성원으로 행동하는 것만으로도 충분하다.  따라서 만약 여섯 명의 사람들이 공동의 목표를 설정하고 그 목표를 달성하는데 서로의 노력을  성공적으로 결합시킨다면, 권력 과정에 대한 욕구가 채워질 수 있다. 그러나 만약 그들이 일체의 자율적인 결정권과 주도권을 허용치 않는 엄격한 상부의 명령에 따라 일한다면, 그들이 지닌 권력 과정의 욕구는  채워지지 않는다. 집단적인 결정 방식을 채택할 경우에 그 집단이 너무 커서 각 개인의 역할이  무의미해지는 집단에서도 이와 똑같은 일이 벌어진다.\hyperlink{7}{$^{7}$} 


43. 어떤 개인들은 자율성에 대한 욕구가 별로 없어 보이기도 한다. 권력욕이 원래 약하거나, 아니면  자신이 속한 강력한 조직과 자신을 동일시함으로써 권력욕을 충족시키거나, 둘 중 하나일 것이다. 한편 오직 물리적인 권력에만 만족하는 거의 짐승 수준의 바보들도 있다.(뛰어난 군인이 그 사례다. 그는  상관에게 맹목적으로 복종하며 전투력을 향상시키는데서 자신이 권력을 쥐고 있다는 느낌을 얻는다.) 


44. 하지만 대부분의 사람들은 목표를 설정하고, 자율적인 노력을 기울이며, 목표를 달성하는 권력  과정을 통해서 자존심과 자신감, 권력을 쥐고 있다는 느낌을 얻게된다. 어떤 사람이 이 권력 과정에  참여할 정당한 기회를 갖지 못했을 때, 그 결과는 (개인에 따라, 그리고 권력 과정이 붕괴되는 방식에 따라 달라지지만) 권태, 타락, 자기 비하, 열등감, 패배주의, 절망, 불안, 죄책감, 좌절, 적대감, 배우자 또는  자녀 학대, 탐욕스러운 쾌락주의, 변태성욕, 불면증, 폭식증 또는 거식증 등으로 나타난다.\hyperlink{8}{$^{8}$} 


\section*{사회 문제들의 근원} 
45. 앞에 열거한 증상들은 어느 사회에서나 생겨날 수 있다. 하지만 현대 산업사회에서는 이런 증상들이  대규모로 만연해 있다. 오늘날 세상이 점점 미쳐가고 있음을 지적한 것은 우리가 처음이 아니다. 이런  것들은 인간 사회에서 정상적인 것이 아니다. 원시 인류가 스트레스와 좌절로 인한 고통을 적게 겪었으며, 현대인보다 자신의 인생에 더 만족했으리라고 믿는 데에는 충분한 근거가 있다. 물론 원시 사회라고 해서  모든 것이 편안하고 쾌적하지는 않았을 것이다. 호주 원주민 부족에서는 여성학대가 흔하게 벌어지고  있으며, 일부 아메리카 인디언 부족에서는 성 전환을 심심치 않게 볼 수 있다. 그러나 일반적으로 말해서,  우리가 앞 문단에서 열거했던 것과 같은 문제들은 분명 현대 사회에서처럼 흔하게 나타나는 문제는  아니었다. 


46. 현대 사회의 사회적, 심리적 문제들이 발생하는 이유는, 현대 사회가 지금까지 인류가 진화해 왔던 환경과는 근본적으로 다른 환경 속에서 살아가도록 사람들에게 요구하고 있기 때문에, 또 과거 환경에서  살면서 발전시켜 온 행동 양식과 전혀 다른 방식으로 행동하기를 요구하고 있기 때문이라고 우리는  생각한다. 이미 언급했듯이 현대 사회가 사람들에게 강요하는 온갖 비정상적인 환경 중에서도 가장  중요한 문제는 권력 과정에 제대로 참여할 기회가 부족하다는 점이다. 그것만이 사회 문제의 유일한  근원은 물론 아니다. 사회 문제의 한 근원으로서의 권력 과정 붕괴를 다루기에 앞서 우리는 몇 가지 다른  근원을 이야기하려고 한다. 


47. 현대 산업사회가 안고 있는 비정상적 환경으로는 과도한 인구 밀도, 자연으로부터의 인간 소외,  지나치게 빠른 사회 변화, 대가족이나 마을, 부족 등과 같은 자연스러운 소규모 공동체의 붕괴 등을 들 수  있다. 


48. 인구 과밀이 스트레스와 공격성을 증가시킨다는 사실은 널리 알려져 있다. 오늘날의 인구 과밀과  자연으로부터의 인간 소외는 기술 발전의 결과로 빚어진 현상이다. 산업화 이전의 대부분의 사회는 농경  사회였다. 산업혁명으로 인해 도시의 규모는 엄청나게 커졌으며, 도시에 사는 인구 역시 엄청나게  늘어났다. 한편 현대적 농업 기술 덕분에 지구는 이제껏 한번도 겪어 보지 못한 과밀한 인구를 먹여 살릴  수 있게되었다. (또한 기술로 인해 인구 과밀의 부정적 효과는 더욱 강화되었는데, 이는 기술이 사람들의  손에 더욱 더 강한 파괴력을 쥐어 주었기 때문이다. 전동 잔디깎이, 라디오, 오토바이 등등 저 수많은  종류의 소음 발생 장치들을 생각해 보라. 만일 이런 기계들을 규제하지 않는다면, 평화와 고요를 바라는 
사람들은 소음으로 인해 좌절에 빠지게 될 것이다. 반면 기계들을 규제한다면, 이번에는 기계를 사용하고  싶어하는 사람들이 규제로 인해 좌절에 빠지게 될 것이다. 하지만 만약 이런 기계들이 애초에 발명되지  않았더라면 아무런 갈등도 좌절도 생겨나지 않았을 것이다.) 


49. 원시 사회에는 (아주 서서히 변화할 뿐인) 자연이 안정적인 생활기준을 제공해 주었고, 사람들은  안정감을 얻을 수 있었다. 반대로 현대 사회에서는 인간 사회가 자연을 지배하고 있으며, 기술의 변화에  따라 현대 사회는 급속히 변화하고 있다. 안정적인 생활기준이 존재하지 않는 것이다. 


50. 보수주의자들은 바보다. 그들은 전통적 가치들이 망가지고 있다고 불평하면서도, 기술의 발전과 경제 성장을 열광적으로 지지한다. 한 사회의 기술과 경제의 급격한 변화는 사회의 다른 측면들의 급격한 변화  없이는 이루어질 수 없으며, 그런 급격한 변화가 필연적으로 전통적 가치들을 붕괴시킨다는 자명한  사실을 보수주의자들은 꿈에도 생각하지 못한다. 


51. 전통적 가치의 붕괴는 전통적인 소규모 사회 집단을 묶어 주고 있는 유대 관계가 붕괴된다는 것을  의미하게 마련이다. 현대의 환경이 개인들로 하여금 자신들이 속한 공동체로부터 떨어져 나와 새로운  지역으로 옮겨가도록 요구하거나 유혹한다는 점도 소규모 사회 집단의 와해를 촉진한다. 그런 이유를  제쳐놓더라도, 우선 기술 사회가 효율적으로 기능을 수행하기 위해서는 가족 간의 유대와 지역 공동체를  
약화시켜야만 한다. 현대 사회에서 개인은 체제에 최우선으로 충성해야 하며, 소규모 공동체에 대한  충성은 부차적인 것일 수밖에 없다. 소규모 공동체의 내부적 충성도가 체제에 대한 충성도보다 더 강력할  경우, 그 공동체는 체제를 희생시키면서 자신들의 이익을 추구할 것이기 때문이다. 


52. 고위 공무원이나 기업 임원이 어떤 자리에 능력있는 사람이 아닌, 자신의 사촌이나 친구, 같은 종교의 신도를 앉힌다고 생각해 보자. 그는 체제에 대한 충성보다 사적인 충성을 우선시한 것이고, 이는 곧 `족벌주의' 또는 `차별'이 되는데, 둘 다 현대 사회에서는 끔찍한 범죄 행위다. 사적 또는 지역적 충성을  체제의 충성에 완전히 굴복시키지 못한 채 산업사회를 지향하는 사회는 대부분 몹시 비효율적이다. (라틴 아메리카를 보라.) 따라서 선진 산업사회는 오로지 거세되고, 길들여진, 그리고 체제의 도구가 되어버린 소규모 공동체들만을 허용할 수 있다.\hyperlink{9}{$^{9}$} 


53. 인구 과밀, 급격한 변화와 공동체의 붕괴는 그 동안 사회 문제의 근원으로 널리 인정되어 왔다.  그러나 우리는 그것들만으로는 오늘날 볼 수 있는 광범위한 사회 문제를 설명하는데 충분하지 않다고  믿는다. 


54. 산업화 이전의 도시들 중 몇몇 도시는 매우 큰 규모와 높은 인구 밀도를 지니고 있었다. 하지만 그  도시의 주민들이 현대인들처럼 심리적 문제들 때문에 고통을 겪었던 것 같지는 않다. 오늘날에도  미국에는 여전히 인구 밀도가 낮은 농촌들이 남아 있으며, 우리는 그런 농촌들에서도 도시에서 일어나는  것과 동일한 문제들을 발견한다. 물론 농촌 지역에서는 문제들이 상대적으로 덜하다는 차이는 있다.\hyperlink{10}{$^{10}$} 따라서 인구 과밀이 사회 문제의 결정적인 요인이라고 볼 수는 없을 것이다. 


55. 미국 개척자들이 서부로 나아가던 19세기, 당시에도 인구의 이동으로 말미암아 대가족과 소규모  사회 집단은 붕괴되었다. 그 붕괴의 정도는 오늘날에 비해 결코 약하지 않았다. 사실, 수 마일에 걸쳐  이웃이 전혀 없으니 도저히 공동체에 소속될 방법이 없는 고립된 상황에서, 많은 사람들은 자발적으로  핵가족을 선택했다.\hyperlink{11}{$^{11}$} 하지만 그렇다고 개척민들이 공동체의 붕괴 때문에 사회 문제들을 겪었던 것  같지는 않다. 


56. 아울러 미국 개척자 사회에서의 사회 변화는 매우 빠르고 심오하게 이루어졌다. 법률과 질서의  손길이 미치지 않는 상황에서, 통나무집에서 태어나고 자라며, 야생 동물의 고기를 주식으로 먹으며  성장한 사람도 많을 것이다. 그러다가 노인이 되어서야 그는 제대로 된 직장에서 일하게 되고, 법률적  제재가 효력을 발휘하는 질서 잡힌 공동체에서 살게 되었을 것이다. 이런 식의 사회 변화는 전형적인  현대인의 삶에서 벌어질 수 있는 것보다 훨씬 더 심오한 변화이다. 그럼에도 불구하고, 그런 사회 변화가 
사람들에게 심리적 문제들을 일으켰던 것으로는 보이지 않는다. 오히여 19세기 미국 사회에는  오늘날과는 달리 낙관적이고 자신만만한 분위기가 펴져 있었다.\hyperlink{12}{$^{12}$}


57. 우리가 볼 때 두 사회가 다른 점은, 19세기의 개척자들은 자신의 선택에 따라 스스로 변화를 만들어  냈다고 느낀 반면(이는 상당히 정당한 논리다.), 현대인은 변화가 자신에게 강요되었다고 느낀다는(이  역시 상당히 정당한 논리다.)점이다. 개척자는 자신이 스스로 선택한 땅에 정착해 자신의 노력을 통해 그  땅을 농장으로 만들어냈다. 그 시절의 어느 군(County)에는 군을 통틀어 고작 수백 명의 거주자만이  있었고, 현대의 군보다 훨씬 더 고립되고 자율적인 집단을 이룰 수 있었다. 그랬기 때문에, 개척자는  비교적 작은 집단의 일원으로서 질서 잡힌 새로운 공동체를 창조하는 작업에 참여할 수 있었다. 이 같은  공동체의 창조를 과연 진보로 볼 수 있을 것인가 하는 의문을 제기할 수도 있겠지만, 어쨌든 개척자는  그런 공동체 창조 작업을 통해서 권력 과정에 대한 욕구를 충족시킬 수 있었다. 


58. 급격한 사회 변화와 공동체의 친밀한 유대 관계의 상실을 겪으면서도, 현대 산업사회에서 나타나는  광범위한 이상 행동을 보이지 않는 사회는 그 밖에도 얼마든지 찾을 수 있다. 사람들에게 권력 과정을  정상적인 방법으로 통과할 수 있는 기회가 충분히 주어지지 않는다는 것. 우리는 현대 사회의 사회적,  심리적 문제들의 가장 중요한 원인은 바로 이것이라고 주장한다. 현대 사회에서만 권력 과정이  붕괴되었다는 뜻은 아니다. 전부는 아닐지라도 대부분의 문명 사회에서 권력 과정은 심하게든 약하게든 항상 방해를 받아 왔다. 그런데 유독 현대 사회에서는 권력 과정의 붕괴라는 문제가 특히 극심하게  벌어지는 것이다. 좌파, 적어도 최근(20세기 중반부터 후기까지)의 좌파는 권력 과정이 붕괴되면서  나타난 박탈 증상이라고 볼 수 있다. 


\section*{현대 사회에서 권력 과정의 붕괴} 
59. 우리는 인간의 욕망을 세 부류로 나눈다. (1)최소한의 노력으로 충족시킬 수 있는 욕망, (2)상당한  노력을 치러야만 충족시킬 수 있는 욕망, (3)아무리 노력해도 충족시킬 수 없는 욕망. 권력 과정은 그 중에  두번째 부류의 욕망을 충족시키는 과정이다. 세번째 부류의 욕망이 많으면 많을수록 좌절과 분노, 그리고  궁극적으로 패배주의와 절망도 더 빈번히 생겨나게 된다. 


60. 현대 산업사회에서 자연적인 인간의 욕망은 주로 첫번째와 세번째 부류에 속하며, 두번째 부류의  욕망은 점점 더 인위적으로 형성된 욕망들로 이루어지고 있다. 


61. 원시 사회에서 생활 필수품은 대개 두번째 욕망에 속한다. 필수품을 구할 수는 있지만, 거기엔 상당한 노력이 반드시 필요하다. 하지만 현대 사회에선 최소한의 노력만으로도 누구나 생활 필수품을 얻는 것이  보장되어 있고\hyperlink{13}{$^{13}$}, 따라서 신체적 욕구는 첫번째 부류의 욕망으로 전환된다. (직업을 확보하는 데 필요한  노력이 정말 `최소한의' 노력인지에 대해서는 논란의 여지가 있다. 그러나 중급 이하 수준의 직업에서  요구되는 노력은 기껏해야 복종하는 노력 정도에 불과하다. 서라면 서고, 앉으라면 앉고, 시킨대로만 하면 되는 것이다. 몸과 마음을 다해 노력해야하는 경우는 거의 없으며, 작업에서 자율성을 발휘할 수 있는  기회는 거의 없다. 그러니 권력 과정에 대한 욕구가 제대로 충족되지 않는 것이다.) 


62. 현대 사회에서의 섹스, 사랑, 신분 등과 같은 사회적 욕구는 개인이 처한 상황에 따라 다르지만, 흔히  두번째 부류에 머문다.\hyperlink{14}{$^{14}$} 그러나 특별히 강력한 신분 상승의 욕망을 지닌 사람들을 제외하면, 사회적  욕구를 총족시키는 데 필요한 노력은 권력 과정에 대한 욕구를 제대로 충족시키기에는 역부족이다. 


63. 그래서 두번째 부류에 속하는 인위적 욕구들이 만들어지고, 사람들은 그런 인위적 욕구를 채움으로써 권력 과정에 대한 욕구를 대신 채운다. 진보한 광고나 마케팅 기법은 사람들로 하여금 그들의 할아버지  세대는 결코 욕망하지 않았을, 아니면 아예 꿈도 꾸지 않았을, 그런 것들이 필요하다고 느끼도록 만들고  있다. 그런 인위적 욕구들을 충족시키려면 많은 돈을 벌어야 하고, 돈을 벌기 위해선 또 막대한 노력이  필요하고, 따라서 인위적 욕구는 두번째 부류에 속한다. (그러나 문단 80${\sim}$82를 볼 것.) 현대인은 광고와  마케팅 산업에 의해 만들어진 인위적 욕구를 추구하면서\hyperlink{15}{$^{15}$}, 그리고 대리 활동을 통해서, 권력 과정에 대한  욕구를 충족시켜 주어야만 한다.


64. 대부분의 사람들에게 권력 과정에 대한 인위적 욕구만으로는 충분치 않다. 20세기 전반부의 사회  비평서에서 계속 반복된 주제는 현대 사회 대다수의 사람들을 괴롭히는 목적 상실감이었다(이 목적  상실감은 흔히 `무규범 상태', 또는 `공허한 중산층' 등 다른 이름으로 불리기도 했다.). 우리가 보기엔 이른바 `정체성의 위기'라는 것도 사실은 목적을 찾기 위한, 그리고 흔히 적절한 대리 활동에 몰두할  방법을 찾기 위한 모색 작업이라고 볼 수 있다. 실존주의는 현대의 삶에 목적이 없음에 대한 반발로 나온 것이라고 할 수 있다.\hyperlink{16}{$^{16}$} 현대 사회에서 `만족'을 위한 모색 활동은 대단히 널리 퍼져 있다. 그러나 우리는  만족감이 목표인 활동(즉 대리 활동)을 통해서 만족감을 완벽하게 충족시킬 수 있는 사람은 별로  없으리라고 생각한다. 달리 말해, 대리 활동은 권력에의 욕구를 온전히 충족시켜 주지를 못한다.(문단 41을 볼 것). 권력 과정에 대한 욕구는 생활 필수품, 섹스, 사랑, 신분, 복수 등과 같이 외적인 목표를 지니고  있는 활동을 통해서만 비로소 완벽히 충족될 수 있다. 


65. 아울러 돈을 벌거나 신분 상승의 사다리를 오르거나, 기타 다른 방식으로 체제의 부속품 역할을  수행하면서 목표를 추구할 때, 대개의 사람들은 자율적으로 자신의 목표를 추구할 만한 지위에 있지 않다. 대부분의 노동자들은 우리가 문단 61에서 지적했듯이 피고용자의 신분이며, 그저 누군가가 시킨 대로  따라 하는 것으로 하루하루를 보내야만 한다. 자영업자들의 경우에도 역시 자율성은 제한적이다.  중소기업의 사장들은 언제나 정부의 과도한 규제가 자신들의 손발을 묶어 놓고 있다는 불만을 토로하고  있다. 이 같은 규제들 중 일부가 불필요하다는 점에는 의심의 여지가 없다. 하지만 대부분의 사회  부문에서 정부 규제는 극단적으로 복잡한 우리 사회에서는 필수적이고 또 불가피한 것이다. 오늘날  소규모 기업의 상당 부분은 프랜차이즈 제도에 따라 운영되고 있다. 몇 년 전 \textlangle{}월 스트리트 저널\textrangle{}지는,  많은 프랜차이즈 제공 회사들이 프랜차이즈 희망자들에게 특별한 인성 검사를 받도록 요구하고 있다는  기사를 실었다. 기사에 따르면, 그 검사의 목적은 창의적이고 주도적인 사람들을 제외하기 위한 것으로서, 그런 사람들은 프랜차이즈 제도를 군소리 없이 따라갈 만큼 고분고분하지 않기 때문이라는 것이었다.\hyperlink{17}{$^{17}$} 따라서 자율성이 강한 많은 사람들은 소규모 사업에서 제외될 수밖에 없다.


66. 오늘날 사람들은 자신을 위해 스스로 행한 무엇보다는, 체제가 그들을 위해서, 아니면 그들에게 행한  무언가에 의지해서 살아가고 있다. 그리고 그들이 스스로를 위해 무엇인가를 할 때에도 점점 더 체제가  마련해 놓은 통로를 따라가고 있다. 기회란 것도 체제에 의해 제공되며, 기회를 잡기 위해서는 규칙과  규제\hyperlink{18}{$^{18}$}에 순응해야 한다. 그리고 성공을 위해서는 전문가들이 미리 정해 놓은 기법을 따라야만 한다. 


67. 우리 사회에서 권력 과정은 진정한 목표의 부재, 그리고 목표 추구에서의 자율성의 부재로 인해  붕괴되고 있다. 권력 과정이 붕괴되는 것은 세번째 부류에 속하는 욕망들 때문이기도 하다. 제아무리  발버둥쳐도 도저히 충족시킬 수 없는 그런 욕망들 말이다. 이제 우리의 삶은 다른 사람이 내리는 결정에  좌우된다. 우리는 그런 결정에 전혀 관여할 수 없으며, 그런 결정을 내리는 사람들이 누구인지도 모른다.  ("우리는 아마 500명에서 1,000명 정도의, 세계 인구와 비할 때 거의 없는 것이나 마찬가지인 숫자의  사람들이 모든 중요한 결정을 내리는 세계에서 살고 있다." - 필립 헤이만, 하버드대 법학 교수, \textlangle{}뉴욕 타임즈\textrangle{} 1995.4.21자에서 안토니 루이스가 인용.\hyperlink{19}{$^{19}$}) 우리의 삶은 핵발전소에서 안전 기준이 제대로  지켜지고 있는지의 여부에 좌우된다. 얼마나 많은 농약이 음식에 들어가도록 허용되는지. 대기에 오염 물질이 얼마나 허용되는지, 의사가 얼마나 숙련된 사람인지(또는 무능한지)에 따라 우리의 삶이  좌지우지되는 것이다. 직업을 잃고 얻는 것도 정부의 경제 부처나 기업 임원들이 어떤 결정을 내리느냐에  달려 있다. 대부분의 사람들은 그 같은 위협들로부터 자신을 지킬 수 있는 지위를 확보하지 못하고 있다.  그러니 자신의 안전을 지키고자 하는 개인들의 노력은 좌절될 수 밖에 없고, 그것은 곧 무력감으로  이어진다. 


68. 짧은 평균 수명에서도 볼 수 있듯이, 원시인은 현대인보다 신체적으로는 더 불안한 삶을 살았다는  반론이 제기될 수도 있을 것이다. 그러니 현대인에게 고통을 주는 불안함의 총합은 인간에게 통상적으로  주어지는 불안함보다 적으면 적었지, 더 크지는 않을 것이라는 반론이다. 그러나 심리적 안정은 신체적  안정과 정확히 일치하지 않는다. 우리를 안전하다고 느끼게 만드는 것은, 스스로를 보호할 수 있는 능력에 대한 자신감과 같이, 별로 객관적이라고 할 수 없는 보호 장치다. 맹수와 배고픔의 위협에 둘려싸여  있지만, 원시인은 자기 방어를 위해 싸울 수 있고 음식을 찾아 먼 길을 여행할 수도 있다. 원시인에겐
노력한 만큼 성공하리라는 보장도 없다. 하지만 원시인은 자신을 위협하는 것들에 대해 가만히 앉아서  당하지는 않는다. 그런데 원시인과 달리 현대인은 자신이 도저히 어떻게 해 볼 도리가 없는 것들에 의해  위협을 당하고 있는 것이다. 원자력 사고, 식품 속의 발암 물질, 환경 공해, 전쟁, 늘어나는 세금, 거대  조직에 의한 사생활 침해, 인생을 엉망으로 만들어 버릴 전국적인 사회 현상 또는 경제 현상 등이 그런  위협들이다. 


69. 물론 원시인 역시 자신을 위협하는 어떤 것들에 대해서는 무력하다. 가령 질병을 그런 위협의 예로 들 수 있다. 그러나 원시인은 질병의 위험조차도 냉철하게 받아들일 수 있다. 질병은 사물 본성의 한  부분이었으며, 상상 속의 악마\hyperlink{20}{$^{20}$}의 잘못이 아니라면 누구의 잘못도 아니다. 하지만 현대의 개인에게  가해지는 위협들은 인간이 만든 것이 대부분이다. 그 위협들은 우연의 결과가 아니라, 개인은 도저히  영향을 끼칠 수 없는 타인들의 결정에 의해 개인에게 강요된 위협이다. 그러니 그가 좌절과 모욕을  느끼고, 분노하는 것은 당연한 일이다. 


70. 원시인이 대개의 경우 자기 손(개인으로서건, 아니건 작은 집단의 일원으로서건)으로 자신의 안전을  지킬 수 있는 반면, 현대인의 안전은, 너무나 멀리 떨어져 있거나 너무나 거대해서 개인적으로 도저히  아무런 영향력도 행사할 수 없는 타인들이나 조직의 손에 맡겨져 있다. 그러니 현대인의 안전에 대한  욕망은 첫번째와 세번째 부류의 욕망으로 전락해 버린다. (음식이나 주거지 같은) 일부 영역에서는 아주  조금만 노력해도 안전이 보장될 수 있다. 하지만 그 밖의 영역에서는 전혀 안전을 보장받을 수 없다. (지금까지의 논의는 현실 상황을 극단적으로 단순화 시킨 것이다. 하지만 현대인의 환경이 원시인의  환경과 어떻게 다른지를 개략적으로 보여주는데는 충분할 것이다.) 


71. 현대 생활에서 사람들이 지닌 부질없는 욕망 중 상당 부분은 어쩔 수 없이 좌절되게 마련이고, 그  결과 세번째 부류의 욕망으로 전락한다. 여기에 분노를 느끼는 사람이 생기겠지만, 현대 사회에서는  싸움이 허용되지 않는다. 심지어 많은 경우에 말을 심하게 하는 것조차 허용되지 않는다. 어디론가 급히  가야 할 때도 있고, 아니면 천천히 움직이고 싶을 때도 있을 것이다. 하지만 교통 흐름을 따르고 교통  신호를 지키며 움직이는 것 말고는 다른 방법이 없다. 다른 방식으로 작업해 보고 싶을 수도 있을 것이다.  하지만 대개는 고용주가 정한 규칙을 따라야만 한다. 현대인은 그런 식으로 (명시적이건 암묵적이건)  규칙과 규제의 그물에 꽁꽁 묶에 있으며, 그 규칙과 규제들은 현대인의 욕망을 좌절시키고, 그 결과 권력  과정을 교란시킨다. 이 규제들 대부분은 산업 사회의 기능을 수행하는 데 필수적이므로, 그것 없이는  살아갈 수 없다. 


72. 어떤 면에서 현대 사회는 극도로 자유 방임적이다. 체제의 기능 수행과 무관한 사안에 대해서는  우리는 거의 뭐든지 하고 싶은대로 할 수 있다. (그 종교가 체제를 위협하는 행동을 장려하지 않는 한)  우리는 어느 종교든 마음대로 믿을 수 있다. (`안전한' 섹스를 실천하는 한) 우리는 누구라도 마음에 드는  상대와 잠자리를 같이 할 수 있다. 우리는 그것이 중요하지 않은 것이라면 무엇이든 하고 싶은 대로 할 수  있는 것이다. 하지만 모든 중요한 사안에서 체제는 갈수록 우리의 행동에 대한 규제를 강화하고 있다. 


73. 명시적인 법률과 정부에 의해서만 행동에 대한 규제가 이루어지는 것은 아니다. 통제는 간접적인  강요를 통해서, 심리적 억압 또는 조작을 통해서, 그리고 정부보다는 조직에 의해서 이루어진다. 대부분의 대형 조직들은 대중의 태도나 행동을 조작하기 위해 어떤 형태의 프로파간다\hyperlink{21}{$^{21}$}를 사용한다. 프로파간다는 `상업 광고'와 홍보에 국한되지 않으며, 때로는 그것을 만드는 사람들조차도 자신들이 프로파간다를  만든다는 의식을 갖고 있지 않을 때도 있다. 예를 들어 오락 프로그램의 내용은 강력한 프로파간다 형식 중의 하나다. 간접적인 강요의 한 예를 들어보자. 우리가 매일 직장에 출근해야 하고 고용주의 명령에  따라야 한다고 써있는 법률은 없다. 우리가 원시인들처럼 대자연 속에 살아가는 것, 또는 우리 자신을  위한 사업에 뛰어드는 것을 금지하는 법률은 없다. 그러나 현실적으로 자연은 거의 남아 있질 않으며,  소규모 자영업자를 위한 경제적 공간은 극히 제한되어 있다. 그러니 우리들 대부분에게 생존할 수 있는  방법은 단 하나, 다른 사람의 피고용자가 되는 것 뿐이다. 


74. 우리는 현대인이 지닌 장수(長壽)에 대한 강박 관념, 그리고 노년에 이르러서도 여전히 육체적 힘과  성적 매력을 유지해야 한다는 강박 관념이 바로 권력 과정에의 욕구 충족이 박탈당한데서 초래된 병적 
증상이라고 본다. `중년의 위기' 또한 그 같은 병적 증상이다. 현대 사회에서 일반화된 출산에 대한  무관심은 원시 사회에서는 거의 볼 수 없는 현상이다.  


75. 원시 사회에서 삶은 일련의 단계들이 지속되는 것이었다. 어느 한 단계의 욕구와 목적이 충족되면  곧장 다음 단계로 망설이지 않고 넘어갈 수 있었다. 젊은 남성은 사냥꾼이 됨으로써 권력 과정을  통과했으며, 이 때 사냥은 스포츠나 만족감을 위한 것이 아니라, 식량으로 필요한 고기를 얻기 위한  것이었다.(젊은 여성의 경우 그 과정은 좀 더 복잡하다. 우리의 논의는 사회적 권력을 중심으로 한  것이므로, 여성에 관해서는 언급하지 않겠다.) 그 단계를 성공적으로 통과하면, 청년은 망설임 없이  곧바로 가정을 이루는 책임을 떠맡았다. (대조적으로 현대인들은 수많은 `만족감'을 채우느라 너무나  바쁜 탓에 자녀 갖기를 끝없이 미루고 있다. 우리가 보기에 그들에게 정말로 필요한 것은 대리 활동이라는 인위적 목표의 달성에서 오는 만족감이 아니라, 정당한 권력 과정을 경험하는 일이다.) 자녀를 제대로  기르고, 자녀들에게 생활 필수품을 공급하면서 권력 과정을 통과하고 나면, 원시인은 자신이 할 일을  다했다고 느끼며 (그 때까지 그가 살아 있다면) 노년기\hyperlink{22}{$^{22}$}와 죽음을 받아들일 준비를 한다. 반대로 많은  현대인은 그들이 육체적 컨디션과 외모, 건강을 지키기 위해 쏟아 붓는 엄청난 노력에서 볼 수 있듯이,  육체적 쇠락과 죽음 앞에서 전전긍긍하고 있다. 현대인은 단 한번도 실용적인 목적으로 자신의 육체적  힘을 사용하지 못하며, 권력 과정을 통과하는데 있어 한번도 자신의 육체를 제대로 사용하지 못한다.  그럼으로써 현대인의 욕구는 결코 채워지지 않으며, 그것이 노화와 죽음에 대한 두려움을 낳는 것이다.  실용적인 목적을 위해 매일 자신의 육체를 사용한 원시인은 노화(老化)를 두려워하지 않았다. 그런데  자동차에서 집까지 걷는 일 말고는 일체 자신의 육체를 사용하지 않는 현대인이 오히려 노화를  두려워하고 있다. 인생을 통해서 권력 과정에 대한 욕구를 충족시킬 수 있었던 사람만이 인생의 종말을  두려움 없이 받아들일 수 있는 것이다. 


76. 어떤 사람은 우리의 주장에 대해 이렇게 말할 것이다. "사회는 사람들에게 권력 과정을 통과하기 위한 기회를 제공해 줄 방법을 찾아낼 것이 틀림없다." 하지만 문제는 사회가 기회를 제공해 준다는 바로 그  사실 때문에 기회의 가치가 훼손된다는 점이다. 사람들에게 필요한 것은 스스로 기회를 찾거나 만들어  내는 것이다. 체제가 사람들에게 기회를 제공해 주는 한, 사람들은 여전히 체제의 사슬에 묶일 수 밖에  없다. 자율성을 획득하기 위해서는 그 사슬을 벗어 던져야 한다. 


\section*{일부 사람들이 체제에 적응하는 방법} 
77. 산업-기술 사회의 모든 사람이 심리적 문제로 인한 고통을 겪고 있는 것은 아니다. 일부 사람들은  현재의 사회에 만족하고 있다. 이제 왜 사람들이 현대 사회에 대해 그렇게도 확연히 다른 반응을 보이는지 그 이유를 살펴보기로 하자. 


78. 우선, 의심할 바 없이 사람마다 권력욕이 다르다. 권력욕이 약한 개인들은 상대적으로 권력 과정을  통과해야할 필요성이 적을 수도 있다. 아니면 적어도 권력 과정에서 상대적으로 자율성이 덜 필요할 수도  있다. 이들은 `옛날 남부(Old South)'의 농장에서 일하던 흑인 노예들처럼 고분고분한 타입으로, 쉽게  행복해 질 수 있는 사람들이다. (옛날 남부의 `흑인 노예'를 비웃으려는 것이 아니다. 대부분의 노예들은  자신의 노예 생활에 만족하지 않았다. 우리가 비웃는 것은 노예 생활에 만족하는 부류들이다.)  


79. 어떤 사람들은 권력에의 욕망을 추구하면서 몇 가지 예외적인 욕망을 품게 될 수도 있다. 예를 들어,  신분 상승에 대해 예외적으로 강력한 욕망을 지닌 사람들은 신분 상승이라는 게임에 결코 싫증을 느끼지  않으며 평생을 그저 신분의 사다리를 오르며 보낼 수도 있다.  


80. 광고 및 마케팅 기법에 넘어가는 정도도 사람마다 다르다. 어떤 사람들은 그 정도가 너무나 심해,  아무리 많은 돈을 벌어도 광고 산업이 그들의 눈 앞에 대고 흔들어 대는 눈부신 신형 장난감을 향한  끝없는 갈망을 충족시킬 수가 없다. 그러니 그들의 수입이 엄청남에도 불구하고 언제나 경제적으로  궁핍함을 느끼며, 결국 그들의 갈망은 좌절로 이어진다.


81. 어떤 사람들은 광고 및 마케팅 기법에 잘 넘어가지 않는다. 이런 사람들은 돈에도 별로 관심이 없다.  물질을 얻는 것으로는 권력 과정에 대한 욕망이 채워지지 않는 부류의 사람들이다.  


82. 광고 및 마케팅 기법에 중간 정도로 넘어가는 사람들은 상품과 서비스를 향한 갈망을 채우기에  충분한 정도의 돈을 벌 수 있지만, 그러기 위해선 상당한 노력(시간 외 근무, 부업, 승진을 위한 음모 등)을 대가로 치러야 한다. 물질을 얻음으로써 권력 과정에 대한 욕망이 채워지는 것이다. 그러나 그렇다고 해서 그들의 욕망이 완전히 채워지는 것은 아니다. 권력 과정에서 자율성이 부족할 수도 있고(그들의 노동이  명령을 따르는 것으로만 이루어져 있을 수도 있다.) 그들이 지닌 욕망 중 일부는 좌절될 수도 있다.(안전,  공격성 등). (문단 80${\sim}$82에서 우리가 물질을 얻고자 하는 욕망이 순전히 광고 및 마케팅 산업에 의해  만들어진 것이라고 가정한 까닭에 문제가 지나치게 단순화되어 버린 것은 우리의 잘못이다. 물론 문제는  그렇게 단순하지 않다.)  


83. 어떤 사람들은 강력한 조직이나 대규모 운동과 자신을 동일화시킴으로써 권력 과정에 대한 욕망을  부분적으로나마 충족시킨다. 목표나 힘을 지니지 못한 개인들은 운동이나 조직에 참여해 운동과 조직의  목표를 자신의 목표로 받아들이고 그 목표를 추구하며 일한다. 이들 목표의 일부가 이루어질 경우, 개인은 자신의 노력은 목표 달성에 아주 미미한 힘을 보탰음에도 불구하고, (운동 또는 조직과 자신을  동일시함으로써) 권력 과정을 통과한 듯한 느낌을 얻게 된다. 파시스트와 나치, 그리고 공산주의자들은  이러한 현상을 악용했다. 우리 사회 역시 조금 덜 노골적이긴 하지만 그것을 이용하고 있다. 파나마의  독재자 노리에가는 미국에게는 눈엣가시 같은 존재였다.(목표:노리에가 처벌), 미국은 파나마를 침공했고 (노력), 노리에가를 처벌했다(목표의 달성). 미국은 권력 과정을 통과했고, 많은 미국인들은 자신을  미국과 동일시했기에 권력 과정을 대리 경험했다. 광범위한 대중이 파나마 침공을 용인한 것은 그  때문이었다. 즉, 파나마 침공은 사람들에게 권력을 갖고 있다는 느낌\hyperlink{23}{$^{23}$}을 제공해 주었던 것이다. 군대,  기업, 정당, 인권단체, 종교 운동, 이념 운동에서도 똑같은 현상이 벌어진다. 특히 좌파 운동들은 권력  욕구를 충족시킬 방법을 찾고 있는 사람들에게 매력적으로 다가온다. 하지만 대개의 사람들은 거대 조직 또는 대규모 운동에 대한 동일화를 통해서는 권력 욕구를 완전히 충족시키지 못한다. 


84. 사람들이 권력 과정에 대한 욕구를 충족시키는 또 하나의 방법은 대리 활동이다. 문단 38${\sim}$40에서  설명했듯이, 대리 활동은 인위적 목표를 추구하는 활동이다. 개인은 목표 자체의 달성이 필요해서가  아니라, 그저 목표를 추구하면서 얻어지는 `만족감'을 위해서 그런 활동에 매달리는 것이다. 예를 들어,  보디빌딩을 통해 우람한 근육을 키우는 것이나, 조그만 공을 구멍으로 집어넣는 것, 기념 우표 세트를  전부 모으는 것 등에는 아무런 실용적인 동기가 없다. 그런데도 우리 사회의 많은 사람들은 보디빌딩과  골프와 우표 수집에 혼신의 힘을 다하고 있다. 어떤 사람들은 다른 사람들에 비해 보다 더 `타인 지향적' 이다. 그런 사람들은 오로지 주변 사람들이 그것을 중요하게 생각한다는 이유만으로, 또는 사회가  중요하다고 말한다는 이유만으로, 대리 활동이 매우 중요한 일이라고 단정지어버린다. 어떤 사람들이  스포츠, 카드 게임, 체스, 학문적 비밀의 추구 등 깨인 의식을 지닌 사람들의 눈에는 그저 대리 활동에  불과한 활동, 즉 본질적으로 사소한 일에 필사적으로 매달리는 이유는 바로 그것이다. 그리고 그 결과,  정작 권력 과정에 대한 욕망을 충족시키는 일은 전혀 중요하다고 생각하지 않게 된다. 많은 경우에  생활비를 벌는 일조차도 역시 대리 활동이 된다. 생활 필수품을 얻고 높은 사회적 신분을 획득하고 광고에 의해 촉발된 사치품에의 욕구를 충족시키기 위한 활동은 물론 순수한 대리 활동이라고 할 수는 없다.  그러나 많은 사람들은 자신들에게 필요한 돈과 신분을 훨씬 초과하는 것을 얻기 위해 엄청난 노력을 쏟아  붓고 있으며, 바로 그 초과분의 노력이 대리 활동이 되는 것이다.이 초과분의 노력은 거기에 수반되는  정서적 투자와 함께, 개인의 자유에 대해서는 부정적인 영향을 미치면서, 체제의 영속적인 발전과 완성을  위한 가장 강력한 잠재력 중의 하나가 된다(문단 131을 볼 것). 특히 가장 창의적인 과학자와  엔지니어들에게조차, 작업은 대부분 대리 활동에 불과하다. 이는 특히 중요한 문제이므로 잠시 후에 따로  논의하기로 한다(문단 87${\sim}$92).  


85. 여기서 우리는 현대 사회의 많은 사람들이 어떤 식으로 권력 과정에 대한 욕구를 충족시키는지  알아보았다. 그러나 우리는 대부분의 사람들이 권력 과정에 대한 욕구를 제대로 충족시키지 못하고  있다고 생각한다. 그 이유는 첫째, 끝없는 신분 상승에의 욕망을 지닌 사람들이나 대리 활동에 `코가 꿰인' 사람들, 권력욕을 충족시키기 위해 사회 운동이나 조직과 자신을 완전히 일체화하는 사람들은 오히려
예외적인 성격의 소유자들이라는 것이다. 그렇지 않은 나머지 사람들은 대리 활동이나 조직과의  일체화로는 결코 완전한 충족을 얻을 수 없다(문단 41과 64를 볼 것). 둘째, 명시적인 규제나 사회화를  통해서 체제가 너무나 강력한 통제력을 행사하고 있다. 그런 통제 때문에 자율성은 사라지고, 목표 달성이 불가능한 데다가 수많은 욕망을 강하게 억눌러야 하는데서 오는 좌절감은 늘어나는 것이다. 


86. 설령 산업-기술 사회에서 대부분의 사람들이 만족한다고 할지라도 우리는 여전히 그런 형태의 사회를 반대한다. (다른 이유도 있지만 무엇보다도) 우리는 진정한 목표의 추구를 통해서가 아니라, 대리  활동이나 조직과의 일체화를 통해서 권력 과정에 대한 욕구를 충족시키는 것을 천박한 행위라고 생각하기 때문이다.  


\section*{과학자의 동기} 
87. 과학과 기술은 가장 중요한 대리 활동 사례들이다. 어떤 과학자들은 자신들이 과학 연구에 몰두하는  이유가 "호기심"이나 "인류의 행복"을 향한 열망이라고 주장한다. 그러나 대부분의 과학자들에게 있어  이것이 주된 동기가 아니라는 사실은 쉽게 알아차릴 수 있다. "호기심"이라는 것은 그저 헛소리에  불과하다. 대부분의 과학자들은 정상적인 호기심의 대상이 아닌, 고도로 전문화된 문제들을 놓고  씨름한다. 가령, 천문학자나 수학자, 곤충학자가 아이소프로필트라이메틸메탄의 성분에 대해 호기심을  느끼겠는가? 물론 아니다. 그런 것에 호기심을 느낄 사람은 오직 화학자 뿐이며, 화학자가 그것에  호기심을 느끼는 것은 단 하나, 화학이 그의 대리 활동이기 때문이다. 화학자가 새로운 종의 딱정벌레를  정확하게 분류하는 일에 호기심을 느끼겠는가? 아니다. 그 질문에 흥미를 느끼는 것은 곤충학자 뿐이다.  그리고 그가 흥미를 느끼는 것 역시 곤충학이 그의 대리 활동이기 때문이다. 만약 화학자나 곤충학자가  생활 필수품을 얻는 데 전력을 다해야 하고, 그러한 노력을 하는 과정에서 그들의 능력을 과학과는  무관하지만, 흥미로운 방식으로 발휘할 수 있다면, 그들은 아이소프로필트라이메틸메탄이나 딱정벌레의  분류 따위에 대해서는 콧방귀도 뀌지 않을 것이다. 대학원의 연구 기금이 바닥난 화학자가 할 수 없이  보험 외판원이 되었다고 가정해 보자. 그렇다면 화학자는 이제 보험에 관련된 문제에는 대단한 흥미를  보이겠지만, 아이소프로필트메틸메탄에 대해서는 일체 관심도 두지 않을 것이다. 어쨌든 간에,  과학자들이 자신의 작업에 쏟아붓는 시간과 노력이 그저 단순히 호기심을 충족시키기 위한 것이라면  정상이 아니다. 과학자들의 동기를 "호기심"으로 설명하려는 시도에는 아무런 근거가 없다. 


88. "인류의 행복" 또한 "호기심"보다 별로 나을 것이 없는 설명이다. 어떤 과학적 연구는 인류의 행복과는 전혀 관련이 없다. 고고학이나 비교언어학 같은 것이 그 사례다. 몇 가지 다른 과학 분야는 아예 명백한  위험성을 갖고 있다. 그런데도 이 분야의 과학자들은 예방 백신 개발이나 대기 오염을 연구하는 학자들과  똑같이 자신의 연구에 열광하고 있다. 핵발전소 건설을 추진하는 데 온 마음을 다 바쳤던 에드워드 텔러(Edward Teller) 박사의 경우를 생각해 보자. 그가 그렇게 온 마음을 바친 것은 인류의 행복을 위한 열망 때문이었는가? 만약에 그렇다면, 텔러 박사는 왜 다른 `인도주의적' 문제들에 대해서는 전혀 흥분하지  않았는가? 그가 그렇게 인도주의자였다면, 도대체 왜 수소폭탄의 개발을 도왔던 것인가? 다른 수많은  과학적 업적과 마찬가지로, 핵발전소가 과연 인류의 행복을 위한 것이냐 하는 질문에는 서로 상이한  대답들이 나올 수 있다. 저렴한 전기가 쌓여 가는 핵 폐기물과 원자력 사고의 위험을 감수할 가치가 있을  정도로 중요한가? 텔러 박사는 질문의 오직 한 측면만을 보았다. 분명히 그가 핵발전소에 온 마음을 다  바친 것은 "인류의 행복"을 위한 열망 때문이 아니라, 자신의 연구에서 얻어지는, 그리고 연구 결과가 실제 사용되는 것을 보면서 얻어지는 개인적 만족감 때문이었다. 


89. 과학자들 일반에 대해서도 똑같은 진실이 통용된다. 희귀한 예외를 제외하곤, 그들의 동기는  호기심도 아니고 인류의 행복도 아니다. 진짜 동기는 권력 과정을 통과하고자 하는 욕구이다. 목표를 갖는 (풀어야 할 과학적 문제), 노력하는 것(연구), 그리고 목표를 달성하는 것(문제의 해결)이다. 과학이 대리  활동인 이유는 과학자들이 주로 연구 그 자체로부터 충족감을 얻기 위해 연구하고 있기 때문이다. 


90. 물론 문제는 그렇게 단순하지 않다. 많은 과학자들에게는 그 밖의 다른 동기들도 작용한다. 예를 들어 돈과 신분 같은 동기들이다. 어떤 과학자들은 신분 상승에 대해 끊없는 욕망을 지닌 사람들일 수도 있고 (문단 79를 볼 것), 이런 욕망이 그들의 연구에 상당한 동기를 부여할 수도 있다. 일반 대중과 마찬가지로 과학자들의 대다수 역시 정도의 차이는 있지만 광고 및 마케팅 기법에 현혹되며, 따라서 상품과 서비스에 대한 갈망을 충족시키기 위한 돈을 필요로 한다는 것은 의심할 여지가 없다. 따라서 과학을 무조건 대리  활동이라고 할 수는 없다. 그렇지만, 크게 봤을 때 과학은 역시 대리 활동에 속한다. 


91. 한편 과학과 기술은 거대한 권력을 갖고 있으며, 많은 과학자들은 이 거대한 운동에 자신을  일체화함으로써 권력 욕구를 충족시킨다. 


92. 따라서 과학은 과학자들과 연구 기금을 제공하는 정부 관료 및 기업 이사들이 지닌 심리적 욕구에만  복종하는 맹목적인 행진일 뿐이다. 그들은 인류의 진정한 행복이라든가 그 밖의 다른 사안에 대해서는  전혀 신경 쓰지 않는다.\hyperlink{24}{$^{24}$} 


\section*{자유의 본질} 
93. 우리는 산업-기술 사회가 인간의 자유를 서서히 침해하지 못하도록 개혁할 수 없음을 밝히려 한다.  그러나 "자유"란 단어는 다양한 의미로 해석될 수 있으므로, 우선 우리가 어떤 종류의 자유에 관심을 두고  있는지를 먼저 밝혀 두겠다. 


94. 우리가 말하는 "자유"란 권력 과정을 통과할 수 있는 기회를 뜻한다. 여기에는 대리 활동이라는  인위적 목표가 아니라 진정한 목표가 있어야하며 누구로부터도, 특히 어떤 거대 조직으로부터도, 일체의  간섭이나 조작 또는 감독이 있어서는 안된다. 자유란(개인으로건 아니면 작은 집단의 일원으로서건)  당사자의 존재가 걸린 삶과 죽음의 문제를 스스로 통제할 수 있음을 뜻한다. 음식, 옷, 주거지와 환경 내에 있을 수 있는 모든 종류의 위협에 대한 방어 능력 등이 그런 문제들이다. 자유란 권력을 확보하는 것을  뜻한다. 여기서 말하는 권력은 다른 사람을 통제하기 위한 권력이 아니라, 자신의 삶을 둘러싼 환경을  통제하기 위한 권력이다.\hyperlink{25}{$^{25}$} 누군가(특히 거대 조직이) 우리에게 권력을 행사할 때, 그 권력이 아무리  호의적, 관용적으로 행사되거나, 방임을 허락한다 해도, 우리에겐 자유가 없다. 자유와 체제가 허락하는  방임을 혼동하지 않는 것이 중요하다.(문단 72를 볼 것) 


95. 사람들은 우리가 헌법으로 보장된 몇 가지의 권리를 확보하고 있다는 이유를 들어 우리가 자유로운  사회에서 살고 있다고 말한다. 그러나 그런 헌법적 권리들은 별로 중요하지 않다. 한 사회 내에 존재하는  개인적 자유의 정도는 그 사회의 법률이나 정부 형태가 아니라, 그 사회가 지닌 경제적,기술적 구조에  의해 결정된다.\hyperlink{26}{$^{26}$} 뉴잉글랜드 지역에 있던 인디언 국가들은 대부분 왕국\hyperlink{27}{$^{27}$}이었다. 그리고 르네상스 시대의 이탈리아 도시 국가들 중 상당수는 독재자들의 지배를 받았다. 그러나 이들 사회를 가만히 들여다보면, 그 사회들이 우리 사회보다 훨씬 더 많은 개인적 자유를 허용했다는 사실을 알 수 있을 것이다. 부분적으로,  그 이유는 그 사회들이 지배자의 의지를 강요하기 위한 효과적 수단을 갖추지 못했기 때문이기도 하다. 그 사회에는 잘 조직된 현대적 경찰도 없었고, 고속 장거리 통신망도 없었으며, 감시 카메라도, 시민들의  개인정보도 없었던 것이다. 따라서 통제를 피하기도 상대적으로 쉬웠다.  


96. 우리의 헌법적 권리라는 것이 도대체 어떤 것인지, 언론의 자유를 예로 들어 알아보자. 우리는 결코 언론의 자유를 공격하려는 것이 아니다. 이 권리는 정치권력의 집중을 막는 데, 또 정치권력자들이  저지르는 잘못을 대중에게 알리는데에 매우 중요한 수단이다. 그러나 언론의 자유는 개인으로서의  평범한 시민에게는 거의 쓸모가 없다. 대중매체들은 거의 대부분 거대 조직의 통제하에 놓여 있으며, 이  거대 조직들은 체제에 통합되어 있다. 누구라도 약간의 돈만 있으며 인쇄물을 만들 수 있고, 그것을  인터넷이나 그 밖의 다른 방법을 통해 배포할 수도 있다. 하지만 그가 말하고자 하는 내용은 미디어가  쏟아내는 엄청난 정보의 늪에 잠겨 버리고, 결국 아무런 실질적인 효과를 거두지 못한다. 따라서  개인들이나 작은 집단들이 말로써 사회의 이목을 끌기란 거의 불가능하다. 우리 FC가 바로 그 사례이다.  우리가 폭력을 저지르지 않은 상태에서 지금 여러분이 읽고 있는 글을 출판사에 보냈다고 가정해 보라.  출판사는 아마 받아주지 않았을 것이다. 설령 출판사가 그것을 받아 출판한다 하더라도, 아마 많은  독자들의 관심을 얻지는 못했을 것이다. 심각한 논문을 읽기보다는 미디어가 제공하는 오락거리를 보는  것이 훨씬 더 재미있는 일이기 때문이다. 게다가 설령 많은 독자들이 그것을 읽는다해도, 대부분의  독자들은 금방 자신들이 읽은 것을 잊어버릴 것이다. 미디어가 그들에게 심어 놓은 엄청난 정보들로 머리가 가득차기 때문이다. 우리의 글이 오래도록 기억되게끔 대중에게 깊은 인상을 주기 위해, 어쩔 수  없이 사람들을 죽여야만 했다.  


97. 헌법적 권리들은 어떤 면에서는 유용하지만, 그것들은 부르주아적 개념에서의 자유라고 부를 수 있는 것 이상을 보장해 주지는 못한다. 부르주아들의 자유 개념을 따르자면, "자유로운" 인간은 근본적으로  사회라는 기계의 부품일 뿐이며, 고작해야 사전에 정해지고 제한된 자유를 확보할 수 있을 뿐이다.  부르주아의 자유란 개인의 욕구가 아니라 사회라는 기계의 욕구를 충족시키기 위해 고안된 자유에  불과하다. 따라서 부르주아의 "자유인"은 경제적 자유가 성장과 진보를 촉진하기 때문에 경제적 자유를  누린다. 그가 언론의 자유를 누릴 수 있는 것은 공적인 비판이 정치 지도자들의 악행을 제한하기  때문이다. 그가 공정한 재판의 권리를 갖는 것은, 권력자들이 멋대로 사람들을 감옥에 처넣는 것이 사회에 해롭기 때문이다. 사이먼 볼리바(Simon Bolivar)가 취했던 태도가 바로 그런 것이었다. 그가 볼 때,  사람들의 자유는, 오직 그 자유가 진보(그것도 부르주아가 생각하는 진보)를 촉진하는 경우에 한해  허용될 수 있는 것이었다.\hyperlink{28}{$^{28}$} 다른 부르주아 사상가들도 그와 비슷한 자유에 대한 시각을 지니고 있었는 바, 그들에게 자유는 집단적인 목적을 이루기 위한 수단에 불과했다. 체스터 탄(Chester C. Tan)은 "20세기  중국의 정치 사상" 202페이지에서, 국민당 지도자 후 한민(Hu Han-Min)의 사상을 다음과 같이  설명하고 있다. "한 개인에게 권리가 보장되는 이유는 그가 사회의 일원이기 때문이며, 그의 공동체적  삶이 그러한 권리를 필요로 하기 때문이다." 후 한민에게 있어 공동체란 곧 국가 사회 전체를 뜻하는  말이었다.\hyperlink{29}{$^{29}$} 이어 259페이지에서 탄은 장 춘매(Chang Chun-Mai, 중국 국가 사회당 당수)에 따르면,  자유는 국가와 전체 인민의 이익을 위해 사용되어야 한다고 말하고 있다.\hyperlink{30}{$^{30}$} 하지만, 오로지 다른 사람이  정해 놓은 대로밖에는 사용할 수 밖에 없는 자유라면, 그것을 자유라고 할 수 있는가? 우리가 생각하는  자유는 볼리바나 후 한민, 장 춘매, 또는 기타 부르주아 이론가들이 생각하는 그런 자유가 아니다. 이  이론가들의 문제는 그들이 대리 활동으로서 사회 이론들을 발전시키고 적용해 왔다는 점이다. 따라서, 그  이론들은, 그 이론이 강요되는 사회에서 살아야 하는 불행한 사람들의 욕구를 충족시켜주기 위해서가  아니라, 이론가들 자신의 욕구를 충족시키기 위해 고안된 것이다. 


98. 한 가지만 더 지적하고 넘어가기로 하자. 어떤 사람이 자기가 충분한 자유를 누리고 있다고 말한다는  이유만으로 그가 충분한 자유를 누리고 있다고 단정해서는 안된다. 자유는 사람들이 전혀 인식하지  못하는 심리적 통제에 의해 제한 당한다. 게다가 무엇이 자유를 구성하는가에 관한 사람들의 생각 자체가  자신들의 진정한 욕구보다는 사회적 관습에 의해 더 많이 결정된다. 예를 들어, 과잉 사회화된 좌파들은  아마도 자신들을 포함해서 대부분의 사람들이 과잉 사회화된 것이 아니라 충분히 사회화되지 못한  것이라고 주장할 것이다. 하지만 그렇게 주장한다 해도, 과잉 사회화된 좌파들이 고도의 사회화를 위해  무거운 심리적 대가를 치르고 있음은 부인할 수 없는 사실이다.  


\section*{역사의 몇 가지 원칙}


99. 역사가 두 가지 요소들이 합해진 결과라고 생각해보자. 한 요소는 전혀 예측 불가능한 패턴을 따라  벌어지는 우발적인 사건들로 구성된 불규칙한 요소이고, 다른 요소는 장기간에 걸친 역사적 경향을 따라  구성된 규칙적 요소이다. 여기서 우리가 다룰 것은 장기간에 걸친 경향이다. 


100. 제1원칙. 장기간에 걸친 역사적 경향에 영향을 미칠만한 작은 변화가 일어날 경우, 그 변화의  효과는 거의 언제나 단기간에 그치며, 역사적 경향은 곧 원래의 상태로 되돌아가 버리고 만다. (예: 한  사회의 정치적 부패를 해소하기 위해 마련된 개혁 운동이 단기적 효과 이상을 발휘하는 경우는 거의 없다. 머지 않아 개혁가들의 긴장은 풀리고, 부패가 다시 슬며시 고개를 들기 시작한다. 한 사회의 정치적  부패는 항상 일정한 수준에 머물며, 변한다 해도 사회 진화에 따라 아주 천천히 변할 뿐이다. 일반적으로  정치적 청소 작업은 광범위한 사회 변동이 수반될 경우에 한해 지속될 수 있다. 해당 사회에서의 작은  변화로는 충분치 않은 것이다.) 만약 장기간에 걸친 역사적 경향에서 작은 변화가 지속적인 것으로  보인다면, 그것은 변화가 이미 경향이 움직이는 방향으로 함께 일어나기 때문이다. 따라서 한 발을 더  내딛는다고 해서 그것만으로 경향이 바뀌지는 않는다. 


101. 제1원칙은 거의 동어반복이라 할 수 있다. 만약 그 경향이 작은 변화에도 흔들릴 만큼 불안정한  것이라면, 하나의 확고한 방향을 따라 진행되기보다는 마구잡이로 흘러갈 것이다. 달리 말하자면, 그런  경향은 결코 장기적인 경향이 될 수 없다.  


102. 제2원칙. 장기간에 걸친 역사적 경향을 지속적으로 바꿀만큼 커다란 변화가 일어날 경우, 그 변화는 사회 전체를 바꿀 것이다. 다시 말해서, 사회는 모든 부분들이 연관지어진 체제이며, 그 다른의 부분들을  동시에 영구히 바꾸지 않고서는 어느 중요 부분도 영구히 바꿀 수 없는 것이다.  


103. 제3원칙. 장기간에 걸친 역사적 경향을 영구적으로 바꿀 만큼 커다란 변화가 일어날 경우, 사회  전체에 걸쳐 일어날 변화의 결과를 사전에 미리 예측할 수 없다. (다양한 다른 사회들이 똑같은 변화를  겪고 똑같은 결과를 경험하지 않는 한, 그래서 어느 다른 사회가 똑같은 변화를 겪었을 때 비슷한 결과를  경험할 것이라는 것을 실증적으로 예측할 수 있기 전에는, 예측이란 불가능하다.)  


104. 제4원칙. 새로운 유형의 사회를 종이 위에 설계할 수는 없다. 즉, 어떤 형태의 새로운 사회를 사전에 계획하고 세운 후, 본래의 설계 의도대로 기능할 것을 기대할 수는 없다. 


105. 제3원칙과 제4원칙은 인간 사회의 복잡성에서 생겨난 것이다. 인간 행동의 어떤 변화는 그 사회의  경제와 물리적 환경에 영향을 미친다. 경제는 환경에 영향을 미치며, 환경은 경제에 영향을 미친다.  그리고 경제와 환경의 변화는 복잡하고 예측 불가능한 방식으로 인간 행동에 영향을 미친다. 원인과  결과의 그물망이 너무나 복합적인 탓에 그것을 온전히 이해하기란 불가능하다.  


106. 제5원칙. 사람들은 의식적으로, 이성적으로 자신이 속한 사회 형태를 선택하지 않는다. 사회는 사회 진화의 과정을 통해 발전하는 바, 그 진화 과정은 인간의 이성적 통제를 벗어나 있다.  


107. 제5원칙은 다른 네 가지 원칙의 결과이다.  


108. 결론: 제1원칙에 따라, 일반적으로 사회 개혁의 시도는 두 가지 방식으로 이루어진다. 어떤 식으로든 사회가 발전해 가는 방향을 따라 이루어지거나(따라서 개혁은 이미 진행되고있는 변화를  가속시킬 뿐이다.), 아니면 그저 일시적인 효과에 그침으로써 사회는 곧 원래의 상태로 슬며시  되돌아간다. 사회의 어느 중요한 측면의 발전과 같은 방향으로 지속적인 변화를 일으키려면, 개혁은  충분치 않으며 혁명이 필요하다.(혁명이라고 해서 반드시 무장 봉기를 일으키거나 정부를 무너뜨려야  하는 것은 아니다.) 제2원칙에 따라, 혁명은 사회의 어느 한 측면만이 아니라 사회 전체를 변화시킨다. 제 3원칙에 따라, 변화는 결코 기대하지 않았던, 혹은 바라지 않았던 방식으로 일어난다. 제4원칙에 따라,  혁명가나 이상주의자들이 어떤 새로운 유형의 사회를 세운다 해도 그 사회는 결코 계획대로 움직여주지  않는다.  


109. 미국 독립 혁명 역시 이 원칙에서 벗어나지는 않는다. 미국 독립 “혁명”은 우리가 말하는 의미에서의 혁명은 아니었다. 그것은 독립 전쟁이었으며, 이후 광범위한 정치적 개혁이 뒤따랐을 뿐이다. 건국의  아버지들은 미국 사회의 발전 방향을 바꾸지도 않았을 뿐더러, 그러기를 원하지도 않았다. 그들은 다만  영국의 지배로 인해 미국 사회의 발전이 느려지는 것을 막았을 뿐이다. 그들의 정치적 개혁은 어떤 근본적 경향도 바꾸지 않았으며, 미국의 정치 문화가 자연스러운 발전 방향으로 전개되도록 도와주었을 뿐이다.  미국 사회의 모체인 영국 사회는 이미 오래 전부터 대의제 민주주의 방향으로 옮겨가고 있는 중이었다.  게다가 독립 전쟁 이전에도 이미 미국인들은 식민지 의회를 통해 상당한 정도의 대의 민주주의를  실천하고 있었다. 헌법에 의해 성립된 정치 제도는 영국의 제도와 식민지 의회의 제도를 그대로 본뜬 것이었다. 중요한 변화와 함께 미국 건국의 아버지들이 대단히 중요한 발걸음을 내딛은 것은 부인할 수  없는 사실이다. 그러나 그 발걸음은 이미 영어권 세계가 움직이고 있었던 길을 그대로 따라간 것이었다.  영국과 영국 출신이 인구의 절대 다수를 점유했던 모든 식민지 국가들이 미합중국의 그것과 근본적으로  똑같은 형태의 대의 민주주의 제도를 구축했다는 것이 그 증거다. 설령 미국 건국의 아버지들이 뚝심이  부족해 독립선언에 서명하지 않았다고 해도, 오늘 우리들의 삶은 별로 달라진 것이 없을 것이다. 어쩌면  영국과 여전히 긴밀한 유대 관계를 지속하면서 국회(Congress)와 대통령 대신에 의회(Parliament)와 
총리를 갖게 되었을지도 모르겠다. 어차피 그게 그거다. 미국 혁명은 우리가 제시한 역사의 원칙을  반박하는 것이 아니라, 오히려 그것이 옳음을 보여주는 좋은 사례다.  


110. 그렇지만, 이 원칙들을 적용할 때는 상식을 벗어나지 말아야 한다. 이 원칙들은 엄밀하지 못하며,  그에 따라 다양한 해석이 가능하다. 한편 그 원칙들에서 벗어나는 예외적인 역사도 있을 수 있다. 따라서  우리는 이 원칙들이 신성불가침의 법칙이 아니라 대략적인 원칙, 또는 일종의 지침으로서 사회의 미래에  대한 순진한 생각들에 대한 부분적 해독제 정도의 역할을 수행할 수 있을 것이라는 점을 밝혀 둔다. 이  원칙들을 항상 염두에 두어야 하며, 이 원칙들에 어긋나는 결론에 도달할 때마다 조심스럽게 자신의  생각을 재검토해 보아야 할 것이며, 원칙에 어긋나는 결론은 정당한 근거가 있는 경우에 한해 내려야할  것이다. 


\section*{산업-기술 사회의 개혁은 불가능하다} 


111. 앞서 든 원칙들을 보면, 산업 체제가 우리의 자유를 점점 침해하는 것을 막는 정도의 방법으로는  산업 체제를 개혁하는 것이 얼마나 절망적으로 어려운 일인가를 알 수 있다. 산업혁명부터 계속되어 온 하나의 일관된 경향이 있으니, 그것은 기술이 개인의 자유와 지역의 자율성의 희생이라는 비싼 대가를  치르며 체제를 강화해 왔다는 것이다. 그러니 기술로부터 자유를 보호하겠다는 목적으로 계획된 어떤 
변화도 우리 사회의 근본적인 발전 추세와는 대치되는 것이다. 결과적으로 그런 변화는 역사의 파도에 곧 묻혀버릴 일시적 변화에 그치거나, 아니면 우리 사회의 전체의 본질을 영구적으로 바꿔 놓을 만큼 거대한  변화이거나 둘 중 하나일 것이다. 이것은 제1원칙과 제2원칙에 의한 결과다. 더구나 사회란 사전에  예측할 수 없는 방식으로 변할 것이므로(제3원칙), 거기엔 커다란 위험성이 담겨 있다. 자유를  지키면서도 이전의 사회와는 확연히 다른 사회를 오래 유지할 수 있을 만큼 중대한 변화는 시작되지도  않을 것이다. 그런 변화들이 체제를 교란시킨다는 것을 깨달을 것이기 때문이다. 따라서 개혁을 향한 모든 시도들은 너무나 소심한 나머지 아무런 효과도 보지 못한다. 설령 영구적인 차이를 낳을 만큼 커다란  변화가 시작된다 해도, 그것들이 사회를 혼란시키는 효과가 눈에 드러날 때에는 꽁무니를 빼게 될 것이다. 결국, 오로지 체제 전체의 급진적이고 위험하며 예측불가능한 변화를 받아들일 준비가 되어 있는  사람들만이 자유를 수호하는 영구적 변화를 이루어 낼 수 있다. 다시 말하자면, 개혁가가 아니라  혁명가만이 변화를 이룰 수 있다.  


112. 자유를 회복하되 기술의 혜택을 포기하지 않으려고 전전긍긍하는 사람들은 자유와 기술이 공존하는 새로운 형태의 사회를 건설하는 순진한 계획을 제시할 것이다. 그런 제안을 내놓는 사람들이 이 새로운  형태의 사회를 어떻게 세울 수 있을지에 대해 전혀 실천적인 방안을 제시하지 못한다는 사실은 일단 무시하자. 설령 어찌어찌해서 새로운 형태의 사회가 일단 세워진다고 해도, 그 사회는 제4원칙을 따라  무너져 버리거나 아니면 기대했던 바와는 전혀 다른 결과를 낳게 될 것이다.  


113. 그러니 아무리 좋게 봐준다 해도 사회를 변화시킴으로써 자유와 현대 기술을 조화시킨다는 것은  거의 불가능한 일로 보인다. 이어질 몇개의 문단들에서 우리는 자유와 기술의 발전이 공존할 수 없는  구체적 이유를 살펴볼 것이다. 


\section*{산업사회에서 자유의 제한은 불가피하다}


114. 문단 65${\sim}$67, 70${\sim}$73에서 설명한 바와 같이, 현대인은 규칙과 규제의 그물에 사로잡혀있고, 그가 영향을 줄 수 없는 사람들의 행동에 그의 운명이 결정된다. 이것은 우연이나 거만한 관료들이 횡포를 부린 결과가 아니다. 이는 기술적으로 진보된 모든 사회에서 필요하고 불가피하다. 체제는 기능을 수행하기  위해서 인간의 행동을 빈틈없이 규제해야만 한다. 작업장에서 인간은 자신이 지시받은 일을 해야만 한다.  그렇지 않으면 생산라인이 혼란에 빠지기 마련이다. 관료제는 엄격한 규칙에 따라 시행되어야 한다. 하위  관료들에게 실질적인 재량권을 부여할 경우, 체제가 망가지거나, 각 관료들이 자신의 재량으로 시행한  방법의 차이에서 기인하는 불공정에 대한 문책으로 이어진다. 우리들의 자유를 침해하는 몇몇 제한들은 제거될 수도 있는 것이 사실이다. 하지만 일반적으로, 거대 조직에 의한 우리의 삶에 대한 규제는 산업 기술 사회의 정상적 기능을 위해서 필요한 것이다. 결과적으로 평범한 사람들은 무력감을 느끼게 된다.  물론, (프로파간다,\hyperlink{31}{$^{31}$} 교육 기술, “정신 건강” 프로그램 등)심리학적 수단들을 이용해 공식적인 규제  없이도 사람들이 체제가 필요로 하는 것을 스스로 원하도록 만들 수도 있다. 


115. 체제는 인간에게 점점 더 인간의 자연스러운 행동 방식과는 거리가 먼 방식으로 행동하도록  강요한다. 예를 들어, 체제는 과학자, 수학자, 그리고 기술자를 필요로 한다. 체제는 그들 없이는 제대로  작동할 수 없다. 그러므로, 아이들에게는 이러한 분야들을 공부하도록 아주 무거운 압력이 가해진다.  사람들이 청소년기에 많은 시간을 책상에서 공부하는데 빼앗기면서 보내는 것은 자연스럽지 못하다.\hyperlink{32}{$^{32}$} 평범한 청소년들은 그들의 시간을 현실세계와 능동적인 접촉을 하면서 보내기를 원한다. 원시인들의  세계에서 아이들이 훈련받은 것들은 타고난 인간의 충동과 자연스러운 조화를 이루고 있었다. 아메리카  인디언들의 세계에서는, 예를 들어, 소년들은 (소년들이 좋아하는) 능동적인 야외 활동을 하도록  훈련받는다. 그러나, 우리 사회에서 아이들은 기술과 관련된 과목들을 공부하도록 내몰리고 있고  대부분이 마지못해 그것을 한다.  


116. 체제가 인간의 행동을 바꾸기 위해 꾸준히 행사하는 압력때문에, 사회의 요구를 수용하지 못하거나  하지 않으려는 사람들(복지에 의존해 사는 사람, 청소년 조직폭력배, 광신도, 반정부 세력, 급진적  환경주의자, 낙오자, 그리고 여러 종류의 반항아들)이 꾸준히 증가하고 있다.  


117. 기술적으로 진보된 사회에서 개인의 운명은 그가 직접 영향력을 줄 수 없는 결정들에 달려있다.  기술이 발달한 사회는 아주 많은 사람들과 기계들의 협동을 통해 생산하기 때문에 자치적인 소규모  공동체들로 쪼개어 질 수 없다. 기술 사회는 고도로 구성되어야만 하며 결정은 아주 많은 사람들에게  영향을 끼친다. 예를 들어, 어떤 결정이 100만명에게 영향을 미친다면 영향을 받는 각 개인은 평균적으로 의사결정의 100만분의 1의 몫만을 가지고 있을 뿐이다. 실제로는 공무원, 기업의 임원이나 기술  전문가들이 결정을 내리며, 결정에 대한 투표에서조차 투표 참가자가 너무나 많은 탓에 개인의 투표는  중요하지 않다.\hyperlink{33}{$^{33}$} 따라서 대부분의 개인들은 그들의 삶에 영향을 미칠 주요한 결정에 가시적인 영향을  미치지 못한다. 기술적으로 진보된 사회에서는 이 문제를 치유할 만한 어떤 납득할 만한 방법도 존재하지  않는다. 체제는 사람들이 그들에게 내려진 결정을 스스로 원하도록 만드는 프로파간다를 통해서 이러한  문제를 “해결”하려 한다. 설령 이 “해결책”이 사람들을 기분 좋게 하는데에 완벽한 성공을 거두더라도,  이는 모욕적이다. 


118. 보수주의자들과 몇몇 사람들은 “지방 자치”에 호소한다. 지역 공동체들은 한때 자치권을 가졌었다.  그러나, 그러한 자치는 지역 공동체들이 공공 시설, 컴퓨터 네트워크, 고속도로망, 대중 매체, 현대 의료 체계 등과 같은 거대한 규모의 체제와 연결되고 의존하게 됨에 따라 점점 더 가능성이 줄어들고 있다. 한  지역에 적용된 기술이 종종 멀리 떨어진 다른 지역의 사람들에게 영향을 미친다는 사실 또한 자치를  방해한다. 샛강 주변에서의 살충제나 화학약품의 사용은 수백 마일 떨어진 하류의 상수원을 오염시킬  수도 있고, 온실효과는 전 세계에 영향을 미친다.  


119. 체제는 인간의 필요를 충족시키기 위해 존재하는 것이 아니며, 그럴 수도 없다. 대신 인간의 행동이  체제의 필요에 맞춰 수정되어야 한다. 이것은 기술 체제를 유도하는 것 처럼 보이는 정치적인 혹은  사회적인 이념과는 무관하다. 체제는 이념에 의해서가 아니라 기술적인 필요에 의해 유도되기 때문에  그것은 기술의 잘못이다.\hyperlink{34}{$^{34}$} 물론 일반적으로 체제는 인간의 필요에 무관심하지만, 인간의 필요가 체제에  유익할 경우에는 그렇게 한다. 가장 중요한 것은 체제의 필요이지 인간의 필요가 아니다. 예를 들어, 모든 사람이 굶주린다면 체제는 제대로 기능하지 않기 때문에 체제는 사람들에게 음식을 공급한다. 너무나  많은 사람이 침체되어 빠져있거나 반항적이라면 체제는 제대로 기능할 수 없기 때문에, 가능할 때 마다,  체제는 사람들의 심리적인 요구를 봐준다. 그러나, 체제는, 합당한, 확고한, 실용적인 이유에 의해,  사람들이 그들의 행위를 체제의 필요에 맞추도록 끊임없이 압력을 가해야한다. 너무 많은 쓰레기가  쌓이는가? 정부, 매체, 교육 제도, 환경주의자, 모두들 우리에게 재활용에 관한 대량의 선전을 퍼뜨리고  있다. 더 많은 기술자가 필요한가? 수많은 목소리들이 어린이들에게 과학을 공부하도록 권한다. 아무도  청년들에게 그들이 싫어하는 과목을 공부할 것을 강요하는게 비인간적인지 물어보려 하지 않는다. 
숙련된 노동자가 기술적 진보에 의해 직장에서 쫓겨나서 “재교육”을 받아야 하게 될때도, 아무도 그들이  그런식으로 내몰리는 것이 잔인하지 않는가를 물어보지 않는다. 모든 사람들이 기술적인 필요성과  합당한 이유에 굴복해야 한다는 것이 당연한 것으로 쉽게 받아들여지고 있다. 만약 인간의 필요가  기술적인 필요보다 앞서 놓여진다면 경제 문제, 실업, 물자 부족 및 심각한 사태가 발생할 것이다. 우리  사회에서 “정신 건강”의 개념은 많은 부분이 개인이 체제의 필요에 맞게 행동하고 스트레스 징후를  보이지 않는 것으로 정의되고 있다.


120. 체제 내에서 목적의식과 자율성을 부여하려는 노력은 농담에 불과하다. 예를 들어, 어떤 회사에서,  각 직원이 목록의 오직 한 부분만을 조립하도록 하는 대신에 전 목록을 조립하게 한다면 이것은 그들에게  목적의식과 성취동기를 부여하는 것으로 간주될 수 있다. 어떤 회사는 직원들에게 그들의 작업에 있어서  
더 많은 자율성을 주려고 시도했었다. 그러나, 실용적인 이유로, 대부분의 경우 매우 제한된 정도로만  행해질 수 있다. 그리고 어느 경우에서나 직원은 궁극적인 목적에 비교하면 어떤 자율성도 부여받지  못했다. -- 그들의 "자율적인" 노력은 결코 그들 자신이 직접 선택한 목표로 향하지 못했고 오직 회사의  생존과 성장과 같은 고용주의 목표만을 향했다. 그들의 직원에게 그와는 달리 행동할 수 있도록  허용한다면 그 회사는 얼마안가 문 닫게될 것이다. 비슷하게, 사회주의 체제의 어떤 사업이든, 노동자들은 그들의 노력을 사업의 목적에 맞추어야 한다. 그렇지 않으면 그 사업은 체제의 일부로서 그 목적을 다할  수 없을 것이다. 다시 말하지만, 순전히 기술적인 이유 때문에, 대부분의 개인 또는 작은 집단이 산업  사회에서 자율권을 가지는 것은 불가능하다. 중소기업의 오너조차도 단지 제한된 자율권을 가진다.  정부의 규제는 별도로 하고라도, 그는 경제 체제에 맞추어야 하고 그것의 요구를 따르지 않으면 안된다는  사실에 의해 제한되어 있다. 예를 들어, 누군가가 새로운 기술을 개발한다면, 중소기업인은 경쟁력을  유지하기 위해 대부분의 경우 그가 원하건 원하지 않건 간에 그 기술을 사용해야만 한다.  


\section*{기술의 “나쁜” 부분은 “좋은” 부분과 분리될 수 없다} 
121. 산업사회가 자유를 위한 방향으로 개혁될 수 없는 또 한가지 이유는 현대의 기술은 모든 부분이  다른 부분에 의존하는 통합된 체제라는 것이다. 당신은 기술의 “나쁜” 부분을 제거하고 오직 “좋은”  부분만을 취할 수는 없다. 현대 의학을 예로 들어보자. 의학에서의 진보는 화학, 물리학, 생물학, 전산학,  그리고 다른 분야들의 진보에 의존한다. 첨단 의학기술은 기술적으로 진보했고, 경제적으로 부유한  사회에서만 감당할 수 있는 비싼 첨단 장비를 필요로 한다. 분명히, 모든 기술적 체제와 그에 필요한 제반 조건들 없이는 의학에서의 진보는 있을 수 없다.  


122. 의학의 진보가 나머지 기술 체제 없이 이루어진다 할지라도, 그 자체로서 어떤 해악을 가지게 될  것이다. 예를 들어 당뇨병에 대한 치료법이 발견되었다고 가정해보자. 그러면 당뇨병에 대한 유전적  경향을 지니고 있는 사람들은 계속 생존하여 다른 사람들과 마찬가지로 자식들을 낳을 것이다. 당뇨병을  유발하는 유전자에 대한 자연 선택은 멈추고 그 유전자들이 사람들 사이에 퍼지게 될 것이다.(당뇨병이  완치될 수는 없지만, 인슐린에 의해 억제될 수 있기 때문에, 이런 현상은 이미 어느 정도 발생하고 있다.)  같은 일이 인류의 유전적 퇴화에 의해 영향을 받는 (어린이의 암 발생 같은)다른 질병에도 일어날 것이다.\hyperlink{35}{$^{35}$} 유일한 해결책은 일종의 우생학 프로그램이나, 인류에 대한 광범위한 유전공학일 것이며, 그러므로  미래의 인간은 더 이상 자연이나 우연, 또는 (당신의 종교적, 철학적 믿음에 따라)신의 창조물이 아니라  체제의 생산품이 될 것이다.  


123. 만약 당신이 거대한 정부가 지금 당신의 생활에 너무 많이 간섭하고 있다고 생각한다면, 정부가  당신 자녀의 유전자를 규제하기 시작할 때까지 한번 기다려보라. 유전공학을 규제하지 않을 경우, 결과는  심각할 것이기 때문에, 인간 유전공학의 도입 후 그러한 규제가 불가피하게 따를 것이다.\hyperlink{36}{$^{36}$} 


124. 이 문제에 대한 대부분의 반응은 "의학 윤리"에 호소하는 것이다. 하지만 윤리 규약은 의학의  진보에 대항하여 자유를 보호하는 역할을 할 수 없을 것이다. 그것은 오히려 사태를 더욱 악화시킬  것이다. 유전공학에 적용되는 윤리 규약은, 실제로는 인간의 유전적 구성을 규제하는 수단이 될 것이다.  (아마도 중상류층)사람들이 유전공학의 “윤리”적인 부분과 윤리적이지 못한 부분을 결정할 것이며,  자신들의 결정을 대중에게 강요할 것이다. 윤리 규약이 전적으로 민주적인 절차를 통해 결정되었다고 
할지라도, 유전공학의 “윤리”적인 부분에 동의하지 않는 소수에게 다수의 가치를 강요하게될 것이다.  진정으로 자유를 보호하는 윤리 규약은 인간 유전자 조작을 원천봉쇄하는 것이지만, 당신은 기술 사회가  이런 규약을 절대로 받아들이지 않으리라는 것을 확신할 수 있다. 생명공학의 엄청난 힘이 제시하는  유혹은 거부할 수 없고, 특히 (육체적, 정신적 질병을 제거하고, 현대사회를 살아가는 데 필요한 능력을  주니까)대다수의 사람들에게 그것이 명백하게 좋아보이기 때문에, 유전공학을 제한하는 어떠한 규약도  오래가지는 못할 것이다. 불가피하게, 유전공학은 광범위하게, 그러나 오직 산업-기술 체제의 필요에  부합하는 방향으로만 사용될 것이다. 
기술은 자유를 향한 열망보다 더 강력한 사회적 권력이다. 


125. 기술과 자유 사이의 타협을 유지하는 것은, 기술이 훨씬 더 강력한 사회적 권력이고, 반복되는  협상을 통해 계속해서 자유를 침해하기 때문에 불가능하다. 처음에는 같은 크기의 땅을 소유하고 있지만,  한 사람이 다른 사람보다 힘이 센, 두 이웃의 경우를 상상해보자. 힘 센 사람이 상대의 땅의 일부분을  요구한다. 약한 사람은 거절한다. 힘이 센 사람이 말한다. "좋소. 타협 합시다. 내가 요구한 것의 절반을  주시오." 약한 사람은 굴복하는 수 밖에 없다. 얼마 후에, 힘 센 사람이 땅의 다른 부분을 요구하고, 다시  타협이 이루어지고, 계속 반복한다. 약한 사람에게 일련의 타협을 강요함으로써, 힘이 센 사람은 결국  모든 땅을 얻는다. 기술과 자유의 갈등에서도 그러하다.  


126. 왜 기술이 자유를 향한 열망보다 더 강력한 사회적 권력인지 설명하겠다. 


127. 자유를 위협하지 않는 것처럼 보이는 기술의 진보는 종종 자유를 위협하는 것으로 밝혀지며 나중에  가서는 자유를 매우 심각하게 위협하는 것으로 밝혀진다. 예를 들어, 자동차에 대해 생각해 보자. 옛날  보행자는 그가 원하는 어디든, 어떤 교통 법규도 살피지 않고 자신이 원하는 속도로 갈 수 있었고,  기술적인 지원 체제에 독립적이었다. 자동차가 소개되었을 때, 이것이 인간을 더욱 자유롭게 해줄 것 처럼 보였다. 자동차는 보행자로부터 어떠한 자유도 박탈하지 않았으며, 그가 원하지 않는다면 자동차를 가질  필요가 없었고, 자동차를 갖고 있는 사람은 보행자보다 훨씬 빨리 이동할 수 있었다. 그러나 자동차의  등장은 곧 사회를 인간의 이동의 자유를 상당히 제한하는 방향으로 바꾸었다. 자동차가 매우 많아졌을 때, 자동차 사용을 광범위하게 규제할 필요가 생겼다. 특히 인구가 밀집된 지역에서는, 자동차를 자신이  원하는 곳에 자신이 원하는 속도로 갈 수는 없다. 그의 움직임은 교통의 흐름과 여러가지 교통법규에 의해 좌우된다. 운전자는 면허 취득, 운행 테스트, 갱신 등록, 보험, 안전 점검, 자동차 할부금 등 여러가지  의무에 묶여 있다. 게다가, 자동차의 사용은 더 이상 선택사항이 아니다. 자동차의 등장 이후 도시는  사람들이 자동차 없이는 출퇴근, 쇼핑, 여가활동을 할 수 없도록 재구성되었다. 자동차가 없으면  대중교통을 이용해야 하는데, 이 경우에는 그들은 자동차를 사용할 때보다 그들 자신의 이동에 대해 훨씬  더 적은 통제력을 가지게 된다. 보행자의 자유 또한 이제는 극도로 제한된다. 도시에서는 그는 반복해서  멈춰서서, 자동차 통행을 위주로 설계된 교통 신호를 기다려야 한다. 지방에서는 자동차 통행이  고속도로를 따라 걷는 것을 위험하고 불쾌하게 만든다.(우리가 자동차의 사례와 함께 제시한 중요한  논점에 집중하라. 기술의 새로운 요소가 그가, 개인의 선택에 달려있는 것으로 도입되었을 때, 그것은  반드시 선택적인 것으로 남아있지는 않는다. 많은 경우에, 새로운 기술은, 사람들이 결국 그것을  사용하도록 강요하는 방향으로 사회를 바꾼다.) 


128. 기술의 진보가 `대체로' 끊임없이 우리의 자유를 축소시키는 반면에, 전기, 실내 수도관, 고속  장거리 통신 같은 각각의 새로운 기술적 진보는 그 자체로서 바람직한 것으로 간주된다. 누가 감히 이러한 기술, 또는 그 외의 수없이 많은 현대사회의 기술적 진보들을 반대할 수 있겠는가? 예를 들어 전화기  발명에 반대하는 것은 어리석은 짓이었을 것이다. 전화기는 많은 장점을 제공하면서도 어떠한 단점도  없었다. 하지만 단락 59${\sim}$76에서 설명한 바와 같이, 이러한 기술적 진보들이 결합된 결과 보통 사람들의  운명이 더 이상 그 자신이나 그의 이웃, 친구의 손이 아닌, 개인이 영향을 미칠 수 없는 정치가, 기업 임원, 그리고 익명의 기술자와 관료의 손에 매달려 있는 사회를 만들었다.\hyperlink{37}{$^{37}$} 같은 과정이 미래에도 계속된다.  유전공학을 예로 들어보자. 유전병을 제거하는 유전공학의 도입에 어느 누구도 반대하지 않을 것이다.  그것은 무해하고, 많은 고통을 예방한다. 하지만 수많은 유전공학적 개선들은 인간을 우연(혹은 당신의  종교적 믿음에 따라, 신)의 창조물이 아니라, 공학적 생산품으로 전락시킬 것이다. 


129. 기술이 강력한 사회적 권력인 또 다른 이유는, 주어진 사회의 배경 속에서 기술적 진보는 오직 한  방향으로만 나아가기 때문이다. 그 방향은 결코 되돌려 질 수 없다. 일단 어떤 기술 혁신이 도입되면  사람들은 대개 그것에 의존하게 되고, 그렇지 않으면 그 기술은 훨씬 더 진보된 기술적 혁신에 의해  대체될 것이다. 사람들이 개인적으로 신기술에 의존할 뿐만 아니라, 그 이상으로 체제 역시 대체로 그것에 의존하게 된다.(예를 들어,만약 컴퓨터가 없어진다면 현재의 체제에 어떠한 일이 일어날지 상상해보라.)  그러므로 체제는 더 거대한 기술화(technologization)를 향해, 오직 한 방향으로만 나아간다. 기술 체제  전체를 제거하지 않는 한, 기술은 계속해서 자유가 물러서도록 강요한다. 


130. 기술은 급격히 진보하며, 동시에 많은 부분에서 자유를 위협한다.(인구과밀, 규칙과 규제,  거대조직에 의존하게되는 개인, 프로파간다와 다른 심리학적 기술, 유전공학, 감시 장치와 컴퓨터를 통한  개인의 사생활 침해 등) 자유에 대한 위협 중 어느 하나라도 억제하기 위해서는 길고 어려운 투쟁이  필요하다. 자유를 보호하길 바라는 사람들은 새로운 공격의 압도적인 물량과 그것의 발전 속도에  압도당해, 비참해지고 더 이상 저항할 수 없을 것이다. 각각의 위협들과 개별적으로 싸우는 것은 무의미할 것이다. 오직 기술 체제 전체와 싸울 때만 성공을 기대할 수 있을 것이다. 하지만 이건 개혁이 아니라,  혁명이다. 


131. 기술자들(우리는 이 용어를 훈련을 필요로 하는 전문화된 작업을 수행하는 모든 사람들을 가리키는  광범위한 의미로 사용할 것이다.)은 그들의 일에 너무나 열중해 있어서, 그들의 기술과 자유 사이에  갈등이 발생할 경우에는 거의 언제나 그들의 기술에 호의적으로 판단하는 경향이 있다. 이것은 과학자의  경우 명백하다. 그러나 다른 곳에서도 역시 나타난다. 교육자들, 인권단체들, (환경)보호 단체들은 그들의 거창한 목적을 달성하기 위하여 프로파간다\hyperlink{38}{$^{38}$}를 비롯한 심리학적 수단들을 사용하는데 주저하지 않는다.  기업들과 정부 기관은, 그것이 유용하다고 판단될 때, 개인의 사생활에는 신경쓰지 않고 그들에 관한  정보를 모으는 것을 망설이지 않는다. 법 집행기관은 용의자 내지는 흔히 완전히 결백한 사람들의  헌법상의 권리때문에 종종 불편을 느낀다. 그래서 그들은 그러한 권리를 제한하거나 기만하기 위해서  합법적으로(가끔씩 불법적으로) 할 수 있는 일이면 뭐든지 한다. 이러한 교육자들, 정부 관료, 그리고  법률가 대부분은 자유, 사생활, 그리고 헌법상의 권리가 중요하다고 믿는다. 그러나 이것이 그들의 일과  충돌을 일으킬 때는, 그들은 보통 그들의 일이 더욱 중요하다고 느낀다. 


132. 일반적으로, 처벌 같은 부정적인 결과를 피하기 위해서 보다는, 보상을 위해 노력할 때, 사람들이  더욱 꾸준히, 잘 노력한다는 사실은 잘 알려져 있다. 과학자들 그리고 다른 기술자들은 대개 그들의 일을  통해 얻을 수 있는 보상에 의해 고무된다. 그러나 기술에 의한 자유 침해에 반대하는 사람들은 부정적인  결과를 피하기 위해 노력하고 있다. 따라서 이런 비관적인 노력을 꾸준히 그리고 잘 하는 사람은 거의  없다. 만약 개혁가들이, 기술적 진보에 의한 자유 침해를 확실하게 막을 견고한 방벽을 쌓아 큰 승리를  거둔다면, 대부분은 이에 안심하고 다른 목표에 관심을 돌릴 것이다. 그러나 과학자들은 여전히 그들의  연구실에서 바쁘게 지내고, 기술은 진보를 통해 방벽을 우회할 방법을 찾아내 개인에 대해 더욱 강한  통제력을 행사하고, 사람들이 언제나 체제에 더욱 의존하도록 만들 것이다. 


133. 법률이든, 제도든, 관습이든, 윤리적 규범이든 간에, 어떠한 사회적 장치도 기술에 대항해 영속적인  보호를 제공할 수 없다. 역사는 모든 사회적 장치들은 일시적이며, 결국에는 모두 바뀌거나 무너진다는  것을 보여준다. 그러나 기술의 진보는 주어진 문명의 배경에서 영속적인 것이다. 예를 들어 유전공학을  인간에게 사용하거나, 또는 자유나 존엄을 위협하는 쪽으로 사용하는 것을 방지하는 사회적 장치를  만드는 것이 가능하다고 가정해 보자. 여전히 기술은 살아있다. 조만간 사회적 장치들은 무너질 것이다.  아마도 곧, 우리 사회에서 변화의 움직임이 일어날 것이다. 그러면 유전 공학은 우리의 자유를 침해하기  시작할 것이다. 그리고 이 침해는 (기술 문명을 무너뜨리지 않는 한)돌이킬 수 없을 것이다. 현재 환경보호 입법에 일어나고 있는 일들을 통해, 사회적 장치를 통해 영속적인 무언가를 얻으려는 시도가 환상에  불과하다는 것을 알 수 있다. 수년 전만 해도, 환경파괴를 부분적으로나마 막을 수 있는 안전한 법적  방벽이 있는 것처럼 보였다. 정치계의 정세 변화로 인해 그 방벽들은 무너지기 시작했다. 


134. 앞에서 제시한 이유들로 인해, 기술은 자유를 향한 열망보다 더욱 강력한 사회적 권력이다. 그러나  이 명제는 중요한 전제 조건을 필요로 한다. 다음 수십 년 동안 산업-기술 체제는 경제, 환경 문제, 그리고  특히 인간의 행위로 인한 문제(소외, 반란, 적대감, 여러가지 사회적, 심리적 난제들)때문에 엄청난  고통을 받을 것으로 보인다. 우리는 그 고통이 체제를 붕괴시키기를 바란다. 그러한 혁명이 일어나고  성공한다면, 바로 그 때 자유를 향한 열망이 기술보다 더욱 강력함이 입증될 것이다.  


135. 단락 125에서 우리는 힘센 이웃에게 일련의 타협을 강요당해, 땅 전부를 빼앗기는 약한 사람의  비유를 사용했다. 그러나 이제 힘센 이웃이 병들어서 그가 자신을 방어할 수 없다고 가정해 보자. 약한  사람은 힘센 이웃에게 자신의 땅을 돌려줄 것을 요구하거나, 그를 죽일 수도 있다. 만약 그가 강자를  죽이지 않고 그저 자신의 땅을 되돌려줄 것만을 요구한다면 그 사람은 바보다. 강자가 건강을 회복하면  그는 다시 모든 땅을 빼앗을 것이기 때문이다. 약자의 유일한 선택은, 기회가 있을 때 강자를 죽이는  것이다. 같은 이유로, 우리는 산업 체제가 병들어 있을 때 파괴하여야 한다. 만약 우리가 체제와  타협한다면, 그리고 체제가 회복되도록 내버려 둔다면, 그것은 결국에는 우리의 모든 자유를 빼앗아갈 것이다.  
우리는 간단한 사회 문제도 해결할 수 없었다. 


136. 누군가 아직도 기술과 자유를 공존시키기 위해 체제를 개혁하는 것이 가능하다는 꿈을 꾸고 있다면, 우리 사회가 훨씬 더 간단하고도 단순한 사회 문제들조차 어설프게 다루다가, 결국에는 아무 것도  해결하지 못해왔다는 사실을 차분히 생각해보기 바란다. 체제는 환경파괴, 정치적 부패, 마약, 가정폭력  같은 간단한 문제도 해결하지 못했다.  


137. 환경 문제를 예로 들어보자. 당장의 경제적 편의와 우리 후손들을 위한 환경자원 저축이라는 두 가치의 갈등은 단순하다.\hyperlink{39}{$^{39}$} 하지만 우리는 이 주제에 대해 권력자들이 지껄이는 헛소리들 외에는 들을  수가 없다. 명확하고 일관된 행동은 찾아볼 수 없다. 그러는 와중에 우리 후손들은 그 문제들을 그대로  떠안고 살아가게 될 것이다. 환경 문제를 해결하려는 시도들은 서로 다른 파벌 간의 투쟁과 타협으로  점철된다. 어느 때는 한쪽이 주도권을 잡았다가, 어느 때는 다른 쪽이 주도권을 잡는다. 투쟁의 노선은  여론에 따라 계속 바뀐다. 이것은 결코 합리적인 과정이 아니고, 이런 방식으로 문제를 제 시간에 제대로  풀 수 있을 것 같지도 않다. 사회의 중요한 문제들은 설령 “해결”될 수 있다쳐도, 결코 합리적, 포괄적  계획을 통해서 해결되지 않는다. 각자의 (주로 단기적인)이익\hyperlink{40}{$^{40}$}을 좇는 수많은 경쟁 집단들이 (대개는  그저 운이 좋아서) 간신히 타협점을 찾아내 해결된다. 우리가 문단 100${\sim}$106에서 제시했던 역사적  원칙들을 생각하면, 합리적이고 장기적인 사회적 계획은 절대로 성공할 수 없다. 


138. 그러니 인류는 비교적 간단한 사회 문제조차도 어쩌다 겨우 풀 수 있을 만큼 무능하다는 것이  명백한 사실이다. 그런데 자유와 기술의 공존이라는 훨씬 더 어렵고 복잡한 문제를 대체 어떻게 해결할 수 있겠는가? 기술이 분명한 물질적 장점을 제시하는 반면, 자유는 100명의 사람에게 100가지 서로 다른  의미를 지니는 추상적 개념이다. 게다가 자유의 상실은 프로파간다와 달콤한 거짓말로 쉽게 감출 수 있다. 


139. 중요한 차이를 기억하기 바란다. (예를 들어) 언젠가 우리의 환경 문제가 합리적이고 포괄적인  계획을 통해 해결될 수도 있을 것이다. 그러나 만약 환경 문제가 해결된다면, 그것은 환경 문제를  해결하는 것이 장기적으로 볼 때 체제에 이익이 되기 때문이다. 하지만 자유와 작은 집단의 자율성을  보존하는 것은 체제에 이익이 되지 않는다. 오히려 반대로, 인간 행동을 최대한 통제하는 것이 체제에게  이익이 된다.\hyperlink{41}{$^{41}$} 따라서 실질적인 이유 때문에 체제가 억지로나마 환경 문제에 대해 합리적이고 신중하게  접근할 수도 있는 반면, 똑같은 실질적인 이유 때문에 체제가 보다 더 치밀하게 (대부분의 경우 자유를  침해할 것 처럼 보이지 않는 간접적 수단을 통해서) 인간 행동을 규제할 수도 있다. 이것은 우리만의  견해가 아니다. (제임스 윌슨 같은)저명한 사회과학자들 역시 보다 효과적으로 사람들을 “사회화”하는  것이 중요함을 끊임없이 강조해 왔다. 


\section*{혁명이 개혁보다 쉽다}

140. 우리는 독자들이 자유와 기술을 화해시키는 방식으로 체제를 개혁할 수 없다는 사실을 확실히  인식했기를 희망한다. 산업-기술 체제를 모조리 내다버리고 사는 것이 유일한 해결책이다. 이는 혁명을  의미한다. 이 혁명이 반드시 무장봉기가 되어야할 필요는 없지만, 사회의 본질을 뿌리부터 근본적으로  바꿀 것은 분명하다.  


141. 사람들은 흔히 혁명이 개혁보다 많은 변화를 담고 있으므로 개혁보다 어렵다고 생각한다. 하지만  실제로는 일정한 조건 하에서 혁명이 개혁보다 훨씬 쉽다. 혁명은 개혁에 비해 사람들을 더욱 강렬하게  사로잡기 때문이다. 개혁 운동은 고작해야 사회의 특정 문제의 해결을 제안할 뿐이다. 혁명 운동은 단번에 모든 문제를 해결하고, 완전히 새로운 세계를 건설할 것을 제안한다. 따라서 (가령 유전 공학과 같은) 기술의 어떤 한 분야의 개발과 응용을 효과적으로, 또 지속적으로 제한하는 일보다는, 아예 기술 체제  전체를 폐기해 버리는 편이 훨씬 더 쉽다. 오로지 유전공학을 제한하는 일에만 열정을 바칠 사람은 별로  없다. 그러나 조건만 제대로 주어진다면, 아주 많은 사람들이 산업-기술 체제에 항거하는 혁명에 몸을  바칠 것이다. 문단 132에서 지적했듯이 기술의 특정 부문을 제한하고자 하는 개혁가들은 그저 부정적인  결과를 피하기 위해 일하는 것이다. 하지만 혁명가들은 엄청난 보상을 얻기 위해 일한다.  


142. 변화가 너무 과도하게 진행될 경우에 초래될 고통스러운 결과에 대한 두려움 때문에, 개혁은 언제나 제한적일 수 밖에 없다. 하지만 일단 혁명의 열기가 사회를 장악하는 날에는, 사람들은 자신들의 혁명을  완수하기 위해 그 어떤 어려움도 기꺼이 감수한다. 프랑스 혁명과 러시아 혁명에서 그 사실이 이미  증명되었다. 프랑스 혁명과 러시아 혁명에서 진정하게 혁명에 자신의 전부를 바친 사람은 전체 인구에  비하면 소수에 불과했을지도 모른다. 하지만 이 소수는 이미 수적으로는 충분했으며, 적극적인  소수였기에 사회의 지배 세력이 될 수 있었다. 혁명에 관해서는 문단 180${\sim}$205에서 더 자세히 다룰 것이다.  


\section*{인간 행동의 통제} 
143. 문명이 시작될 때부터 조직 사회는 사회 유기체의 기능 수행을 위해서 인간에게 억압을 가해 왔다.  억압의 종류는 사회에 따라 판이하게 다르다. 어떤 억압이 육체적인 것(나쁜 음식, 과도한 노동, 환경  공해)이라면, 어떤 억압은 심리적인 것(소음, 인구 과밀, 사회의 요구에 따른 인간 행동의 개조)이었다.  과거에는 인간의 본성은 어느정도 일관성을 유지할 수 있었고, 달라진다 해도 한정된 범위 안에서  달라졌을 뿐이다. 그 때문에 사회가 사람들을 밀어붙이는데도 일정한 한계가 있었다. 인간이 참을 수 있는 한계를 넘어섰을 때, 사태는 겉잡을 수 없게된다. 반란, 범죄, 부패, 노동 기피, 우울증을 비롯한 정신 질환, 사망률의 증가, 출산율의 감소 등으로 인해 사회가 붕괴될 수도 있고, 아니면 사회의 기능 수행이  너무나 비효율적인 탓에 결국 보다 효율적인 다른 형태의 사회로(신속하게 또는 서서히, 정복이나 쇠퇴,  진화를 통해서) 대체되기도 한다.\hyperlink{42}{$^{42}$} 


144. 과거에는 인간의 본성으로 인해 사회의 발전에 어느 정도 한계가 있었다. 사람들을 밀어붙이는 것도 어느 정도까지만 가능했다. 그러나 오늘날 상황이 바뀌고 있다. 현대 기술이 인간을 개조하는 방법을  개발하고 있기 때문이다.  


145. 사람들을 끔찍하게 불행한 환경에 몰아넣고 나서, 사람들에게 불행을 잊게 만드는 약물을 건네주는  사회를 상상해 보라. 공상과학 소설이냐고? 이건 우리 사회에서 이미 어느 정도 벌어지고 있는 일이다.  최근 몇십 년 간 우울증 발병률이 급속히 증가하고 있음은 널리 알려진 사실이다. 우리는 그것이 문단 59${\sim}$76에서 설명한 대로 권력 과정의 붕괴로 인해 빚어진 것이라고 믿는다. 설령 우리가 틀렸다해도,  우울증의 증가는 분명히 오늘의 사회에 존재하는 어떤 환경이 빚어낸 결과다. 사람들을 절망하게 만드는  환경을 제거하는 대신에, 현대 사회는 사람들에게 항우울제를 건네주고 있다. 실제로 항우울제는 개인의  정신 상태를 개조하는 수단이다. 항우울제를 먹어가며, 그 약을 먹지 않고서는 도저히 참을 수 없는  사회적 환경을 이겨낼 수 있게 되는 것이다.(우리도 우울증이 가끔씩 오로지 유전적 원인에서 비롯된다는  것을 알고 있다. 우리가 여기서 예로 든 것은 환경이 주된 역할을 담당하는 경우의 우울증이다.) 


146. 항정신성 약물은 현대 사회가 개발 중인 새로운 인간 행동 통제 수단들 가운데 하나에 불과하다. 또  다른 수단들을 살펴보기로 하자.  


147. 우선 감시 기술을 들 수 있다. 이제 거의 모든 상점과 기타 많은 장소에 숨겨진 비디오 카메라가  설치되어있다. 컴퓨터는 어마어마한 양의 개인정보를 수집하고 처리한다. 그렇게 얻어진 정보는 (예를  들면 공권력의)물리적 강제력을 대폭 강화시킨다.\hyperlink{43}{$^{43}$}이어서 대중매체가 효과적으로 전달하는,  프로파간다를 들 수 있다. 선거에서 승리하고, 상품을 팔고, 여론을 바꾸기 위한 효과적인 기법들이  개발되어 왔다. 오락 산업은 그것이 엄청난 섹스와 폭력을 퍼부을 때조차도 여전히 체제의 중요한 심리적  도구 역할을 수행한다. 오락은 현대인에게 있어 필수적인 도피 수단이다. 텔레비전과 비디오에 빠져있는  동안 현대인은 스트레스와 불안, 좌절, 불만을 잊어버릴 수 있다. 원시인들은 할 일이 없을 때면 그저 아무 것도 하지 않으면서 몇 시간이고 앉아 있을 수 있었고, 거기에 불만을 품지도 않았다. 자기 자신과 세상에  대해 아무런 갈등없이 살 수 있었기 때문이다. 하지만 현대인은 끝없이 무엇인가에 열중하거나,  오락거리를 갖고 있어야 한다. 그렇지 않으면 금방 “지루함”을 느끼게 되고, 한시도 가만히 있지를 못한다.\hyperlink{44}{$^{44}$}


148. 또 다른 기법들은 감시와 프로파간다를 훨씬 능가한다. 교육은 이제 더 이상 아이가 배운 것을  잊어버리면 엉덩이를 때리고, 잘 기억하면 머리를 쓰다듬어 주는 식의 간단한 일이 아니다. 교육은 이제  어린이의 성장을 통제하는 과학적 기법이 되어 가고 있다. 예를 들어 실반 학습 센터(Sylvan Learning Center)는 아이들이 공부에 몰두하도록 만드는데 커다란 성공을 거두었고, 전통적인 학교에서도 점차  심리적 기법들을 성공적으로 활용하고 있다. 부모들은 자녀들이 체제의 기본적 가치 체계를  받아들이도록 만드는 “양육” 방법을 배우고, 체제가 바람직한 것으로 제시하는 방식대로 행동한다. “정신 건강” 프로그램, “중재” 기법들, 심리 요법 등은 겉으로 보기엔 개인의 행복을 위해 고안된 것 같지만,  사실은 개인들이 사회가 요구하는대로 생각하고 행동하도록 주입하는 수단으로서 이용된다. (이는  모순되지 않는다. 어떤 개인이 지닌 태도와 행동이 체제와 갈등을 일으킬 경우, 그 개인은 도저히 자기  힘으로는 이길 수도 없고, 그렇다고해서 그로부터 달아날 수도 없는 강력한 세력에 대해 저항하고 있는  것이다. 그러니 그 개인은 스트레스와 좌절, 패배감으로 인해 고통받을 수 밖에 없다. 만약 그가 그저  체제가 요구하는대로 생각하고 행동한다면 인생이 훨씬 편해질 것이다. 그런 의미에서, 체제가 개인을  세뇌해 순응하도록 만듦으로써, 개인의 행복을 위한 선행을 베풀고 있는 것이다.) 거의 대부분의  문화권에서는 아동 학대를 금지하고 있다. 거의 모든 사람이 사소한 이유로, 또는 아무 이유 없이  어린이를 매질하는 것에 대해 분노한다. 그런데 심리학자들은 학대의 개념을 훨씬 확대해서 해석한다.  이성적이고 일관된 훈육체계의 한 부분으로서 엉덩이 때리기는 학대에 해당하는가? 이에 대한 대답은  궁극적으로 엉덩이를 때리는 것이 아이의 행동을 현존 체제에 적절한 것으로 만들기 위한 것이냐 아니냐의 여부에 따라 달라진다. 실제로는, “학대”라는 단어는 어린이가 체제와 갈등을 일으키도록  조장하는 모든 양육법으로 해석되는 경향이 있다. 따라서 “아동 학대” 방지 프로그램이 명백하게  무자비하고 잔인한 학대를 막는 범위를 넘어설 때, 그 프로그램은 체제를 위해 인간 행동을 통제하는  방향으로 진행된다.  


149. 예상컨데, 각종 연구를 통해 인간 행동을 통제하기 위한 심리적 기법들의 효과는 끝없이 커지게 될  것이다. 그러나 우리가 생각하기에 심리적 기법만 가지고 지금 기술에 의해 창조되고 있는 사회에 맞춰 인간을 개조하기는 힘들 것 같다. 결국에는 이를 위해 생물학적 방법이 이용될 수밖에 없을 것이다.  우리는 이미 그런 맥락에서 약물이 이용되고 있음을 언급했다. 신경학이 아마 인간 정신을 개조하는 또  다른 방법을 마련해 줄 것이다. 인간 유전공학은 이미 “유전자 치료”이라는 이름 아래 시작되었다. 그리고  그런 방법들이 결국 정신 기능에 영향을 미치는 신체적 부분까지 개조하는 지경에까지 이르지는 않을  것이라고 믿을 근거는 없다. 


150. 문단 134에서 언급했듯이, 산업사회는 한편에는 인간 행동의 문제 때문에, 또 한편에는 경제  문제와 환경 문제 때문에, 심각한 스트레스를 겪는 단계로 진입하게 될 것으로 보인다. 그리고 체제의  경제 문제와 환경 문제의 상당 부분은 인간의 행동 양식에 의해 빚어진 문제다. 소외, 자기 비하, 절망,  적대감, 반항, 공부 안하는 아이들, 청소년 조직범죄, 불법 약물 복용, 강간, 아동 학대, 기타 범죄, 
무분별한 섹스, 10대 임신, 인구 증가, 정치적 부패, 인종 갈등, 민족 갈등, (낙태 찬성론과 반대론 같은) 극심한 이념 갈등, 정치적 극단주의, 테러리즘, 사보타주, 반정부 단체, 증오 단체 등 이 모든 것들이  체제의 생존 자체를 위협한다. 체제는 결국 어쩔 수 없이 인간 행동을 통제하기 위해 가능한 모든 수단을  총동원해야만 할 것이다.  


151. 오늘날 우리가 보고 있는 사회 붕괴 현상은 분명 단순한 우연의 결과가 아니다. 체제가 사람들에게  강요하는 삶의 조건들이 빚어낸 결과일 뿐이다(우리는 앞에서 이 조건들 중 가장 중요한 것이 권력 과정의 붕괴임을 밝혔다.) 만약 체제가 인간 행동을 완전히 통제하는데 성공한다면, 그 때 인간의 역사는 새로운  
분수령을 넘어서게 될 것이다. 예전에는 인간이 지닌 인내심의 한계 때문에 (문단 143${\sim}$144에서  설명한대로) 사회의 발전에도 한계가 있을 수 밖에 없었다. 그러나 산업-기술 사회는 심리적 방법이나  생물학적 방법, 아니면 그 둘을 사용해 인간을 개조함으로써 그런 한계를 뛰어넘을 수 있게 될 것이다.  미래의 사회 체제는 인간의 욕구에 맞춰 적응하지 않는다. 반대로 인간이 체제의 욕구에 맞춰 적응하게 될 것이다.\hyperlink{45}{$^{45}$} 


152. 인간 행동에 대한 기술의 통제가 전체주의적 의도에서, 또는 인간의 자유를 제한하려는 의식적  욕망에서 시작되지는 않을 것이다.\hyperlink{46}{$^{46}$} 인간 정신에 대한 통제가 하나씩 새롭게 등장할 때, 그것은 사회가  직면한 문제를 해결하기 위한 합리적 반응으로서 간주될 것이다. 알코올 중독을 치료한다든지, 범죄를  줄인다든지, 젊은이들을 과학과 공학 연구에 몰두하도록 한다든지하는 식으로 말이다. 많은 경우에  통제는 인도주의라는 가면을 쓰고 나타날 것이다. 예를 들어 정신과 의사가 우울증 환자에게 항우울제를  처방해 줄 때, 의사는 분명히 그 개인에게 호의를 베풀고 있는 것이다. 약이 필요한 사람으로부터 약을  빼앗아 버리는 것이야말로 비인간적인 짓이다. 부모들이 자녀를 실반 학습 센터에 보내 공부에 열광하는  아이들로 만드는 것은, 부모들이 아이들의 행복을 바라기 때문이다. 그런 부모들 중에 어떤 사람은 자신이 직업을 얻기 위해 받아야만 했던 전문가 훈련을 다른 이들은 더 이상 받지 않기를, 그리고 자기 자식은 더  이상 세뇌를 통해 컴퓨터 오타쿠가 되는 일이 없기를 바랄지도 모른다. 하지만 그런 부모가 대체 무엇을  할 수 있는가? 그들은 사회를 바꿀 수 없다. 그리고 그들의 아이가 기술을 익히지 못하면 실업자가 될지도 모른다. 그래서 아이를 실반 학습 센터에 보낸다.  


153. 인간 행동에 대한 통제는 정부 당국의 치밀한 계산에 의해서가 아니라, 사회의 (급격한) 진화의  과정을 통해서 시작될 것이다. 그 과정에 저항하기는 불가능할 것이다. 각각의 진보는 그 자체만 놓고 볼 때는 이익이 될 것으로 보이기 때문이다. 아니면 최소한, 진보를 이루는 과정에서 생겨나는 해악이  나중에는 이익이 되거나, 진보하지 않을 때 빚어지는 해악보다는 진보하는 과정에서 생겨나는 해악이 덜 해롭기 때문일 수도 있다(문단 127을 볼 것). 프로파간다만 해도 아동 학대 금지나 인종 분쟁 근절과 같이 여러 가지 좋은 목적으로 이용될 수 있다.\hyperlink{47}{$^{47}$} 성교육은 분명히 유용한 것이지만, 성교육으로 인해 (그것이  성공할 경우) 아이들에게 성에 관한 태도를 심어주는 일은 가족의 손을 떠나 공교육을 통해 대표되는  국가의 손에 넘어가 버린다. 


154. 가령 어린이가 범죄자로 성장할 가능성을 높이는 생물학적 요인이 발견된다고 가정해보자. 그리고  모종의 유전자 치료법이 그런 요인을 제거할 수 있다고 가정해보자.\hyperlink{48}{$^{48}$} 물론 그런 요인을 가진 아이들의  부모는 당연히 아이들을 치료할 것이다. 그렇게 하지 않는다면, 그것은 비인간적인 행위이다. 아이가 커서 범죄자가 되면 불행한 삶을 살아갈 것이기 때문이다. 하지만 대부분의 원시 사회에는 첨단 자녀 양육법이나 가혹한 처벌 없이도, 범죄률은 우리 사회에 비해 훨씬 낮다. 현대인이 원시인보다 훨씬 강한  가학적 성향을 지니고 있다고 가정할 근거는 없다. 따라서 우리 사회의 높은 범죄율은 현대적 환경이  사람들에게 가하는 압박 때문임이 틀림없다. 많은 사람들은 그 압박에 적응할 수 없고, 아니면 아예  적응하려고 하지도 않는 것이다. 그러니 잠재적인 범죄 성향을 제거하기 위한 치료법은, 최소한  부분적으로라도 사람들을 개조하는 식으로 이루어질 것이고, 그럼으로써 사람들은 체제가 요구하는  조건에 자신을 맞춰 갈 것이다.  


155. 우리 사회는 체제와 충돌을 일으키는 모든 종류의 생각이나 행동을 “질병”으로 간주한다. 어느  개인이 체제에 적응하지 못할 때, 그것은 체제에 문제가 될 뿐만 아니라 개인도 고통을 겪게 되므로, “질
병”으로 간주해도 무방하다. 그러니 체제에 적응하도록 개인을 조작하는 것은 “질병”에 대한 “치료”이고,  좋은 일이다.  


156. 문단 127에서 우리는 새로 등장한 도구의 사용 여부가 처음에는 선택 사항이었다 해도, 이후에도  선택 사항으로 남지는 않는다는 사실을 지적했다. 새로운 기술이 그 기술을 이용하지 않고는 개인이  기능을 수행하기가 어렵거나 불가능한 방식으로 사회를 변화시키기 때문이다. 이는 인간 행동을  통제하는 기술에도 똑같이 적용된다. 대부분의 아이들이 공부를 좋아하도록 만드는 프로그램에  참가하는 세상에서는, 어느 부모라도 어쩔 수 없이 자기 아이를 그 프로그램에 보내야 한다. 그렇지  않으면 아이가 열등생이 될 것이고, 따라서 직업도 얻지 못할 것이기 때문이다. 아니면 우리 사회의  수많은 사람들을 괴롭히고 있는 심리적 스트레스를 부작용 없이 크게 줄여 줄 수 있는 생물학적 치료법이  개발되었다고 가정해 보자. 많은 사람들이 치료를 받는다면 사회의 전반적인 스트레스 수준이 떨어질  것이고, 그러면 사회는 다시 스트레스를 유발하는 억압을 증가시킬 수 있을 것이다. 실제로 그런 일이  이미 우리 사회에서 벌어지고 있다. 사람들의 스트레스를 덜어 주는(혹은 일시적으로 스트레스를 잊게  해주는) 우리 사회의 가장 중요한 심리적 도구의 하나인 이른바 대중 오락에서 벌어지는 일이다(문단 147 을 볼 것). 우리가 대중 오락을 이용하는 것은 “선택” 사항이다. 우리에게 텔레비전을 보고, 라디오를 듣고, 잡지를 읽으라고 요구하는 법률은 없다. 그런데도 우리들 대부분은 스트레스 해소, 도피 수단으로 점점 더 대중 오락에 의존하고 있다. 모든 사람이 TV 방송이 쓰레기라고 불평한다. 그러나 거의 모든 사람이  텔레비전을 본다. 소수의 사람들만이 텔레비전을 보는 습관을 벗어 던졌다. 하지만 대중 오락을 전혀 이용하지 않으면서 오늘날을 살아갈 수 있는 사람은 거의 없을 것이다. (불과 얼마 전까지만 해도  대부분의 사람들은 지역 공동체가 스스로 창조한 오락거리 말고는 일체의 오락없이 아주 잘 살아왔다.)  만약 오락 산업이 없었다면, 체제는 지금처럼 우리에게 엄청난 스트레스를 일으키는 억압을 가할 수도  없었을 것이다.  


157. 산업 사회가 살아남는다고 가정하면, 기술은 결국 인간 행동에 대해 완전한 통제를 행사할 방법을  찾아낼 것이다. 인간의 생각과 행동이 상당 부분 생물학적 토대를 갖고 있다는 점에는 지금까지 별다른  이론(異論)이 없었다. 연구자들이 밝혀 냈듯, 뇌의 특정 부분에 전기 자극을 가함으로써 배고픔, 기쁨,  분노, 공포 등의 감정을 불러일으킬 수도 있고 가라앉힐 수도 있다. 뇌의 어느 부분을 다치면 기억이  사라져 버릴 수도 있으며, 전기 자극에 의해 그렇게 잊혀진 기억이 다시 떠오를 수도 있다. 약물로 환각을  일으킬 수 있으며, 분위기를 바꿀 수도 있다. 비물질적인 영혼이 있을 수도, 없을 수도 있다. 하지만 설령 영혼이 있다고 해도 인간 행동이 지닌 생물학적 메커니즘의 힘보다는 분명히 약할 것이다. 그렇지  않고서야 약물과 전기를 가지고 인간의 감정과 행동을 그렇게 쉽사리 조작할 수는 없을 것이다.  


158. 모든 사람이 머리 속에 전기 장치를 달고 정부 당국에 의해 통제될 것이라는 주장은 아마 황당한  소리일 것이다. 하지만 인간의 생각과 행동이 생물학적 간섭에 그토록 무방비 상태에 놓여 있다는 사실은  인간 행동 통제라는 문제가 기본적으로 기술적 문제임을 보여준다. 문제는 뉴런과 호르몬, 고분자이고,  과학은 얼마든지 그런 문제에 달려들 수 있다. 우리 사회가 기술적 문제의 해결에 보여준 뛰어난 성적을  생각하면, 인간 행동의 통제에도 엄청난 진보가 이루어질 것을 쉽게 짐작할 수 있다.  


159. 대중적 저항을 통해 기술의 인간 행동 통제를 막을 수 있을까? 만약 통제가 모조리 한꺼번에  시작된다면 막을 수도 있을 것이다. 그러나 기술의 통제는 장기간에 걸친 점진적 진보를 통해 등장한다.  따라서 합리적이고 효과적인 대중적 저항도 있을 수 없다.(문단 127, 132, 153을 볼 것.)  


160. 이런 이야기가 전부 공상과학 소설처럼 들린다는 사람들에게, 우리는 어제의 공상과학 소설이  오늘의 현실이라는 이야기를 해주고 싶다. 산업 혁명은 인간의 환경과 생활 양식을 근본적으로 바꾸어  놓았다. 이제 기술이 점차 인간의 육체와 정신에도 적용되리라는 것은 분명하다. 환경과 생활 양식이  근본적으로 변한 것처럼, 인간 그 자체도 역시 근본적으로 변하게 될 것이다. 


\section*{갈림길에 선 인류} 
161. 이야기가 너무 멀리 나갔다. 문제는 두 가지다. 하나는 실험실에서 인간 행동을 조작하기 위한  심리적 또는 생물학적 기술을 개발하는 것이고, 또 하나는 그런 기술을 사회 체제의 기능에 통합시키는  것이다. 두번째 문제가 더 어렵다. 예를 들어, 교육 심리학의 기술은 그것이 개발된 “실험 학교”에서는  두말할 것도 없이 아주 잘 돌아갔겠지만, 그것을 우리의 전체 교육 체제에 효과적으로 적용하기가 반드시  쉽지만은 않을 것이다. 우리는 우리의 학교가 어떤 것인지를 잘 알고 있다. 교사들은 아이들에게서 칼과  권총을 빼앗느라 바빠 아이들을 컴퓨터 오타쿠로 바꿀 새로운 교육법을 써볼 시간이 없다. 인간 행동과  관련된 그 모든 기술적 진보에도 불구하고, 아직까지 체제는 인간 행동을 통제하는데 그다지 크게  성공하지 못했다. 그래도 상당히 고분고분하게 체제의 통제에 따라 행동하는 사람들은, 이른바 “부르주 아”라고 불리는 사람들이다. 그러나 한편 어떤 식으로든 체제에 저항하는 반항아들도 점차 늘어나고 있다. 복지 거머리, 청소년 폭력배, 광신도, 악마 숭배자, 나치, 급진적 환경주의자, 민병대 등이 그  반항아들이다.  


162. 체제는 지금 자신의 생존을 위협하는 문제들을 상대로 필사적인 투쟁을 벌이고 있다. 그 중에 가장  중요한 문제는 바로 인간 행동의 문제다. 만약 체제가 빠른 시일 내에 인간 행동을 통제할 효과적인  수단을 찾아낸다면, 체제는 살아남을 수 있을 것이다. 그렇지 않으면 체제는 붕괴한다. 우리는 앞으로  몇십 년 안에, 즉 40년에서 100년 사이에 결과가 나오리라고 생각한다.  


163. 체제가 앞으로 수십 년 간의 위기에서 살아남는다고 가정해 보자. 그 때 쯤이면 사회는 중요한  문제들을 다 해결했거나, 최소한 통제는 할 수 있을 것이다. 체제가 해결한 문제 중에는 특히 인간을 `사회화'하는 문제, 즉 더 이상 그들의 행동이 사회에 위협이 되지 않도록 사람들을 순화시키는 문제가  포함된다. 일단 그렇게 되면 더 이상 기술의 발전을 가로막는 장애물은 없을 것처럼 보인다. 기술은  인간과 그 외 모든 주요 생명체를 포함한 지구 전체를 완벽하게 통제하는 마지막 논리적 귀결점을 향해  진보해 나갈 것이다. 체제는 하나의 단일한 독재적 조직이 될지도 모른다. 아니면 오늘날 정부와 기업들,  그 외 다른 대형 조직들이 서로 협동하고 동시에 경쟁하듯, 협동과 경쟁 관계를 맺고 공존하는 몇 개의  분리된 조직들로 나누어질 수도 있다. 인간의 자유는 거의 전부 사라져 버릴 것이다. 개인들과 작은  집단들이, 슈퍼 테크놀로지와 온갖 감시 장치로 무장한 물리적 강제력은 물론, 인간을 조작하기 위한  수많은 심리학적, 생물학적 수단을 갖춘 거대 조직과 맞서 싸우기란 불가능하기 때문이다. 오직 소수의  사람들만이 진짜 권력을 갖게 될 것이다. 그러나 이들조차 아주 제한된 자유만을 누릴 것이다. 그들의  행동에도 역시 규제는 가해질 것이기 때문이다. 오늘날과 똑같이 정치가들과 기업 임원들은, 그들의  행동이 좁은 한계에 머물러 있는 한, 여전히 권력있는 지위를 유지할 수 있을 것이다.  


164. 다가오는 몇십 년 간의 위기가 끝나고, 체제의 생존을 위해 통제를 강화하는 것이 불필요해진다고  해서, 체제가 인간 행동과 자연을 통제하기 위한 기술 개발을 멈출 것이라고는 꿈도 꾸지 말라. 오히려,  일단 시련기가 끝나면 체제는 인간과 자연에 대한 통제를 더욱 빨리 강화시켜 갈 것이다. 지금 체제가  겪고 있는 어려움이 더 이상 없을 것이기 때문이다. 체제가 생존만을 위해 통제를 강화하는 것은 아니다.  문단 87${\sim}$90에서 설명했듯, 기술자들과 과학자들은 대리 활동으로서 자신의 일을 수행한다. 즉, 기술적인 문제를 해결함으로써 권력욕을 충족시키는 것이다. 그들은 미래에도 여전히 식지 않는 열정을 가지고 그  일을 계속할 것이다. 그들에게 있어 가장 흥미롭고 도전해 볼 만한 문제는 인간의 육체와 정신을 이해하는 것, 그리고 그 성장에 간섭하는 일이 될 것이다. 물론 "인류의 행복"을 위해서 말이다.  


165. 반대로, 이번엔 다가오는 몇십 년 간의 스트레스를 체제가 버티지 못하는 경우를 가정해 보자. 만약 체제가 붕괴하면, “혼란의 시대”, 즉 과거의 여러 시대 있었던 것과 유사한 혼란기가 닥쳐올 것이다. 그런  혼란의 시대에 무슨 일이 벌어질지는 아무도 알 수 없다. 하지만 어찌됐든 인류에겐 새로운 기회가 주어질 
것이다. 가장 위험한 시나리오는 산업사회가 붕괴 이후 몇 년 내에 다시 건설되기 시작하는 것이다.  분명히 많은 사람들(특히 권력에 눈이 벌게진 족속들)은 공장들이 다시 돌아가기를 애타게 기다릴 것이  틀림없다. 


166. 따라서 산업 체제에 의한 인류의 노예화를 증오하는 사람들에게는 두 가지의 임무가 주어진다.  첫째, 우리는 체제 내부의 사회적 스트레스를 더욱 증가시켜야한다. 그렇게 함으로써 체제가 붕괴될  가능성을 높이거나, 아니면 체제에 저항하는 혁명이 일어날 수 있을 정도로 체제를 약화시켜야 한다. 
둘째, 체제가 충분히 약해질 때를 대비해 기술과 산업사회를 공격하는 이념을 개발하고 널리 확산시켜야  한다. 그런 이념은, 산업사회가 붕괴되었을 때, 남겨진 잔재를 아예 복구조차 불가능하게 박살내도록  만들어 줄 것이고, 그러면 체제의 재건설도 불가능해질 것이다. 공장들은 파괴되어야하며, 기술 서적들은  불태워 버려야 한다.  


\section*{인간의 고통} 
167. 순전히 혁명적 행동으로만 산업 체제를 무너뜨릴 수는 없다. 산업 체제는 내재적인 발전의 문제로  인해 심각한 장애에 도달하지 않는 한, 혁명의 공격에 끄떡도 하지 않을 것이다. 그러니 만약 체제가  붕괴한다면, 그것은 갑자기 우발적으로 붕괴하거나, 아니면 부분적으론 우발적이되 혁명가들의 힘이  보태지는 일련의 과정을 통해 붕괴할 것이다. 만약 붕괴가 갑작스럽게 이루어지면, 많은 사람들이 죽게 될 것이다. 세계의 인구는 첨단 기술 없이는 먹여 살릴 수 없을 만큼 과도하게 늘어나 있기 때문이다. 설령 사망률을 올리는 대신 출산율을 낮춤으로써 인구가 줄어들 수 있을 만큼 점진적으로 붕괴가 이루어진다  해도,\hyperlink{49}{$^{49}$} 그런 탈산업화의 과정은 여전히 심한 혼란에 빠질 것이고, 커다란 고통을 수반할 것이다. 기술을  무리 없이 질서정연한 방식으로 제거할 수 있으리라는 생각은 너무 순진한 생각이다. 무엇보다도, 기술  숭배자들이 매 단계마다 격렬히 저항할 것이기 때문이다. 그렇다면 체제의 붕괴를 위해 일한다는 것은  잔인한 짓인가? 그럴 수도 있지만, 그렇지 않다. 첫째, 혁명가들이 체제를 무너뜨리는 것은, 오직 체제가  이미 심각한 문제에 빠져 어차피 결국은 무너지게 되어 있는 상황에서만 가능한 일이다. 한편 체제가 더 거대해질수록 그 붕괴로 인해 초래될 결과도 더욱 참혹해진다. 따라서 혁명가들이 붕괴를 앞당기는 것은 오히려 재앙을 줄이는 길이 될 수도 있다.  


168. 둘째, 우리는 자유와 존엄성의 상실에 대항한 투쟁과 죽음 사이에서 균형을 잘 잡아야 한다. 우리  대부분에게는 오래 살거나 육체적 고통을 피하는 것보다는, 자유와 존엄성이 더 중요하다. 게다가, 우리  모두는 언젠가 죽는다. 공허하고 목적 없는 삶을 오래 사느니, 생존을 위해서건 대의를 위해서건 싸우다  죽는 편이 더 나을 것이다.  


169. 셋째, 체제가 살아남는다고 해서, 붕괴될 때보다 더 고통이 적으리라는 보장은 없다. 체제는 이미  지금까지도 엄청난 고통을 전 세계에 안겨 주었고, 지금도 계속 고통을 주고 있다. 수백년에 걸쳐  사람들이 다른 사람들 및 자연과 행복한 관계를 맺으며 살아왔던 고대 문명들은, 산업사회와의 접촉을  통해 일시에 와해되고 말았다. 그리고 그 결과는 만연한 경제 문제, 환경 문제, 사회 문제, 심리  문제들이다. 산업사회의 지배로 인해 빚어진 가공할 결과의 하나는, 전 세계의 전통적 인구 조절 방법들이 전혀 기능을 발휘할 수 없게 되었다는 것이다. 그 때문에 오늘의 인구 폭발이 벌어졌다. 이어서  행운아라고 여겨졌던 서구 국가들에까지 심리적 고통이 확산되었다(문단 44,45를 볼 것). 1995년 현재  오존층 감소, 온실 효과, 그리고 그 외 아직까지 예측 불가능한 환경 문제들로 인해 어떤 결과가 빚어질지  아무도 모른다. 핵 확산에서 볼 수 있듯, 새로운 기술을 독재자와 무책임한 제3세계 국가들의 손에서  안전하게 떼어놓을 수도 없다. 이라크나 북한이 유전공학을 가지고 무슨 짓을 할지 한번 생각해 보라.  


170. “아!” 기술 숭배자들은 이렇게 말한다. "과학이 모든 문제를 해결할 것 입니다! 우리는 기아를 정복하고, 심리적 고통을 제거할 것이고, 모든 사람을 건강하고 행복하게 만들어줄 것 입니다!" 그래 그렇고  말고. 그들은 200년 전에도 똑같은 말을 했다. 예정대로라면 산업혁명은 가난을 몰아내고 모든 사람을  행복하게 해 주었어야 했다. 실제의 결과는 전혀 딴판이었다. 기술 숭배자들은 사회 문제를 이해하는 데  있어서 (스스로를 속이고 있거나)절망적일 정도로 순진하다. 그들은 사회에 거대한 변화, 겉보기에  바람직한 변화일지라도 일단 변화가 시작되면 예측 불가능한 수많은 변화들이 연이어 일어난다는 사실을  (일부러 무시하거나)모르고 있다.(문단 103을 볼 것.) 그 결과는 사회의 붕괴다. 기술 숭배자들이 가난과  질병을 몰아내고, 유전공학을 통해 온순하고 행복한 사람을 만들어 내려고 시도할 때, 그들은 지금보다  훨씬 더 끔찍하게 혼란스러운 사회 체제를 만들어 낼 것이다. 과학자들은 새로운 유전공학적 식량 생산  공장을 만들어 기아 문제를 해결할 수 있다고 큰소리친다. 하지만 그렇게 되면 인구가 무한정 늘어나게 될 것이다. 그리고 인구과밀이 스트레스와 공격성 증가로 이어진다는 것은 널리 알려진 사실이다. 이것은  그저 예측할 수 있는 문제들 중 하나에 불과하다. 우리는 과거의 경험이 보여주듯, 기술 진보가 예측  불가능한 새로운 문제들을 낳을 것임을 다시 한번 강조한다(문단 103을 볼 것). 실제로 산업혁명 이후 기술은 낡은 문제를 해결하는 속도보다 훨씬 더 빠른 속도로 새로운 문제를 만들어 왔다. 그러니 설령 기술 숭배자들이 그들의 “멋진 신세계”에서 결함을 제거하는 데는 성공한다 하더라도, 아주 길고 힘든 시행착오의 시간이 걸릴 것이다. 그 와중에 엄청난 고통이 따른다. 산업사회가 생존할 때의 고통이, 붕괴될 때의 고통보다 덜하리라는 보장은 전혀 없다. 기술은 지금까지 인류를 도저히 탈출할 수 있을 것  같지 않은 곤경으로 몰아왔다.  


\section*{미래\hyperlink{50}{$^{50}$}} 
171. 이제 산업 사회가 앞으로의 수십 년을 살아남고 마침내 체제에서 결함을 제거하고 제대로 기능을  수행하게 된다고 가정해 보자. 그 체제는 과연 어떠한 것일까? 몇 가지 가능성을 생각해 보자.  


172. 우선 컴퓨터 과학자들이 모든 일을 인간보다 잘 처리하는 인공지능 기계를 개발하는데 성공한다고  가정해 보자. 그렇게 되면 거대하고 고도로 조직적인 기계 시스템이 모든 노동을 담당할 것이고, 인간의  노력은 필요없게 될 것이다. 그런 경우의 가능성은 두 가지 중 하나다. 인간의 감독 없이 기계가 스스로  모든 결정을 내리거나, 아니면 여전히 인간이 기계에 대한 통제권을 유지하는 것이다.  


173. 기계가 스스로 모든 결정을 내리게 된다면, 우리는 어떤 결과가 빚어질지 전혀 예측할 길이 없다.  그런 기계가 어떤 식으로 행동할지 짐작도 할 수 없기 때문이다. 우리가 알 수 있는 사실은 다만 인류의  운명이 기계의 자비심에 달려 있다는 것 뿐이다. 인류가 기계에게 모든 힘을 넘겨줄 정도로 멍청하지는  않다는 반론이 있을 수도 있다. 하지만 우리는 인류가 자발적으로 기계에게 힘을 넘겨주거나 기계가  
스스로의 의지로 권력을 장악하리라고는 생각하지 않는다. 우리가 생각하는 것은, 인류가 쉽사리 기계에  종속된 지위로 떨어질 것이며, 결국 기계의 결정을 받아들일 수밖에 없는 상황이 오리라는 것이다. 사회가 복잡해지고, 따라서 사회 문제들도 점점 더 복잡해짐에 따라, 그리고 기계가 점점 더 지능화함에 따라, 사람들은 점점 더 많은 결정권을 기계에게 넘겨줄 것이다. 단순히 기계에 의한 결정이 사람에 의한  결정보다 더 나은 결과를 낳을 것이라는 이유 하나만으로 말이다. 마침내는 체제를 계속 돌아가게 하기  위해 필요한 결정이 너무나 복잡해져서 인간의 지능으로는 아무런 결정도 내릴 수 없는 그런 단계가  도래할 것이다. 그 단계에서는 기계가 통제권을 장악한다. 이제 인간은 기계를 꺼 버릴 수조차 없다.  기계에 철저히 종속된 인간이 기계를 끈다는 것은 곧 자살 행위가 될 것이기 때문이다.  


174. 반대로 인간이 기계에 대한 통제권을 계속 유지할 가능성도 있다. 그런 경우, 보통 사람도 자동차나  PC 같은 개인 소유 기계는 통제할 수 있겠지만, 대형 기계 시스템에 대한 통제권은 극소수 엘리트의 손에  쥐어지게 될 것이다. 오늘날과 비슷한 상황이지만, 거기엔 두 가지 차이점이 있다. 진보된 기술 덕분에  엘리트는 대중에 대해 더 강화된 통제권을 갖게 된다. 그리고 인간의 노동이 불필요해진 탓에 대중은  불필요한 존재, 즉 체제에 떠넘겨진 쓸모 없는 짐더미가 되어 버린다. 무자비한 엘리트라면, 간단히  엄청난 인구를 죽여 없앨지도 모른다. 인간적인 엘리트라면 프로파간다나 심리적, 생물학적 기술을  활용해 출산율을 줄이는 방법으로 대부분의 인구를 멸종에 이르게 한 뒤, 남은 세상을 독차지할 것이다.  만약 엘리트를 구성하는 사람들이 마음 약한 리버럴들이라면 그들은 나머지 인류의 선한 목자 역할을  하겠다고 나설 것이다. 그들은 모든 사람의 신체적 욕구가 충족되고 있는지, 모든 아이들이 심리학적으로  위생적인 환경에서 자라고 있는지, 모든 사람이 유익한 취미 생활로 바쁘게 지내고 있는지, 만족하지  못하는 사람은 제대로 “문제”를 고치는 “치료”를 받고 있는지 꼼꼼히 챙길 것이다. 물론 삶은 너무나  무의미해졌으므로, 사람들은 권력 과정에 대한 욕구를 제거하거나, 안전한 취미로 권력 욕망을 “승화” 시킬 수 있도록 생물학적으로든 심리적으로든 공학적 처치를 받아야 한다. 이들 공학적 처치를 받은  사람들은 해당 사회 안에서 행복하긴 하겠지만 결코 자유롭지는 않다. 그들은 가축의 신분으로 전락한  것이다.  


175. 이번엔 컴퓨터 과학자들이 인공지능을 개발하는데 실패하고, 따라서 인간의 노동이 계속 필요한  경우를 가정해 보자. 그래도 기계는 여전히 점점 더 많은 단순 작업을 떠맡을 것이고, 그에 따라 단순 직종에서는 잉여 노동력이 늘어나게 될 것이다. (이 일은 이미 벌어지고 있다. 현재의 체제 안에서 계속  쓸모 있는 존재로 남으려면 반드시 받아야 하는 훈련을 지적 또는 심리적 이유로 인해 받지 못해 직업을  얻지 못하는 사람은 이미 상당수에 달한다.) 직업을 가진 사람들에겐 끝없는 임무가 부과된다. 그들에겐
더 많은 훈련과 더 많은 능력이 필요하며, 점점 더 거대한 유기체의 세포 같은 존재가 되어 갈 것이므로  더욱 더 믿을만해져야 하고, 순응적이고, 온순해져야 한다. 그들의 작업은 더욱 더 전문화되어 간다.  그래서 그들이 현실의 아주 작은 한 조각에만 몰두하는 탓에, 그들의 노동은 현실 세계와 단절돼 버린다.  체제는 심리적 수단이건, 생물학적 수단이건 가능한 모든 수단을 이용해서 사람들을 공학적으로 처치해 온순하게 만들고, 체제가 요구하는 능력을 갖도록 만들고, 권력 욕망을 전문화된 작업으로 "승화"시키도록 만들어야 한다. 그런데 그런 사회의 사람들이 반드시 온순해져야만 하는가에는 이론이 생겨날 수도 있다. 사회는 경쟁이 쓸모있다는 것을 알아차리고 체제가 필요로 하는 방향으로 경쟁을 유도할 수도 있을 것이다. 우리는 권력을 행사할 수 있는 지위를 차지하기 위해 끝없는 경쟁이 펼쳐지는 미래 사회를 상상할 수 있다. 하지만 진정한 권력을 행사할 수 있는 정상의 자리에 도달할 수 있는 사람은 극소수에 불과할  것이다(문단 163의 마지막 부분을 볼 것). 한 사람이 자신의 권력 욕구를 충족시키기 위해서 수많은  사람들을 밀쳐내고, 권력에의 기회를 박탈해야만 하는 사회. 참으로 끔찍한 사회다.  


176. 지금까지 이야기한 여러 가능성들이 서로 결합되는 시나리오도 머리 속에 그려 볼 수 있다. 예를  들어, 기계가 실질적인 노동을 전부 떠맡는 대신, 인간은 상대적으로 중요하지 않은 노동에 종사하면서도 바쁘게 뛰어다녀야 하는 상황이 생겨말 수 있다. 서비스 산업의 엄청난 발전에 따라 인간을 위한 직업이 계속 생겨나리라는 주장은 이제까지 줄곧 제기되어 왔다. 그렇게 되면 사람들은 서로의 구두를 닦아주며,  서로를 택시에 태워주며, 서로를 위한 수공예품을 만들며, 서로의 테이블에서 음식 주문을 기다리며  인생을 보내게 될 것이다. 인류가 이런 꼴이 되다니, 얼마나 경멸스러운 일인가. 그런 아무런 의미 없는  일에 허겁지겁 살면서 만족스러운 삶을 살고 있다고 느낄 사람이 얼마나 될지 의심스럽다. 생물학적 또는  심리적인 공학적 처치를 통해 그런 생활 양식에 적합하게 자신을 짜맞추지 않는 한, 사람들은 (마약, 범죄, “광신”, 증오 단체 등) 위험한 배출구를 찾게 될 것이다.  


177. 두말 할 필요도 없이, 위에서 이야기한 시나리오들에 모든 가능성이 포함되어 있는 것은 아니다. 그  시나리오들은 그저 우리가 보기에 가장 가능성이 높아 보이는 결과들을 보여줄 뿐이다. 우리는 그  시나리오들보다 더 바람직한 장밋빛 시나리오를 상상할 수가 없다. 만약 산업-기술 체제가 다가오는  40${\sim}$100년을 살아남는다면, 그 때쯤이면 체제는 틀림없이 다음과 같은 일반적 특징들을 거의 확립시켜 놓고 있을 것이다. 개인들은(적어도 체제에 통합되고, 체제를 움직이게 하고, 따라서 모든 권력을 쥐고  있는 “부르주아” 유형의 개인들은) 그 어느 때보다 거대 조직에 강하게 종속되어 있을 것이다. 그들은 그  어느 때보다 심하게 “사회화”될 것이며, 그들의 신체적, 정신적 능력은 상당한 정도로(아마 엄청난  정도로) 우연(또는 신의 의지, 아니면 다른 무언가)의 산물이 아니라 공학의 산물이 될 것이다. 그  무엇이라도 자연에 남겨진 것이 있다면, 그것은 과학 연구를 위해 보존된 잔재에 불과할 것이며,  과학자들이 그것을 감독하고 관리할 것이다. (그러니 더 이상 그것은 순수한 자연이 아니다.) 마지막에는 (지금으로부터 몇백년 뒤) 인간이든 아니면 다른 주요 생물이든 간에 오늘 우리가 알고 있는 형태로  존재하는 생명체는 하나도 남아 있지 않을 것이다. 일단 유전공학을 통해 생물을 개조하기 시작하면, 어느 시점에서 멈춰야 할 이유가 없다. 그러니 아마도 인간과 다른 생물들을 완전히 변형시킬 때까지 개조 작업은 계속될 것이다. 


178. 그 밖에 다른 어떤 경우가 생겨나든 간에, 분명한 것은 기술을 통해 인간이 완전히 새로운 물리적,  사회적 환경을 창조하고 있다는 사실이다. 그 새로운 환경은 인류가 자연선택에 따라 육체적, 심리적으로  적응해 온 환경과는 근본적으로 다른 환경이다. 만약 인간이 스스로를 인위적으로 개조함으로써 이  새로운 환경에 적응하지 않는다면, 길고 고통스러운 자연선택 과정을 거쳐야만 그 환경에 적응할 수 있을  것이다. 후자보다는 전자가 일어날 확률이 훨씬 더 높다.  


179. 이 역겨운 체제를 무너뜨리고, 그 결과를 받아들이는 편이 나을 것이다.  


\section*{전략}
180. 기술 숭배자들은 우리 모두를 지극히 위험한 놀이기구에 태우고 전혀 알려지지 않은 세계로 끌고  가고 있다. 많은 사람들이 기술의 진보로 인해 우리에게 어떤 일이 벌어지고 있는지를 알면서도, 여전히  그것이 불가피하다는 이유를 들어 소극적인 자세를 취하고 있다. 하지만 우리 FC는 그것이 불가피하다고  생각하지 않는다. 우리는 그것을 멈출 수 있다고 생각하며, 그것을 멈추기 위해 어떻게 해야 할 것인지에  대해 몇가지 방법을 이야기하고자 한다.  


181. 문단 166에서 말했듯, 현재 주어진 두 가지 중요한 임무는 첫째, 산업 사회의 사회적 스트레스와  불안정성을 증가시키는 일과 둘째, 기술과 산업 체제에 저항하는 이념을 개발하고 퍼뜨리는 일이다.  체제가 충분히 스트레스 받고 불안정해지면, 기술에 저항하는 혁명이 가능해질 것이다. 혁명의 양상은  프랑스 혁명 및 러시아 혁명과 비슷할 것이다. 프랑스와 러시아 사회는 저 존경할만한 혁명이 일어나기  수십 년 전부터 이미 점점 늘어나는 스트레스와 취약성을 보여주고 있었다. 한편 혁명가들에 의해 개발된 이념들은 낡은 세계관과는 전혀 다른 새로운 세계관을 제시했다. 러시아 혁명의 경우, 혁명가들은 낡은  질서를 깨뜨리기 위해 열정적으로 뛰어들었다. 그러다가 낡은 체제에 (프랑스에서의 재정 위기, 러시아의 전쟁 패배 같은)충분한 스트레스가 더해졌을 때 혁명은 낡은 체제를 휩쓸어 버렸다. 우리가 제안하는 것도 그와 동일한 수순을 따른다.  


182. 프랑스 혁명과 러시아 혁명은 실패한 혁명이라는 반론을 제기하는 사람도 있을 것이다. 하지만  혁명에는 대개 두 가지 목표가 있다. 하나는 낡은 형태의 사회를 파괴하는 것이고, 다른 하나는 혁명가의  이상에 따라 새로운 형태의 사회를 건설하는 것이다. 프랑스 혁명과 러시아 혁명은 그들이 꿈꾸었던 새로운 사회를 창조하는데는 (다행히!)실패했지만, 낡은 사회를 파괴하는데는 상당한 성공을 거두었다.  우리는 새로운 이상 사회를 창조하는 것이 얼마나 어려운 일인지를 잘 알고 있다. 우리의 목표는 그저  현재의 사회를 파괴하는 것이다. 


183. 그러나 어떤 이념이 사람들로부터 열정적인 지지를 얻기 위해서는 부정적인 이상과 함께 긍정적인  이상을 담고 있어야 한다. 이념은 무엇에 저항하는 것임과 동시에 무엇을 지향하는 것이어야 한다. 우리가 제시하는 긍정적 이상은 바로 “자연”, 다시 말해 야생의 자연이다. 여기서 야생의 자연이란, 인간의  관리에서 벗어나 있고 인간의 간섭 및 통제로부터 자유를 누리는 생물들과 지구가 조화롭게 기능을  수행하는 자연이다. 우리는 야생의 자연 안에 인간성을 포함한다. 우리가 말하는 인간성이란 조직 사회의  규제를 받지 않는, 그리고 우연, 자유 의지, 또는 (당신의 종교 혹은 철학적 견해에 따라)신의 창조물인,  개인으로서의 인간이 조화롭게 기능을 수행하는 것을 뜻한다.  


184. 몇가지 이유로 자연은 기술에 대항한 완벽한 대안적 이상이 될 수 있다. (체제의 힘이 미치지 않는  곳에 존재하는)자연은 (끝없이 체제의 힘을 확장하려 하는) 기술의 반대편에 있다. 대부분의 사람들은  자연이 아름답다는 데 동의할 것이다. 확실히 자연은 사람들에게 굉장한 호소력을 지니고 있다. 급진적  환경주의자들은 이미 자연을 찬양하고 기술에 반대 하는 이념을 확보하고 있다.\hyperlink{51}{$^{51}$} 자연을 위해 공상적  유토피아나 새로운 사회 질서를 건설할 필요가 없다. 자연은 스스로를 돌본다. 자연은 인간 사회가  시작되기 오래 전부터 이미 존재해 온 우연의 산물이었다. 그리고 헤아릴 수 없는 세월 수많은 인간  사회들은 자연에 큰 피해 입히지 않고 자연과 공존했다. 오로지 산업혁명의 결과로 인해 인간 사회는  자연에 참으로 막대한 영향력을 행사하게 된 것이다. 자연에 가해지는 억압을 종식시키기 위해 어떤 특별한 종류의 사회를 만들어 낼 필요는 없다. 그저 산업사회를 몰아내면 된다. 딩연한 이야기지만,  산업사회를 제거하는 것만으로 모든 문제를 해결할 수는 없다. 산업사회는 이미 자연에 엄청난 손상을  입혔고, 그 상처가 회복하는데 아주 오랜 시간이 걸릴 것이다. 한편 산업화 이전의 사회들도 자연에  심각한 손상을 입힐 수 있었다. 하지만 산업 사회를 제거하는 것만으로도 일단 상당한 성과를 거두게 될  것이다. 산업 사회가 사라지면, 자연에 가해지는 최악의 억압이 종식될 것이고, 그러면 상처가 회복되기  시작할 것이다. 조직 사회가 (인간성을 포함해)자연에 대한 통제를 강화하는 능력도 사라지게 될 것이다.  산업 사회가 사라진 후 어떤 사회가 생겨나든 대부분의 사람들은 자연과 밀접한 관계를 맺는 삶을 살아갈 것이다. 첨단 기술이 없는 세상에서 사람들이 살아갈 수 있는 길은 그것 뿐이기 때문이다. 먹고 살기  위해서는 농부가 되거나 목동, 어부, 아니면 사냥꾼이 되어야 할 것이다. 그리고 지역 공동체가 늘어날  것이 틀림없다. 첨단 기술과 고속 통신이 사라지면서, 정부를 비롯해 지역 공동체를 통제하는 거대  조직들의 능력도 제한될 것이기 때문이다. 


185. 산업 사회를 제거함으로써 빚어지는 부정적 결과들에 대해서는... 글쎄, 두 마리 토끼를 동시에 잡을 수는 없다. 하나를 얻으려면 다른 하나를 희생해야하는 법이다.  


186. 대부분의 사람들은 심리적 갈등을 싫어한다. 따라서 사람들은 어려운 사회적 이슈에 관해 골치  아프게 생각하는 것을 싫어하고, 다른 사람이 그 이슈에 대해 ‘이것은 무조건 좋고 저것은 무조건 나쁘다’ 식의 간단한 흑백 논리를 제시해 주는 것을 좋아한다. 그러니 두 가지 수준에서 혁명의 이념을 개발해야  한다.\hyperlink{52}{$^{52}$} 


187. 보다 정교한 수준의 이념은 지적이고 신중하며 이성적인 사람들에게 제시되어야 한다. 이 때  추구해야할 목표는 합리적이고 냉철한 판단 위에서 산업 체제에 저항할 핵심적인 혁명가들을 만들어 내는 것이다. 그 사람들은 체제의 문제와 모호한 성격에 대해, 그리고 체제를 제거하기 위해 어떤 대가를  치러야 하는지에 대해서도 철저히 이해해야 한다. 이런 유형의 사람들을 끌어들이는 일은 특히 중요하다.  그들은 유능한 사람들이며, 또한 다른 사람들에게 영향을 줄 수 있는 사람들이기 때문이다. 이런  사람들에게 이야기할 때는 최대한 이성적으로 이야기해야 한다. 고의로 사실을 왜곡하지도 말아야 하며,  선동적인 표현도 피해야 한다. 그렇다고 해서 감정적인 호소를 일체 외면하라는 뜻은 아니다. 다만 감정에 호소할 경우, 진실을 호도하지 않도록 주의하고, 이념이 지닌 지적 품위를 훼손할 만한 일은 저지르지  말라는 것이다.  


188. 두번째 수준의 이념은, 생각없이 살아가는 다수 대중이 기술과 자연의 갈등을 이해할 수 있도록  명료한 언어를 사용해 단순한 형식으로 전파되어야 한다. 하지만 이 두번째 수준에서도 이념을 표현할 때  비겁한 언어나 선동적이고 비이성적인 언어는 사용하지 말아야 한다. 그런 언어를 사용할 경우 신중하고  이성적인 사람들을 뒤돌아서게 만들 수도 있다. 비겁하고 선동적인 프로파간다가 단기적으로는 상당한  효과를 가져올 수도 있다. 하지만 생각없이 사는 변덕스러운 대중은 누군가 더 나은 프로파간다 미끼를  들고 나타나는 즉시 태도를 바꾸어 버린다. 그러니 그런 대중의 열정을 부추기기보다는, 역시 논리적  근거를 통해 헌신적인 소수 사람들의 충성심을 계속 유지하는 것이 장기적으로 볼 때 더 나은 결과를  낳는다. 하지만 체제가 거의 파괴 직전에 이르렀을 때, 그리고 낡은 세계관이 무너졌을 때, 새로운 지배적  세계관의 자리를 두고 경쟁하는 이념들 간의 마지막 투쟁이 벌어질 경우에는 선동적 프로파간다가 필요할 수도 있다.  


189. 혁명가들은 그런 마지막 투쟁이 필쳐지기 전에, 대다수의 사람들이 자기 편에 서주리라고  기대해서는 안된다. 역사를 만드는 것은 적극적이고 확고한 신념을 지닌 소수이지, 다수가 아니다. 다수는 대부분의 경우 자신이 정말로 무엇을 원하는지에 대해 명확하고 일관된 생각을 갖고 있지 않다. 혁명을  향한 마지막 진격의 날이 올 때까지,\hyperlink{53}{$^{53}$} 혁명가의 임무는 다수의 얄팍한 지지를 얻어내는 일이 아니라, 깊게 헌신적인 사람들로 이루어진 작은 핵심 집단을 만들어내는 일이다. 다수에 대해서는 새로운 이념이  존재함을 알리고 수시로 그것을 일깨워 주는 것으로 충분하다. 물론 헌신적인 소수의 집단을 약화시키지  않으면서, 다수의 지지를 얻어낼 수 있다면 그야말로 바람직한 일이다.  


190. 모든 사회적 갈등은 체제를 불안정하게 만드는 데 도움을 준다. 하지만 혁명가는 어떤 갈등을  조장해야 할지를 신중하게 고려해야 한다. 대다수 사람들과 권력을 소유한 산업 사회의 엘리트(정치가,  과학자, 상류층, 기업 임원, 정부 관료 등) 간에 확실한 갈등의 경계선을 그어야 한다. 절대로 혁명가와  대다수 사람들 사이에 경계선이 그어져서는 안된다. 예를 들어, 혁명가가 미국인의 과소비 행태를 비난  하는 것은 좋지 않은 전략이다. 그 대신에 미국인을, 쓸모없는 쓰레기를 자유의 상실이라는 비싼 대가를  치르며 구입하도록 강요당하는 존재로서, 즉 광고 및 마케팅 산업의 피해자로서 부각시켜야 한다. 어느  쪽이든 둘 다 사실이라는 점에서 다를 바가 없다. 대중을 세뇌한다는 죄목으로 광고 산업을 비난하든,  아니면 광고 산업이 자신을 세뇌하도록 허용한다는 죄목으로 대중을 비난하든, 그것은 다만 태도의  문제일 뿐이다. 전략의 문제를 놓고 볼 때, 대중을 비난하는 것은 피해야 한다.  


191. 권력을 소유한 (기술을 지배하는)엘리트와 (기술의 지배를 당하는)일반 대중 간의 갈등 외에 다른  사회적 갈등을 부추길 때에는 실행에 앞서 다시 한번 생각해야 한다. 우선, 여타의 갈등들은 정작 중요한 
갈등(엘리트와 보통 사람들 간의 갈등, 기술과 자연 간의 갈등)으로부터 사람들의 관심을 분산시킬 수  있다. 또한 여타의 갈등들은, 실제로는 기술 발전을 조장할 수도 있다. 그 갈등에서 갈등 당사자들은  상대방을 짖누르기 위해 기술의 힘을 사용하려 들기 때문이다. 국가 간의 경쟁에서 그런 현상은 확연히  드러난다. 또한 한 국가 안의 인종 간 갈등에서도 같은 현상이 나타난다. 예를 들어 미국에서 흑인  지도자들은 기술 권력 엘리트의 자리에 흑인 개인들을 보내는 방법으로 아프리카계 미국인을 위한 권력을 획득하려 몸부림치고 있다. 즉 더 많은 흑인 정부 관료, 흑인 과학자, 흑인 기업 임원이 생기기를 바라는  것이다. 그러나 그런 노력을 통해 흑인 지도자들은 오히려 기술 체제가 아프리카계 미국인의 하부 문화를  흡수하는 것을 도와주고 있다. 간단히 말해, 우리는 권력 엘리트 대(對) 보통 사람들, 기술 대 자연 간의  갈등 틀에 해당하는 사회적 갈등만을 부추겨야 한다.  


192. 소수자의 권리를 전투적으로 쟁취하려 할 경우 오히려 인종 간의 갈등을 증폭시키게 된다 (문단21,  29를 볼 것). 혁명가는 소수자 집단이 다소간 불이익을 겪더라도 그 불이익은 그다지 중요하지 않다는  것을 강조해야 한다. 우리의 진정한 적은 산업-기술 체제이며, 체제에 항거하는 투쟁에서 인종 간의  구분은 아무런 의미가 없다.  


193. 우리가 생각하는 혁명이 반드시 정부에 대한 무장봉기를 포함하는 것은 아니다. 물리적 폭력은  포함될 수도 있고 배제될 수도 있다. 단, 우리의 혁명은 절대 정치적 혁명이 되어선 안된다. 정치가  아니라, 기술과 경제에 혁명의 초점을 맞춰야 한다.  


194. 혁명가는 산업 체제가 심각한 스트레스를 겪고 대다수 사람들이 산업 체제를 명백한 실패로  인정하기 전까지는, 합법적이건 불법적이건, 일체의 정치적 권력을 장악하는 일은 피해야 한다. 가령 녹색당이 선거를 통해 미국 의회를 장악하게 되었다고 가정해 보자. 자신들의 이념을 쇄신하거나  희석시키지 않으려면, 녹색당은 어쩔 수 없이 강경 수단을 동원해 경제 성장을 경제 축소로 전환시켜야 할 것이다. 보통 사람들에게 그 결과는 참혹할 것이다. 대량 실업, 생필품 부족 등의 사태가 벌어진다. 설령 초인적인 경영 능력을 발휘해 최악의 사태를 피한다 해도, 여전히 사람들은 그 때까지 중독되어 있던 안락함을 포기하는 데 어려움을 겪을 것이다. 불만은 늘어나고 녹색당은 선거에 패배해 요직에서 쫓겨날  것이며, 혁명가들은 좌절에 고통받을 것이다. 그러므로 혁명가들은, 산업 체제 자체가 완전히 혼란에 빠져 사람들이 그 모든 고통은 혁명가의 정책 때문이 아니라 산업 체제 자체의 결함에서 비롯된 것이라고  생각할 때까지는, 정치 권력을 장악하려 애쓰지 말아야 한다. 기술에 항거하는 혁명의 주체는  아웃사이더들이어야 하며, 위로부터의 혁명이 아니라 아래로부터의 혁명이어야한다.  


195. 혁명은 국제적으로, 또 범세계적으로 일어나야 한다. 혁명은 결코 국가별로는 수행될 수 없다.  미국인들에게 기술 발전이나 경제 성장을 멈춰야 한다고 주장하면, 사람들은 곧바로 히스테리에 빠져,  우리가 기술에서 뒤쳐지면 일본인들이 우리를 추월할 것이라고 비명을 지를 것이다. 저 빌어먹을 로봇! 일본인들이 우리보다 더 많은 자동차를 팔아먹는다면 세계는 거꾸로 굴러가게 될 것이다! (국가주의는  기술의 든든한 후원자다.) 보다 합리적인 반론으로는, 만약 비교적 민주적인 국가들이 기술에서  뒤쳐지면, 중국이나 베트남, 북한 같은 못된 독재 국가들이 발전을 계속할 것이고 마침내 그들이 세계를  지배할 것이라는 반론이 제기될 수도 있다. 국가주의의 발흥을 막기 위해서라도, 산업체제에 대한 공격은  가능하다면 모든 국가에서 동시에 이루어져야 한다. 물론 산업 체제가 전세계에 걸쳐 거의 동시에  무너진다는 보장은 없다. 그리고 체제를 몰아내려는 시도가 오히려 독재 체제를 낳게 할 가능성도 있다.  하지만 그런 위험은 감수해야 한다. 위험을 감수할 만한 가치도 충분하다. 민주적인 산업 체제와 독재자가 통제하는 산업 체제 간의 차이는 산업 체제와 비산업 체제 간의 차이에 비하면 사소한 차이이기 때문이다.\hyperlink{54}{$^{54}$} 차라리 독재자들이 통제하는 산업 체제가 훨씬 더 낫다고 볼 수도 있다. 독재자가 통제하는 산업 체제는 대개 비효율적이며. 따라서 붕괴할 가능성도 더 높기 때문이다. 쿠바를 보라. 


196. 혁명가는 세계 경제를 하나의 단일체로 묶는 시도들을 잘 이용해야 한다. NAFTA나 GATT 같은  자유무역 협정들은 아마 단기적으로는 환경에 해악을 끼치겠지만, 장기적으로는 국가들 사이의 경제적  종속성을 증대시킴으로써 오히려 혁명에 도옴이 될 것이다. 만약 세계 경제가 완전히 통합되어 어느 한  중요 국가의 붕괴가 모든 산업 국가의 붕괴로 이어질 수 있다면, 전 세계 차원에서 산업 체제를  무너뜨리는 일도 더 쉬워질 것이다. 


197. 어떤 사람들은 현대인이 자연에 대해 과도한 권력과 통제력을 갖고 있다고 주장한다. 그러면서도  그들은 인류에 대해서는 훨씬 더 소극적인 입장을 취한다. 그들은 거대 조직을 위한 권력과, 개인 및 작은  집단을 위한 권력을 구분하지 못하기 때문에, 그들이 내세우는 이론은 항상 모호하다. 무력함과 수동성을  옹호하는 것은 분명 잘못이다. 사람들에겐 권력이 필요하기 때문이다. 집단적 존재로서의 현대인, 즉 산업 체제는 자연에 대해 엄청난 권력을 갖고 있으며, 우리 FC는 그 권력을 사악한 것이라고 간주한다. 그러나  현대의 개인들과 개인들의 작은 집단은 원시인보다도 훨씬 적은 권력을 갖고 있다. “현대인”이 자연에  대해 갖고 있는 어마어마한 권력을 행사하는 것은 개인들이나 작은 집단이 아니라, 거대 조직들이다.  현대의 보통 개인이 어쩌다 기술 권력을 쥐게 된다 해도, 그의 권력은 아주 제한된 한계 안에서, 그것도  체제의 감독과 통제를 받으면서 허용될 뿐이다. (당신의 모든 일에 면허가 필요하며, 그 면허들에는  규칙과 규제가 따른다.) 개인이 지닌 기술의 권력은 고작해야 체제가 그에게 골라서 건네준 권력에  불과하다. 그가 자연에 대해 지닌 개인적 권력은 아주 미미한 것이다.  


198. 원시의 개인들과 작은 집단들은 실제로는 자연에 대해 상당한 권력을 갖고 있었다. 아니, 차라리  자연 안에서 권력을 갖고 있었다고 하는 편이 나을 것이다. 원시인에게 식량이 필요할 때, 그는 어떻게  식용식물 뿌리를 찾고 요리할 것인지를 알고 있었고, 직접 만든 무기로, 어떻게 짐승을 추적해 사냥할지  알고 있었다. 더위와 추위, 비, 홍수로부터 어떻게 자신을 보호하는지를 알고 있었다. 그럼에도 불구하고  원시인이 자연에 입힌 손상은 상대적으로 적었다. 원시 사회의 집단적 권력이 산업 사회의 집단적 권력에  비하면 무시해도 좋을 만큼 미약했기 때문이다.  


199. 무력함과 수동성을 옹호하는 대신에, 우리는 산업 체제의 권력이 파괴되어야 하며, 그렇게 되면  개인들과 작은 집단들의 권력과 자유가 비약적으로 증대되리라는 것을 주장해야 한다. 


200. 산업 체제가 완벽하게 쓰러질 때까지, 혁명가의 유일한 목표는 체제 파괴 뿐이다. 다른 목표들은  중심 목표로부터 관심과 에너지를 분산시킬 것이다. 더욱 중요한 사실은, 만약 혁명가가 기술 파괴 이외의 다른 목표를 갖게 되면, 그 다른 목표를 달성하기 위한 수단으로서 기술을 사용하고 싶은 유혹을 받을  수도 있다는 것이다. 혁명가가 그런 유혹에 굴복할 경우, 곧바로 기술의 함정에 빠지게 된다. 현대의  기술은 하나의 통합된, 치밀하게 조직된 체제로서 어떤 기술을 유지하기 위해선 거의 대부분의 기술을  유지해야만 하고, 따라서 아주 작은 부분을 제외하곤, 기술을 고스란히 살려 둘 수밖에 없는 것이다.  


201. 예를 들어, 혁명가가 “사회 정의”를 목표로 두었다고 가정해 보자. 인간의 본성으로 인해, 사회  정의는 자발적으로 이루어지지 않는다. 강요되어야 하는 것이다. 사회 정의를 강요하기 위해서 혁명가는  중앙 조직과 통제권을 유지해야 한다. 그러려면 고속 장거리 수송 수단과 통신망이 필요하고, 결국 그  수송 및 통신 시스템을 지탱하기 위한 기술 전체가 필요하다. 매사가 그런 식이다. 따라서 사회 정의를  확립하려면 혁명가는 어쩔 수 없이 기술 체제의 거의 전부를 유지해야 하지만, 동시에 기술 체제를  제거하기 위한 노력을 방해해서는 안된다는 모순이 벌어진다. 


202. 혁명가가 일체의 현대 기술 없이 체제를 공격하는 것은 불가능하다. 다른 모든 것을 포기한다 해도,  메시지를 전달하려면 최소한 통신 기술은 이용해야 한다. 그러나 혁명가가 현대 기술을 사용할 경우,  그것은 오직 하나의 목표, 즉 기술 체제를 공격하기 위한 것이어야 한다.  


203. 알코올 중독자가 포도주 한 통 옆에 앉아 있다고 상상해 보라. 그가 혼자 이렇게 중얼거린다고  생각해 보자. “적당히 마신다면 포도주가 나쁠게 뭐가 있나. 약간의 포도주는 건강에 좋다고들 하잖아! 한 모금만 마시면 괜찮을 거야..." 당신은 어떤 일이 벌어질지 잘 알고 있다. 기술을 가진 인류가 포도주 한  통을 가진 알코올 중독자와 똑같다는 사실을 결코 잊지 말라.  


204. 혁명가들은 최대한 많은 자녀를 낳아야 한다.\hyperlink{55}{$^{55}$} 사회적 태도들은 상당 수준까지 유전된다는 강력한  과학적 증거가 나와 있다. 어떤 사람의 사회적 태도가 유전인자의 구성에 의한 직접적 결과라고는 아무도  주장하지 않는다. 하지만 우리 사회의 상황에서는 개인의 성격적 특성은 그 사람이 어떤 특정한 사회적  태도를 갖도록 만들고 있다. 이런 사실에 대한 반론도 제기되어 있지만. 그 반론들은 근거가 빈약하며, 
대개 이념적 동기에 의해 만들어진 것들이다. 어느 경우에든 평균적으로 아이들이 부모와 비슷한 사회적  태도를 갖게 된다는 것은 부인할 수 없는 사실이다. 우리의 관점에서 볼 때, 사회적 태도를 유전적 요인이  만드느냐, 아니면 양육환경이 만드느냐 하는 질문은 별 의미가 없다. 어느 경우든, 사회적 태도는  만들어지는 것이다. 


205. 문제는, 산업 체제에 저항할 가능성이 있는 사람들 중의 다수가 인구 문제에 대해서도 역시 관심을  갖고 있으며, 그에 따라 적은 수의 아이를 낳거나 아예 낳지 않는다는 점이다. 그것은 산업 체제를  지지하거나, 최소한 용인하는 사람들에게 미래의 세계를 통째로 넘겨주는 행위와 다를 바 없다. 다음 세대의 혁명 세력을 강화하기 위해서는 현재의 혁명 세대가 스스로 열심히 번식해야 한다. 그런다고 해서  인구 문제가 크게 악화되지도 않는다. 가장 중요한 문제는 산업 체제를 제거하는 일이다. 일단 산업  체제가 사라지면 세계 인구는 필연적으로 감소한다(문단 167을 볼 것). 반면에 산업 체제가 살아남게  되면, 체제는 식량 생산을 증가시키는 새로운 기술을 계속 개발할 것이고 인구는 무한정 늘어나게 될  것이다.  


206. 혁명 전략과 관련해 우리가 확고하게 지켜야 할 사항은 간단하다. 현대 기술의 제거가 유일한  목표가 되어야 하며, 그 목표와 경쟁하는 다른 어떤 목표도 허용해서는 안 된다는 것이다. 그 나머지에  대해 혁명가는 실증적인 접근 방법을 택해야 한다. 만약 경험에 의해 앞의 문단들에서 추천한 전략들이  전혀 좋은 결과를 내지 못한다는 것이 밝혀지면, 그 때는 미련없이 포기해야 한다.  


\section*{두 종류의 기술} 
207. 우리가 제안한 혁명에 대해, 역사적으로 기술은 항상 진보해 왔고 단 한번도 후퇴하지 않았으며,  따라서 기술의 후퇴는 불가능하기 때문에 혁명도 실패할 수밖에 없다는 주장을 제기하는 사람도 있을  것이다. 하지만 그런 주장은 틀린 것이다.  


208. 우리는 기술을 두 가지로 구분한다. 하나는 소규모 기술이며, 다른 하나는 조직 의존형 기술이다.  소규모 기술은 소규모 공동체가 외부의 도움 없이 사용할 수 있는 기술을 말한다. 조직 의존형 기술은  대규모의 사회 조직에 의존하는 기술이다. 우리는 소규모 기술이 후퇴한 적은 한 번도 없었음을 잘 알고  있다.\hyperlink{56}{$^{56}$} 하지만 조직 의존형 기술은 그것이 의존하고 있는 사회 조직이 붕괴할 때 함께 후퇴한다. 로마  제국이 멸망할 때도 로마의 소규모 기술은 살아남았다. 영리한 시골 장인(匠人)이라면 누구든 물레방아를 만들 수 있었고, 숙련된 대장장이라면 로마식 제련법에 의해 얼마든지 철을 만들 수 있었다. 하지만  로마의 조직 의존형 기술은 후퇴했다. 로마의 대수로는 무너졌고 다시는 재건되지 않았다. 로마의 도로  건설 기술은 잊혀졌다. 로마의 하수도 시스템은 잊혀졌으며 최근까지도 유럽 도시들의 하수도는 로마  제국이 아니라 고대 로마의 시스템에 머물러 있었다.\hyperlink{57}{$^{57}$} 


209. 기술이 항상 진보하는 것처럼 보이는 이유는 산업 혁명 이전 1,2세기 전까지만 해도 대부분의  기술이 소규모 기술이였기 때문이다. 하지만 산업 혁명 이후 개발된 기술은 대부분 조직 의존형 기술이다. 냉장고를 예로 들어보자. 공장에서 생산된 부품이나 최첨단 공구들 없이 몇몇 지역 장인(匠人)들이  모여서 냉장고 한 대를 만드는 것은 실질적으로 불가능하다. 기적적으로 그들이 냉장고를 만들어 낸다고  해도 안정적인 전기 공급 없이는 무용지믈이다. 그러니 강물을 막아 댐을 만들고 발전기를 세워야 한다.  발전기에는 엄청난 구리선이 필요하다. 현대적 공작 기계 없이 구리선을 만드는 일을 상상해 보라. 이번엔 냉동을 위한 냉매(冷媒) 가스를 어디서 구할 것인가? 차라리 냉장고가 발명되기 전에 그랬던 것처럼 얼음 창고를 만들거나 음식을 말리고 절여서 보관하는 편이 훨씬 더 쉬울 것이다.  


210. 그러니 일단 산업 체제가 완전히 붕괴되면, 냉장고 기술은 곧 사라져 버릴 것이 틀림없다. 다른 조직 의존형 기술에도 똑같은 일이 벌어진다. 일단 이 기술이 한 세대 동안 잊혀지면, 그것을 다시 세우는 데는  그것을 처음 세우는 데 수백 년이 걸린 것처럼 역시 수백 년이 걸릴 것이다. 기술 서적은 거의 남아 있지  않을 것이며, 남아 있다고 해도 사방에 흩어져 있을 것이다. 만약 그 잔해 속에서 외부의 도움 없이 어떤 산업 사회가 다시 건설되려면, 다음과 같은 일련의 단계를 밟아야 할 것이다. 도구를 만들기 위해 도구가  필요하고, 그 도구를 만들기 위해 도구가 필요하고 그 도구를 만들기 위해... 길고 긴 경제 개발 과정과 
사회 조직의 발전이 필요하다. 그리고 그 때에는 이미 기술에 저항하는 이념이 지배 이념이 되어 있을  것이며, 설령 그런 이념이 없다 해도, 산업 사회를 재건설하는데 흥미를 느낄 사람은 전혀 없으리라는  점에 대해서는 믿어도 종다. “진보”에 대한 열광은 현대 사회에 국한된 현상일 뿐이며, l7세기 이전까지  그런 현상은 존재하지도 않았다.\hyperlink{58}{$^{58}$}


211. 중세 후기 거의 비슷한 수준으로 진보한 네 개의 중심 문명이 있었다. 유럽, 이슬람, 인도, 그리고  극동(중국, 일본, 한국) 문명이다. 이들 문명 중 세 문명은 어느 정도 안정적인 상태로 머물렀다. 오직 유럽만이 역동적으로 움직이기 시작했다. 유럽이 그 때 왜 그렇게 역동적으로 변했는지는 아무도 모른다.  역사가들은 이런저런 이론을 내세우지만, 공허한 이론에 불과하다. 어쨌든, 분명한 사실은 기술 사회로의  급속한 발전은 한정된 특수 조건 아래서만 일어난다는 것이다. 따라서 장기간에 걸친 기술의 후퇴가 결코 일어나지 않으리라고 믿어야 할 이유는 없다.  


212. 결국에는 산업-기술 형태의 사회가 다시 발전할 것인가? 그럴지도 모른다. 하지만 그것을 걱정할  필요는 없다. 우리가 500년 혹은 l,000년 뒤에 벌어질 일을 예측하고 통제할 수는 없으니 말이다. 그  문제는 미래에 살아갈 사람들이 해결할 문제다.


\section*{좌파의 위험성}


213. 반항에의 욕구와 운동에 소속되고 싶은 욕구 때문에, 좌파나 그와 비슷한 심리적 유형의 사람들은  비좌파적 저항 운동에도 이끌린다. 따라서 운동 내부에 좌파 성향의 사람들이 늘어나고, 비좌파 운동은  좌파 운동으로 변질된다. 그러면서 운동의 기존 목표를 좌파적 목표로 대체해 버리거나, 왜곡시켜 버린다. 


214. 자연을 찬양하고 기술에 항거하는 운동이 그런 사태를 피하기 위해서는, 단호한 반(反)좌파적  입장을 고수해야 하며, 좌파와의 협력을 일체 배제해야 한다. 장기적으로 볼 때, 좌파는 야생의 자연,  인간의 자유, 그리고 현대 기술의 제거와는 공존할 수 없다. 좌파는 집단주의적이다. 좌파는 전 세계( 자연과 인류 모두를) 하나의 통합체로 묶으려 한다. 이는 조직 사회를 통해 자연과 인간의 삶을  관리한다는 것을 의미하며, 그러기 위해선 첨단 기술이 필요하다. 고속 이동 수단과 통신망 없이는 통합된 세계를 만들 수 없다. 정교한 심리적 기술 없이는 모든 사람들이 서로를 사랑하게 만들 수 없다. 기술적  토대가 없으면 “계획적 사회”를 만들 수 없다. 무엇보다도 좌파를 끌고 가는 동력은 바로 권력에 대한  욕구이다. 그리고 좌파들은 대규모 운동, 대규모 조직과의 동일화를 통해 집단주의적 권력을 추구한다.  좌파는 결코 기술을 포기하지 않을 것이다. 집단주의적 권력의 원천으로서 기술은 너무나 소중하기  때문이다.  


215. 아니키스트 역시 권력을 추구한다.\hyperlink{59}{$^{59}$} 다만 개인들 또는 작은 집단들의 권력을 추구할 뿐이다. 그는  개인들과 작은 집단들이 스스로의 삶을 둘러싼 환경을 통제할 수 있게 되기를 바란다. 그가 기술에  저항하는 이유는, 기술로 인해 작은 집단들이 거대 조직에 종속당하기 때문이다.  


216. 일부 좌파는 기술에 저항하는 것처럼 보일 수도 있다. 그러나 그들이 기술에 저항하는 것은 그들이  아웃사이더일 경우에 한해서이며, 기술 체제가 비좌파에 의해 통제되는 경우에 한해서이다. 만약 좌파가  사회의 주도권을 장악하고, 그래서 좌파가 기술 체제를 언제든 쓸 수 있는 도구로 만든다면, 그들은 그  때부터 열광적으로 기술을 이용하고 기술의 발전을 지원할 것이다. 좌파가 역사에서 끝없이 반복해 왔던 그 패턴 그대로의 행동이다. 러시아 볼셰비키가 아웃사이더였을 때는 검열과 비밀 경찰에 대해 격렬히  저항했고, 소수 민족의 자치권을 외쳤다. 그러나 자신들에게 권력이 넘어오자마자 볼셰비키는 더 철저한  검열을 실시했고 차르 치하에서의 비밀 경찰보다 훨씬 잔인한 비밀 경찰을 창설했다. 그리고 소수 민족에  대한 억압도 차르 시대보다 더하면 더했지, 덜하지는 않았다. 미국의 경우, 몇십 년 전 대학에서 좌파가  소수였을 때, 좌파 교수들은 열렬히 학문의 자유를 주장했다. 그러나 오늘날, 좌파가 주도권을 쥔 대학들에서 좌파들은 나머지 모든 사람으로부터 학문의 자유를 빼앗고 있다.(이것이 바로 “정치적 올바름” 운동이다.) 똑같은 일이 좌파와 기술 사이에도 벌어질 것이다. 일단 기술을 자기 통제 하에 넣고 나면,  좌파는 기술을 이용해 나머지 모든 사람을 억압할 것이다. 


217. 과거의 혁명에서, 권력에 눈이 먼 좌파들은 처음에는 진보적 성향의 좌파는 물론 비좌파  혁명가들에게도 협력했다. 그 후 권력을 독점하기 위해 양쪽을 모두 배신했다. 프랑스 혁명에서는  로베스피에르가 그랬고, 러시아 혁명에서는 볼셰비키가 그랬다. 1938년 스페인 내전에서는  공산주의자들이, 쿠바에선 카스트로와 그 일당이 그랬다. 좌파의 지난 역사를 비추어 보았을 때, 오늘날의 비좌파 혁명가들이 좌파와 협력하는 것은 참으로 멍청한 짓이다.


218. 수많은 사상가들이 좌익 이념은 일종의 종교임을 지적해 왔다. 물른 좌익 이념은 일체의 초자연적  존재를 인정하지 않으므로 엄밀한 의미에서의 종교는 아니다. 하지만 좌파에게 있어 좌익 이념은 마치  종교가 사람들에게 하는 것과 거의 동일한 심리적 역할을 수행한다. 좌파는 좌익 이념을 믿어야"만" 한다.  좌익 이념은 좌파의 심리적 경제에서 핵심적인 역할을 수행한다. 그의 신념은 논리나 사실에 의해 쉽게  수정되지 않는다. 그는 좌익 이념이 윤리적으로 “옳다(Right)”는 확고한 신념과, 자신에겐 좌파 윤리를  모든 사람에게 강요할 권리 뿐만 아니라 의무도 있다는 확고한 신념을 가지고 있다. (하지만, 우리가  좌파로 부르는 사람들 중 상당수는 자신을 좌파라고 생각하지 않으며, 자신들의 신념 체계를 좌익 이념이라고 부르지도 않을 것이다. 우리가 “좌익 이념”이라는 용어를 사용하는 것은, 페미니스트들이나  동성애자 권리 운동, 정치적 올바름 운동 등 관련된 운동 부류들을 모두 지칭할만한 더 적합한 용어를  찾을 수 없었기 때문이며, 또한 이들 운동이 과거의 좌파 운동과 강력한 친화력을 지니고 있기 때문이다.  문단 227${\sim}$230을 볼 것.) 


219. 좌익 이념은 전체주의적인 힘이다. 좌익 이념이 일단 권력을 장악하면, 곧바로 모든 사적 영역을  침범해 들어가며 모든 사상을 좌파의 틀에 맞춰 뜯어고친다. 이것은 부분적으로 좌익 이념이 지닌 사이비  종교적 성격 때문이다. 좌익 이념에 어긋나는 모든 것은 “죄악”이다. 더 중요한 사실은, 좌익 이념이  전체주의적 세력이 되는 것은 좌파들의 권력에의 욕망 때문이라는 점이다. 좌파는 사회 운동과의  동일화를 통해 권력욕을 충족시키려 하며, 운동의 목표를 추구하고 달성하도록 협력함으로써 권력  과정을 통과하려 한다 (문단 83을 볼 것). 하지만 운동이 그 목표를 달성하는 경우에도 좌파는 결코 만족하지 않는다. 그의 운동이 대리 활동이기 때문이다(문단4l을 볼 것). 즉 좌파의 진정한 동기는 좌익 이념의 표면적 목표를 달성하는 것이 아니다. 그의 진정한 동기는 사회적 목표를 위해 투쟁하고 목표에  접근하는 과정에서 그가 얻게 되는 권력의 느낌인 것이다.\hyperlink{60}{$^{60}$} 그렇기 때문에 좌파는 자신이 이미 달성한  목표로서는 결코 만족할 수 없다. 권력 과정에 대한 욕구로 인해 좌파는 항상 새로운 목표를 추구할  수밖에 없다. 좌파는 소수 민족에게 동등한 기회가 주어지기를 원한다. 일단 그 목표가 이루어지면 이번엔 소수 민족의 통계적 평등을 이룩해 줄 것을 요구한다. 그리고 어떤 사람이 여전히 소수 민족에 부정적  태도를 보이고 있을 경우에는 그 사람을 재교육시켜야 한다. 소수 민족만으로는 충분치 않다. 누구도  동성애자와 장애인, 뚱뚱한 사람, 노인, 못생긴 사람들에 대해 부정적인 태도를 보여서는 안 된다. 대상은  끝없이 이어진다. 대중에게 흡연의 위험성을 알려주는 것만으로는 충분치 않다. 담뱃갑마다 경고문을  찍어야 한다. 이어서 담배 광고를 금지하지는 못해도, 제한해야 한다. 담배가 불법이 되기 전에는  운동가에게 만족은 결코 있을 수 없고, 담배가 불법이 되고 나면 이번엔 술이, 다음 번엔 인스턴트 음식이  표적이 된다. 운동가들은 아동학대가 광범위하게 자행되고 있음을 세상에 알렸다. 좋은 일이다. 하지만  이제 그들은 엉덩이 때리기까지 금지하고 싶어한다. 엉덩이 때리기를 금지하고 나면, 그 때는 또 그들이  보기에 건전하지 않은 무언가를 금지하려 들 것이다. 금지 대상품목은 끝없이 이어진다. 어린이 양육법에  관한 모든 것을 통제하기 전까지는 그들은 결코 만족하지 않는다. 그리고 그 다음에는 또 다른 사회  문제로 넘어갈 것이다.  


220. 좌파에게 사회 문제점들의 목록을 작성하게 한다고 가정해 보자. 그리고 그들이 요구하는 대로 모든 사회적 변화를 이루었다고 가정해 보자. 틀림없이 몇 년 안에 대부분의 좌파는 새로운 불만, 교정해야 할  새로운 사회”악”을 찾아낼 것이다. 다시 말하지만, 좌파를 움직이는 동기는 사회의 병패를 해소하는 것이  아니라, 자신이 찾아낸 해결책을 사회에 강요함으로써 자신의 권력욕을 충족시키는 것이기 때문이다. 


221. 좌파의 생각과 행동은 그들이 고도로 사회화되어 있는 까닭에 제한될 수밖에 없다. 그렇기 때문에  과잉 사회화된 좌파는 다른 사람들이 하는 방식으로 권력을 추구할 수 없다. 그들의 권력욕을 충족시키는  길은 하나뿐이다. 그 길은 바로 자신의 윤리관을 모든 사람에게 강제로 주입하기 위한 투쟁이다. 


222. 좌파, 특히 과잉 사회화된 부류들은 에릭 호퍼(Eric Hoffer)의 책 \textlangle{}참된 신자(The True Believer)\textrangle{}에서 말하는 바로 그 의미에서의 “참된 신자”들이다. 그러나 모든 참된 신자들이 좌파와 동일한 심리적  특성을 갖고 있는 것은 아니다. 가령 참된 나치 신자는 참된 좌파 신자와는 확실히 다른 심리적 특성을  갖고 있을 것이다.\hyperlink{61}{$^{61}$} 그들은 하나의 사회 문제에 대해 온 마음을 다해 전념할 수 있는 능력을 지니고  있으므로, 참된 신자들은 혁명 운동에 유용한 구성원이 될 수 있으며, 어쩌면 필수적인 구성원일지도  모른다. 이때 제기되는 문제는, 우리가 다룰 수 없는 사람들도 혁명에 받아들여야 한다는 것이다. 우리는  기술에 항거하는 혁명에 참여하는 이들 참된 신자들의 에너지에 어떻게 고삐를 채울 수 있을지 전혀 확신이 서질 않는다. 현 단계에서 우리가 이야기할 수 있는 것은, 참된 신자의 열정을 오로지 기술  파괴에만 국한시킬 수 없을 경우, 참된 신자를 혁명에 끌어들이는 일은 대단히 위험한 일이라는 것이다.  만약 그가 또 다른 이상에 몰두하게 된다면, 그는 그 이상을 추구하는 수단으로서 기술을 이용하려 할  것이다. (문단 220, 221을 볼 것) 


223. 어떤 독자들은 이렇게 말할지도 모르겠다. "좌파 이념에 대해 당신들이 지껄인 이야기는 전부 헛소리다. 내가 아는 철수와 영희는 모두 좌파지만, 당신들이 말하는 전체주의적 성향은 갖고 있지도  않다." 절대 다수의 좌파들이 선량한 사람들로서, 타인의 가치관을 인정해야 한다는 진지한 믿음을  가지고 있으며, 자신들의 사회적 목표를 달성하기 위해 강압적인 수단을 사용하기를 원치 않는다는 것은  맞는 말이다. 우리가 좌익 이념에 대해 내린 평가는 모든 좌파 개인들에게 해당하는 것이 아니라,  운동으로서의 좌익 이념이 지닌 일반적 성격에 해당하는 것이다. 그리고 어느 운동의 성격이 반드시 그  운동에 참여하는 다수 사람들에 의해 결정되는 것은 아니다.  


224. 좌파 운동 내부에서 권력자의 지위에 오르는 사람들은 대개 가장 심하게 권력에 눈이 먼 유형의  좌파들이다.\hyperlink{62}{$^{62}$} 권력에 눈이 먼 사람들만이 권력자의 위치에 오르기 위해 발버둥치기 때문이다. 일단 권력에 눈먼 자들이 운동의 통제권을 장악하면, 보다 얌전한 다수의 좌파들은 지도자들의 행동에 대해  내심으로는 불만을 품게 되지만, 앞장 서서 지도자들에게 저항하지는 못한다. 얌전한 좌파들은 운동에  대한 믿음이 필요하고, 따라서 그 믿음을 포기할 수 없기 때문에 지도자들을 따라가는 것이다. 그렇다.  어떤 좌파들은 대담하게 나서 전체주의적 성향에 대해 저항한다. 하지만 대부분의 경우, 그들은 패배할  수밖에 없다. 권력에 눈먼 좌파들이 훨씬 더 조직적이고 훨씬 더 잔인하며, 마키아벨리적이고, 그 때까지  이미 확고한 권력의 토대를 쌓아 놓고 있기 때문이다.  


225. 그런 현상이 러시아와 그 밖의 좌파가 장악한 국가들에서 나타났다. 그와 비슷한 현상으로,  소비에트 연방에서 공산주의가 붕괴하기 전, 서구의 좌파들은 소비에트 연방을 거의 비난하지 않았다.  가끔씩 어쩔 수 없이 소비에트 연방이 수많은 잘못을 저질렀다고 인정할 때도 있었지만, 그들은 곧바로  공산주의를 위한 변명거리를 찾아내려 애썼고, 다시 서방의 잘못에 대한 이야기로 넘어갔다. 그들은  언제나 공산주의의 공격에 대항한 서방의 군사적 행동을 반대했다. 전 세계의 좌파들이 베트남에서의  미국의 군사적 행동에 대해 격렬히 저항했다. 하지만 소비에트 연방이 아프가니스탄을 침공했을 때,  그들은 아무 것도 하지 않았다. 그들이 소련의 행동을 인정해서가 아니었다. 좌파로서의 믿음을 버릴 수  없었기에, 그들은 차마 공산주의에 대해 저항할 수가 없었다. 오늘날, “정치적으로 올바른” 사람들이  지배하는 우리의 대학에 있는 좌파들 중 다수가 개인적으로는 학문의 자유를 제한해서는 안된다고 생각할 것이다. 하지만 그들은 학문의 자유가 제한된 상황에서 그럭저럭 살아간다. 


226. 따라서 대부분의 좌파들이 개인으로서는 선량하고 관대한 사람들이라고 해서, 좌익 이념 전반이  전체주의적 성향을 지니게 되는 경향을 막아주지는 못한다.  


227. 우리의 좌익 이념 논의는 한 가지 심각한 취약점을 갖고 있다. 우리가 사용하는 “좌파”라는 단어가  도대체 무엇을 뜻하는지가 여전히 모호하다는 점이다. 이 점을 두고 우리가 더 이상 할 수 있는 일은 없는  것 같다. 오늘날 좌익 이념은 수많은 운동으로 분열되어 있다. 물론 모든 운동이 좌파 운동은 아니다.  그리고 (급진적 환경주의 등)몇몇 운동들은 좌파들 뿐만 아니라, 철저한 비좌파들을 포함하고 있다.  바라건대 이들 비좌파들은 좌파와 협력하기에 앞서 좌파에 대해 좀 더 자세히 알아 두어야할 것이다.  다양한 유형의 좌파들이 다양한 유형의 비좌파들 속으로 숨어 들어간다. 그러니 어떤 개인이 좌파인지,  아닌지를 판단하기가 매우 어려운 경우도 흔하다. 좌익 이념에 대한 우리의 개념은 일단 현재까지는 이 
선언문에서 언급된 규정을 따른다. 우리가 독자들에게 줄 수 있는 충고는, 누가 좌파인지를 결정할 때,  자신의 판단을 따르라는 것 뿐이다.  


228. 하지만 좌익 이념을 가려내기 위한 분류 기준이 있다면 도움이 될지도 모르겠다. 단, 이 분류 기준을 곧이 곧대로 적용해서는 안 된다. 어떤 개인들은 좌파가 아닌데도 불구하고 일부 기준에 맞아떨어질 수도  있고, 이 분류 기준에 전혀 맞지 않는 좌파도 있을 것이다. 다시 말하지만, 판단은 당신 스스로 해야 한다. 


229. 좌파는 대규모 집단주의를 지향한다. 그는 오직 사회를 위한 개인의 의무와, 개인을 보호할 사회의  의무만을 강조한다. 그는 개인주의에 대해 부정적인 태도를 보인다. 그는 때로 도덕주의자의 언어를  사용한다. 그는 총기 규제와 성교육, 그 밖의 심리적 “계몽” 교육, 적극적 우대조치, 다문화주의를  지지한다. 그는 스스로를 피해자들과 동일시 한다. 그는 경쟁과 폭력에 반대하면서도, 좌파들의 폭력을  변호한다. 그는 “인종주의”, “성 차별”, “동성애 혐오”, “자본주의”, “제국주의”, “신(新)식민주의”, “인종 청소”, “사회 변혁”, “사회 정의”, “사회적 책임” 같은 좌파적 표현을 즐겨 사용한다. 좌파를 알아보는 가장  좋은 방법은, 그가 페미니즘이나 동성애자 권리, 소수민족 권리, 장애인 권리, 동물 권리, 정치적 올바름에 공감하고 있는지 알아보는 것이다. 이 모든 운동에 대해 강하게 공감하는 사람은 거의 확실히 좌파라고  보면 된다.\hyperlink{63}{$^{63}$} 


230. 좌파 중에서도 권력에 눈먼 부류들은 더 위험하다. 그들은 흔히 거만하고, 이념에 대해 교조적으로  접근한다는 특징을 갖고 있다. 하지만 가장 위험한 좌파는 과잉 사회화 된 부류 중 일부이다. 그들은  자신의 좌익 이념을 공격적으로 반복해서 떠들어댐으로써 사람들을 짜증나게 만들지는 않는다. 그들은  조용히 사람들의 비위를 건드리지 않으면서, 어린이들을 사회화시키고, 개인을 체제에 종속시키는  집단주의적 가치관과 “계몽”적 심리 기법을 확산시킨다. 이들 (우리가 칭하기를)은신 좌파들은 특정  부르주아 유형과 흡사하다. 단, 이들 사이의 유사성은 실천적 행동이 관련된 한에서 그치며, 이들은 심리,  이념, 그리고 동기의 측면에서 서로 확연히 다르다. 보통 부르주아들은 자신의 생활 양식을 지키기 위해서 사람들을 체제의 통제 하에 두려고 하거나, 아니면 단순히 전통적인 태도를 지니고 있기 때문에 그런  행동을 한다. 은신 좌파가 사람들을 체제의 통제 아래 두려고 하는 이유는 그가 집단주의 이념의 ‘참된 신자’이기 때문이다. 은신 좌파가 과잉 사회화된 좌파와 구분되는 부분은, 은신 좌파의 저항적 충동은  약하며, 그가 더욱 더 철저하게 사회화되어 있다는 사실이다. 그는 잘 사회화된 부르주아와도 구분된다.  그는 내면적으로 깊은 박탈감을 갖고 있으며, 그로 인해 사회 문제에 몸바쳐 일하고 집단에 몰두하게 되는 것이다. 그리고 그의 (잘 승화된) 권력에의 욕망은 보통 부르주아의 그것보다 훨씬 강렬하다.  


\section*{결언} 
231. 이 글에는 수많은 부정확한 진술들이 담겨 있으며, 그런 진술들에는 어떻게든 근거 조항들과 유보  조항들을 함께 이야기해야 했을 것이다. 그리고 어쩌면 우리의 진술 중 일부는 완전히 틀린 것일 수도  있다. 정보 부족으로 인해, 그리고 지면이 짧은 까닭에, 우리는 우리의 주장을 더 정확하게 공식화하지  못했고, 필요한 근거 조항을 빠짐 없이 글 속에 포함시키지도 못했다. 물론 이런 종류의 논의에서는  누구나 어쩔 수 없이 직관적 판단에 의존할 수밖에 없으며, 때로는 그런 직관적 판단이 틀릴 수도 있다.  따라서 우리는 이 글이 그저 진실에 어느 정도 가까울 것이라는 정도 밖에는 자신할 수 없다. 


232. 그래도 우리는 우리가 여기서 그린 그림의 전체적 윤곽이 대체로 정확할 것이라고 자신한다. 우리가 자신할 수 없는 한 가지에 대해 언급하겠다. 우리는 현대적 좌익 이념이 우리 시대에 국한된 현상이라고,  그리고 권력 과정의 붕괴로 인해 빚어진 병적 증상이라고 설명했다. 하지만 어쩌면 우리가 틀렸을 수도  있다. 자신들의 윤리관을 모든 사람에게 강제함으로써 권력에의 욕망을 충족하려고 하는 과잉 사회화된 부류들은 분명히 오래 전부터 존재해 왔다. 그러나 우리는 열등감, 자기 비하, 무력감, 스스로는 피해자가  아니면서도 피해자들과 자신을 동일화하는 것 등의 병적 심리가 운동에서 결정적인 역할을 수행하는  현상은 오로지 현대 좌익 이념에서만 나타나는 것이라고 생각한다. 스스로는 피해자가 아니면서  피해자들과 자신을 동일화하는 성향은 사실 19세기의 좌익 이념과 초기 기독교에서도 발견할 수 있는  특성이다. 하지만 그런 운동들에서는 우리가 아는 한, 자기 비하와 같은 병적 증상들이 현대 좌익 이념에서처럼 두드러지게 나타나지는 않았다. 하지만 우리에게는 현대 좌익 이념 이전에는 그런 병적 
증상을 지닌 운동이 전혀 없었다고 자신있게 주장할 만한 자격이 없다. 이는 역사가들이 주목해야할  중요한 질문일 것이다. 


\section*{각주} 


   
 \hypertarget{1}{1.} (문단 1) (2016년 추가) \textlangle{}Technological Slavery\textrangle{}의 \textlangle{}Letter to Dr. Skrbina of Nov. 23, 2004,  Part III.E.\textrangle{}를 참고할 것. 
 
 
\hypertarget{2}{2.} (문단 19) 우리가 모든, 또는 대부분의 깡패들과 무자비한 경쟁자들이 열등감을 겪고 있다고 주장하는  것은 아니다. 


 \hypertarget{3}{3.} (문단 25) 빅토리아 시대에 과잉 사회화된 많은 사람들이 성욕을 억제하거나, 억제하려고 노력한  결과로 심각한 정신질환을 겪었다. 프로이트의 이론은 명백히 이런 유형의 사람들에게 기반을 두었다.  오늘날 사회화의 초점은 성(性)에서 공격성으로 옮겨졌다.  
 

\hypertarget{4}{4.} (문단 27) 공학과 자연과학의 전문가를 반드시 포함하는 것은 아니다. 


\hypertarget{5}{5.} (문단 28) 이러한 가치들에 반대하는 많은 중상류층이 있지만 대부분의 경우 그들의 저항은 다소 숨겨져 있다. 그러한 저항은 대중 매체에 극히 제한된 정도로만 나타난다. 우리 사회의 주요 프로파간다는 앞서 말한 가치들에 호의적이다. 이러한 가치들이 우리 사회의 소위 공식적인 가치가 된 주된 이유는  그것들이 산업 사회에 유용하기 때문이다. 폭력에 반대하는 것은 폭력이 체제의 기능을 혼란에 빠뜨리기  때문이다. 인종차별에 반대하는 것은 인종차별로 인한 인종 갈등 또한 체제를 혼란에 빠뜨리고, 체제에  유용할 수도 있는 소수 집단의 재능을 낭비하기 때문이다. 가난은 “치유”되어야만 하는데, 그것은  하층민이 체제에 문제를 일으키고, 하층민과의 접촉이 다른 계층을 도덕적으로 타락시키기 때문이다.  여성들은 직업을 갖도록 장려되는데, 그것은 그들의 재능이 체제에 유용하며, 더욱 중요한 것은 여성이  정규 직업을 가짐으로써 체제에 통합되고, 가정보다는 체제에 더 구속되기 때문이다. 이것은 가족의  유대감을 약화시키는데 도움을 준다. (체제의 지도자들은 가족이 강화되기를 바란다고 말한다. 그러나 그 말의 진정한 의미는, 가족이 체제의 요구에 부응하도록 자녀를 사회화하는 유용한 도구로써 역할하기를  바란다는 것이다. 우리는 문단 51, 52에서 체제는 가족이나 작은 규모의 사회 집단이 강력해지거나,  자율성을 누리도록 할 수 없다는 것을 논할 것이다.) 


\hypertarget{6}{6.} (문단 33${\sim}$37)(2016년 추가) \textlangle{}Technological Slavery\textrangle{}의 \textlangle{}Letter to Dr. Skrbina of Oct. 12,  2004, Part I\textrangle{}와 부록 1을 참고할 것. 


\hypertarget{7}{7.} (문단 42) 대부분의 사람들은 그들 스스로 결정을 내리기를 원하지 않고, 지도자가 그들을 대신해  생각해 주기를 바란다고 반박할 수도 있다. 이 반론에는 어느 정도 일리가 있다. 사람들은 작은 일에  있어서는 그들 스스로 결정을 내리고 싶어하지만, 근본적이고 어려운 문제는 정신적 갈등을 유발하고,  대부분의 사람들은 정신적 갈등을 싫어한다. 따라서, 사람들은 어려운 결정을 내리는 데에 있어서 다른  이들에게 의지하는 경향이 있다. 다수의 사람들은 타고난 지도자가 아니라, 따르는 사람이지만, 사람들은  그들의 지도자와 직접적인 개인적 접촉을 갖기를 원하고, 어려운 결정을 내리는데 있어서 어느 정도  관여하고 싶어한다. 그들은 최소한 이 정도의 자율을 필요로 한다. 


\hypertarget{8}{8.} (문단 44)(2016년 보충) 나열된 증상들의 일부는, 철창에 갇힌 동물들에게서 나타나는 것과  비슷하다. Morris, passim, 특히 페이지 160${\sim}$225를 참고할 것. 이런 증상들이 권력 과정에 대한 기회  박탈로부터 어떻게 발생하는지 설명해보겠다: 인간 본성에 대한 상식적인 이해는, 성취를 위해 노력을  필요로 하는 목표의 부재는 지루함으로 이어지고, 지루함이 오래 지속되면 대개는 결국 우울감으로  이어지게 된다는 것을 말해준다. 목표 획득의 실패는 좌절과 자기 비하로 이어진다. 좌절은 분노로,  분노는 공격으로 이어지고, 종종 배우자, 자녀 학대의 형태로 나타난다. 오래 지속된 좌절은 대개는  절망으로 이어지고, 그리고 절망은 범죄, 불면증, 과식, 자신에 대한 악감정을 유발시킨다는 사실이  알려져 있다. 절망에 빠진 이들은 해독제로써 (새로운 흥분을 느끼기 위한 수단으로서의 탐욕적 쾌락, 
변태 성행위와 같은)쾌락을 추구한다. 권태 또한 과도한 쾌락 추구를 유발하는 경향이 있는데, 이는 다른  목표가 없는 사람들은 주로 쾌락을 목표로 삼기 때문이다. 앞서 서술은 단순화한 것이다. 실제로는 더욱  복잡한데, 물론 권력 과정에 대한 기회 박탈만이 전술된 증상들의 “유일한” 원인은 아니다. 그건 그렇고,  
우리가 절망을 언급할 때, 그것은 반드시 정신과 의사에 의해 치료되어야만 하는 그런 심각한 절망을  의미하는 것은 아니다. 종종 그저 가벼운 절망이 연루되기도 한다. 그리고 우리가 목표에 대해 이야기  할때, 우리는 반드시 장기적인 안목의 용의주도한 목표를 의미하지는 않는다. 오랜 인류의 역사를 통해  많은, 또는 대부분의 사람들에게 있어서 (단순히 자신과 가족들에게 하루하루 먹을 것을 제공하는) 먹고  사는 목표도 아주 충분한 목표였다. 


\hypertarget{9}{9.} (문단 52)(2016년 수정) 외부 사회에 거의 영향을 주지 않는 아미시(Amish) 같은 일부 수동적이고,  비밀스러운 집단 같은 부분적인 예외가 있을 수도 있다. 이와는 별개로, 몇몇 순수한 작은 규모의  공동체가 현재 미국에 존재하고 있다. 예를 들어, 청소년 조직폭력배와 사교(邪敎)들이다. 모두들 그들을  위험하다고 여긴다. 그리고 실제로 위험하다. 이 집단들의 구성원들은 체제가 아니라, 서로에게  충성하므로, 체제가 그들을 통제할 수 없기 때문이다. 집시들을 예로 들자면, 집시들은 절도나 사기를 잘 치는데, 그들은 언제든지 다른 집시의 결백을 위해 “증언”할 만큼 의리있기 때문이다. 예를 들어, Maas,  페이지 78${\sim}$79를 참고할 것. 만약에 아주 많은 사람들이 그러한 집단들에 속한다면, 명백히 체제는 아주  심각한 곤란을 겪게 될 것이다. 관련 있는 사례들에 대해서는, 부록 7을 참고할 것. Carrillo, 페이지  46${\sim}$47을 참고할 것. 


\hypertarget{10}{10.} (문단 54)(2016년 추가) 사실, 농촌 지역에서 문제가 덜 심각한지에는 의문의 여지가 있다. “작은  마을 우월성이라는 신화(The myth of small-town superiority)”(The Week, 10월 17일 2008년,  페이지 14)와 “뉴욕 주의 마인드(A New York state of mind)”(The Economist, 6월 25일 2011년,  페이지 94)를 비교해보라. 요점은 인구 과밀은 결정적 요인이 아니라는 것이다. 


\hypertarget{11}{11.} (문단 55)(2016년 추가) 예시: “20세기의 남녀들은 (1830년대, 1840년대의) 미시시피 계곡과  대평원 주들의 농부들이 ‘사람이 너무 많다’고 느꼈음을 결코 이해할 수 없을 것이다. 어떤 농부는 자신이 일리노이 주 서부를 떠난 이유가 ‘사람들이 바로 자기 코 아래에서 살기 시작해서’라고 말했다. 그의 가장  가까운 이웃이 12마일 떨어져 있었는데도 말이다.” Schlissel, 페이지 20 참고. Dick, 페이지 25 또한  참고할 것. 


\hypertarget{12}{12.} (문단 56) 그렇다. 우리는 19세기 미국이 심각한 문제들을 가지고 있었다는 것을 안다. 그러나  간략함을 위해 우리는 단순화된 용어로 표현했다. 


\hypertarget{13}{13.} (문단 61) 우리는 “하층계급”은 제외하겠다. 우리는 주류에 대해 얘기하고 있다. 


\hypertarget{14}{14.} (문단 62) 몇몇 사회과학자, 교육자, "정신 건강" 전문가와 같은 이들은 모든 이들이 만족스러운  사회적 삶을 영위하도록 노력함으로써, 사회적 욕구들을 첫번째 부류의 욕구로 만들기 위해 최선을  다하고 있다. 


\hypertarget{15}{15.} (문단 63) 끊임없는 재화 획득의 욕망은 정말 광고와 마케팅 산업의 인위적 산물인가? 분명히  인간에게는 재화 획득을 향한 본능이 없다. 기본적인 물질적 필요 이상으로는 물질적 부를 추구하지  않았던 (호주 원주민들, 멕시코 농경 문화, 몇몇 아프리카 문화들 같은)많은 문화들이 있었다. 반면 재화  획득이 중요한 역할을 한 산업화 이전의 문화들도 많았다. 따라서 우리는 오늘날의 물질추구 문화가  전적으로 광고나 마케팅 산업의 산물이라고는 말할 수 없다. 그러나 광고와 마케팅 산업이 그러한 문화를  형성하는데 중요한 역할을 했다는 것은 명백하다. 수백만 달러를 광고에 쏟아 붓는 대기업들은, 판매량 증가를 통해 그 돈을 되돌려 받을 수 있다는 확실한 증거 없이는 그만한 돈을 쓰지 않을 것이다. 


(2016년 추가) 1958년, 나는 여름 아르바이트를 찾기 위해, 구인광고에서 본 시카고의 한 사무실을  찾아갔다. 거기서 나는 몇몇 10대들과 함께 교외 지역에서 집집마다 방문해서 잡지를 구독하도록  설득하는 일을 맡게 되었다. 우리 모두 단 한명도 설득하지 못했다. 그러자 책임자가 우리에게 솔직하게  말했다. “우리의 일은 사람들이 원하지도 않고, 필요하지도 않은 것들을 사도록 만드는 것입니다.” 그리고 
그는 경력있는 전문 판매원이 우리가 실패했던 동네에 방문해 많은 구독자들을 확보하게될 것이라고  말해주었다. 실제로 구독자를 확보하고자 우리를 보낸게 아니라는 것은 분명하다. 어쩌면 믿음직하지  못한 청년들을 그저 시험해보는게 목적이었을 수도 있다. 목적이 무엇이었든 간에, 이 사례는 전문가들이 사람들이 실제로는 원하지 않는 것들을 사도록 조종할 수 있고, 실제로 조종한다는 사실을 보여준다.  “산업사회와 그 미래”를 썼을 때, 이 서술로 인해 저자의 신원이 들통날 가능성을 피하기 위해 이야기를  바꿔 말했었다. 지금의 서술이 정확한 것이다. 


\hypertarget{16}{16.} (문단 64) 목적 상실감의 문제는 지난 15년 동안은 조금 완화된 것처럼 보인다. 사람들은 이제  예전에 비해 신체적, 경제적 안전하다고 느끼지 못하고 있으며, 안전에 대한 욕구는 사람들에게 목표를  제공하기 때문이다. 그러나 목적 상실감은 안전 확보의 어려움에 의해 좌절감으로 대체되었다.  리버럴들과 좌파들이 사회가 모두의 안전을 보장하도록 함으로써 사회문제를 해결하려 하지만, 만약 그게 가능해진다면, 다시 목적 상실감의 문제를 가지고 올 것이기 때문에, 우리는 목적 상실감의 문제를  강조한다. 진정한 문제는 사회가 사람들의 안전을 잘 지켜줄 것이냐 못 지켜줄 것이냐가 아니다. 문제는  사람들이 그들의 안전을 스스로의 손으로 지키기 보다는, 체제에 의존하고 있다는 것이다. 여담으로,  이것이 일부 사람들이 무장할 권리를 중요하게 여기는 이유 중 하나이다. 총기 소지를 통해 그들은  부분적으로나마 스스로 자신의 안전을 지킬 수 있게 된다. 


\hypertarget{17}{17.} (문단 65)(2016년 추가) \textlangle{}The Missoulian\textrangle{}지, 5월 25일, 1988년자 기사 “Small businesses, take heart”에서 다음과 같이 적고 있다: “...그러나 당신이 진정한 사업가라면, 당신은 프랜차이즈에 적합하지  않을 수도 있다. 피츠버그의 프랜차이즈 개발사는 그들의 고객들이 창의력, 독립성과 같은 강한 사업가적  기질을 지닌 사람을 골라내기 위한 2시간 30분짜리 심리테스트를 이용해왔다고 말한다. 사업가 성향의 사람들은 프랜차이즈가 정해놓은대로 일하기를 거부하는 “문제아”들에 불과하다는 것이다.”\textlangle{}월스트리트 저널\textrangle{}


\hypertarget{18}{18.} (문단 66) 정부 규제를 줄이기 위한 보수주의자들의 노력은 평범한 사람들에게는 거의 아무런 도움이 되지 못한다. 첫째, 대부분의 규제들은 필수적이기 때문에, 약간의 규제들만 제거할 수 있다. 둘째, 대부분의 규제는 평범한 개인 보다는 사업에 영향을 미치는 것들이어서, 규제완화의 주된 결과는 권력을 정부로부터 사기업으로 양도하는 것이다. 이것은 평범한 사람들에게 있어서, 그들의 삶에 대한 정부의  간섭이 대기업들의 간섭으로 대체됨을 의미한다. 따라서 대기업은, 예를 들어 그들의 화학 폐기물을 수원(水源)에 방류할 권한을 허가받을 것이고, 사람들이 암에 걸리도록 만들 것이다. 보수주의자들은 평범한  사람들을 그저 호구로 여기고, 거대 정부에 대한 그들의 분노를 악용해 대기업의 권력을 키워주고 있다. 


\hypertarget{19}{19.} (문단 67)(2016년 추가)Anthony Lewis 인용.(New York Times, 4월 21일 1995년) 


\hypertarget{20}{20.} (문단 69)(2016년 추가)Último Reducto가 많은 원시인들이 질병의 책임을 “상상 속의 악마”가  아닌, 주술로 돌렸다는 점을 지적해주었다. 누군가가 심각하게 앓고 있으면, 누가 “마녀”인지 찾아내  죽였다. 산업화 이전 시대 사람들은 마법과 주술을 믿었다. 이것이 이해할 수 없는 재앙을 설명해줄 수  있었고, 재앙을 막을 수 있다는 환상을 주었기 때문이다. 현대 세계에서 그와 유사한 역할을 하는 믿음에  대한 논의는 대단히 흥미로울 것이다. 하지만 이는 논점에서 벗어난다. 


\hypertarget{21}{21.} (문단 73)누군가가 특정 프로파간다가 사용되는 목적에 동의할 때, 그는 대개 그것을 "교육"이라고  부르거나, 비슷한 완곡어를 사용한다. 그러나 어떤 목적으로 쓰이든 어쨋든 프로파간다는  프로파간다이다. 


\hypertarget{22}{22.} (문단 75)(2016년 추가)내가 지나치게 비약한 것일 수도 있다. 턴불(Turnbull)의 \textlangle{}Wayward  Servants\textrangle{} 페이지 127에 따르면, 음부티족 피그미들은 “남녀를 불문하고 노인이 되는 것을 어느 정도  꺼렸다.”고 한다. 그럼에도 불구하고, 현대인들이 젊음을 유지하기 위해 얼마나 많은 시간을  투자하는지를 보면, 현대인들이 원시인들보다 노화를 훨씬 더 두려워 한다는 점은 사실인 것 같아 보인다.


\hypertarget{23}{23.} (문단 83)우리는 파나마 침공에 대한 찬성 또는 반대하는 것이 아니다. 우리는 그저 논점을 설명한  것이다. 


\hypertarget{24}{24.} (문단 87${\sim}$92)(2016년 추가)과학자들의 동기에 관한 이 논의는 확실히 부적절하다. 더 엄밀한  논의는 \textlangle{}Technological Slavery\textrangle{}의 \textlangle{}Letter to Dr.P.B on the Motivations of Scientists\textrangle{}를 참고할 것. 


\hypertarget{25}{25.} (문단 94)(2016년 추가)Último Reducto가 사람들은 단 한번도 삶의 환경을 완벽하게 통제할 수  없었다는 점을 고려할 때, ‘자유’에 대한 정의(定義)를 정제하거나, 설명할 필요가 있음을 지적해 주었다.  예를 들어, 사람들은 자칫 식량 부족을 유발할 수 있는 나쁜 날씨를 통제할 수 없었다. 나는 더 엄밀한  정의 또는 설명을 추가할 수 있다고 보지만, 여기서는 하지 않겠다. 




\hypertarget{26}{26.} (문단 95)미국 헌법이 발효되기 이전, 아메리카 식민지가 영국의 지배 하에 있을 때는 자유에 대한  실질적인 법률적 보장이 더 적었다. 그러나 독립전쟁 전후의 산업화 이전의 미국인들은, 산업혁명이  일어난 후보다 더 많은 개인적 자유를 누릴 수 있었다. 우리는 휴 데이비스 그레이엄(Hugh Davis  Graham)과 테드 로버트 거(Ted Robert Gurr)가 편집한 "미국에서의 폭력: 역사적이고 비교적인 시각" 중 로저 레인(Roger Lane)이 쓴 12장 476${\sim}$478 페이지에서 인용했다:


"(19세기 미국에서의)점진적인 재산권 제도의 강화와 그에 따른 공식적인 법률의 집행에 대한 의존의  증가는 ... 사회 전체에 공통적인 것이었다. ... 사회적 행위의 변화는 아주 장기적이고 광범위해서  대부분의 (산업 도시화 그 자체의)기본적인 현대 사회 과정과의 연관을 제안할 정도였다. ... 1835년  메사추세츠에는 약 660,940명의 인구가 있었고, 81\%는 산업화 되지 않은 시골에 살고 있었다. 그곳의 시민들은 모두 상당한 개인적 자유에 익숙했다. 마부든 농부든 공예가든 그들은 모두 그들 자신의 일정을 짜는 것에 익숙했고 그들의 작업의 성격은 그들을 물질적으로 독립하도록 만들었다... 개인적인 문제들, 과실 또는 범죄조차 일반적으로 광범위한 사회적 관심을 일으키지는 못했다... 그러나 1835년부터 시작된 도시와 공장으로의 이주의 충격은 19세기와 20세기에 걸쳐 개인적 행동에 점진적인 영향을 주었다. 공장은 행동의 규제, 시계와 달력에의 복종, 십장과 감독자를 요구했다. 도시 또는 읍에서는 높은 인구밀도로 인해, 전에는 불쾌하게 생각되지 않았던 많은 행동들이  금지되었다. 블루 칼라와 화이트 칼라 노동자 모두 더 거대해진 체제에서 그들의 동료들에게  상호의존하게 되었다. 어떤 사람의 작업이 다른 사람의 작업에 맞추어짐에 따라, 그의 일은 더 이상 자신만의 것이 아니게 되었다. 새로운 삶과 노동 방식의 결과는 1900년에 이르러 명백해졌다. 2,805,346명의 메사추세츠 인구 중 약 76\%는 도시인구로 구분되었다. 독립적인 사회에서는 용인되었던 많은 폭력 또는 불법적인 행동은, 나중의 더욱 형식화되고 협동적인 환경에서는 더이상  용납되지 않았다... 도시로의 이주는, 요약하자면, 그들의 선조보다 더욱 온순하고 더욱 사회화되고 더욱 `문명화'된 세대를 만들어냈다." 

\hypertarget{27}{27.} (문단 95)(2016년 추가)Último Reducto가 ‘왕국’보다는 ‘족장제’가 더 정확한 표현임을 지적해  주었다. 하지만 어느쪽이든 논점은 동일하다. 


\hypertarget{28}{28.} (문단 97)(2016년 추가) 볼리바는 다음과 같이 적었다: “인류의 명예와 발전을 위하지 않는다면,  어떤 자유도 정당화 될 수 없다.” 


\hypertarget{29}{29.} (문단 97)(2016년 추가)Tan, p.202. 


\hypertarget{30}{30.} (문단 97)(2016년 추가)Tan, p.259. 


\hypertarget{31}{31.} (문단 114)각주 21을 참고할 것. 


\hypertarget{32}{32.} (문단 115)(2016년 추가) “불만 가득한 학생이, 졸린 표정으로 가방을 메고 마지못해 학교를 향해  달팽이처럼 기어간다.” 셰익스피어, As You Like It, Act 2, Scene 7(약간 수정됨)


\hypertarget{33}{33.} (문단 117) 체제 옹호자들은 한두명의 투표자에 의해 결과가 결정된 사례들을 언급하길 좋아한다.  그러나 그러한 경우는 드물다. 


\hypertarget{34}{34.} (문단 119) "오늘날, 기술적으로 진보된 곳에서, 인간은 지리적, 종교적, 그리고 정치적 차이에도  불구하고 매우 비슷한 삶을 살고 있다. 시카고의 기독교인 은행직원, 도쿄의 불교도 은행직원, 모스크바의 공산주의자 은행직원의 일상 생활은 그들 중 어느 한 사람의 삶과 천년전에 살았던 누군가의 삶이  비슷했던 것 보다 훨씬 더 똑같다. 이러한 유사성은 공통된 기술의 결과이다..." L. Sprague de Camp,  "고대의 기술자들", Ballentine edition, 18 페이지. 세명의 은행 직원들의 삶은 "동일"하지는 않다.  이념이 "어떤" 역할을 했다. 그러나 모든 기술 사회에서는 살아남기 위해서 대부분은 "거의" 같은 궤도를  따라 공전한다.  


\hypertarget{35}{35.} (문단 122) 의학적 진보의 바람직하지 못한 결과에 대한 다른 예로서, 확실한 암 치료법이 발견되었다고 가정해보자. 치료비가 너무 비싸서 엘리트 이외의 사람들은 이용할 수 없다고 할 지라도, 이는 발암 물질 유출을 막아야 할 동기를 크게 줄일 것이다.  


\hypertarget{36}{36.} (문단 123) 무책임한 유전공학자가 수많은 테러리스트들을 만들어낼 수도 있음을 생각해보라. 


\hypertarget{37}{37.} (문단 128) 많은 사람들이 수많은 좋은 것들이 모여 하나의 나쁜 것을 만들어 낼 수 있다는 개념을  모순이라고 여길 것이기에, 우리는 동일한 예를 들어보이겠다. A가 B와 체스를 두고 있다고 가정하자.  그랜드마스터 C는 게임을 A의 어깨너머로 보고 있다. 물론 A는 게임을 이기기를 원하기 때문에, 만약에  C가 훈수해 준다면 C는 A에게 도움을 주는 것이다. 그러나 이제 C가 A에게 모든 수를 훈수해 준다고  가정해 보자. 매순간마다 C는 A에게 최선의 수를 보여줌으로써 도움을 주지만, A의 모든 수를  훈수함으로써 게임을 망치게 된다. 만약 다른 누군가가 모든 수를 훈수한다면 A가 게임을 하는 의미가  없어지기 때문이다. 현대인들의 상황은 A의 상황과 유사하다. 체제가 수없이 많은 방법으로 개인의  생활을 편하게 해주지만, 그럼으로 인해 그에게서 자신의 운명에 대한 통제권을 박탈해간다. 


\hypertarget{38}{38.} (문단 131) 각주 21을 참고할 것. 


\hypertarget{39}{39.} (문단 137) 여기서 우리는 주류 사회 내부의 가치 충돌만을 고려하고 있다. 간결함을 위해서 우리는  야생의 자연이 인간의 경제적 복지보다 더 중요하다는 생각과 같은 "아웃사이더"들의 가치는 제외하였다. 


\hypertarget{40}{40.} (문단 137) 이익이 반드시 “물질적인” 이익은 아니다. 예를 들어, 자신의 이념이나 종교를 고양하는  것과 같은 어떤 정신적인 욕구를 충족시키는 이익이 존재할 수도 있다.  


\hypertarget{41}{41.} (문단 139) 제한: 몇몇 영역에서 어느 정도의 자유를 허용하는 것은 체제에 이로운 것이다. 예를  들어, (적당히 제약된)경제적 자유는 경제성장을 촉진시키는데 효과적인 것으로 판명되었다. 그러나 오직 계획되고, 한정된 자유만이 체제에 이로운 것이다. 가끔 목줄의 끈이 길어진다고 하더라도, 개인은 항상  목줄에 묶여있어야 한다. (문단 94, 97을 볼 것.) 


\hypertarget{42}{42.} (문단 143) 우리는 사회의 효율성이나 지속 가능성이 항상 사회가 사람들에게 가하는 억압이나  고통의 양에 반비례한다고 주장하는 것은 아니다. 이는 분명 사실이 아니다. 많은 원시 사회들이 유럽 사회보다 사람들을 덜 억압했지만, 유럽 사회는 어떤 원시 사회보다도 훨씬 더 효율적이었다는 것이  입증되었으며, 기술이 제공한 이점 덕분에 그런 사회들과의 충돌에서 항상 승리했다.  


\hypertarget{43}{43.} (문단 147) 만약 보다 효율적인 공권력이 범죄를 억제할 수 있기 때문에 명백히 좋은 것이라고  생각한다면, 체제에 의해 정의된 범죄가 당신이 생각하는 범죄와 반드시 같지는 않음을 명심하라. 오늘날, 대마초를 피우는 것은 "범죄"이고, 미국의 어느 지역에서는 등록되지 않은 권총 소지 또한 그렇다.  내일에는, 등록됐든 되지 않았든, “모든” 형태의 총기 소지는 범죄가 될 것이고, 같은 일이 엉덩이를 찰싹 때리는 것과 같은, 승인되지 않은 자녀 양육법에서도 일어날 것이다. 어떤 나라에서는 정치적인 반대  의견을 갖는 것이 범죄이고, 헌법이나 정치제도가 영원한 것은 아니므로, 똑같은 일이 미국에서도 일어나지 말라는 보장은 없다. 만약 어떤 사회가 강력한 공권력을 필요로 한다면, 그 사회에는 심각하게 잘못된 무언가가 있는 것이다. 아주 많은 사람들이 법률을 따르기를 거부하거나, 혹은 강제되기 때문에 어쩔 수 없이 따르고 있다면, 그 사회는 사람들에게 과도한 압력을 가하고 있는 것이다. 과거의 많은 사회들은 어떤 공권력 없이도 잘지냈었다. 


\hypertarget{44}{44.} (문단 147)(2016년 추가) 사람들은 잡생각이 들 때 불행해진다는 증거가 있다.(Killingsworth \& D.T. Gilbert 참고) 에스키모인들이나 인디언들의 진술에 따르면, 오두막 앞에서 서성거릴 때, 그들의  머리 속에는 잡다한 생각이 들지 않고, 그저 생각이 사라진다고 한다. 비슷하게, 나 역시 몬태나에서  전성기를 보낼 때, 내 정신은 떠돌지 않고 그냥 텅 비어 있었다. 적어도, 나에게 지루함이란 거의 존재하지 않았다.(Kaczynski, Technological Slavery(2010), p. 406 참고) 


\hypertarget{45}{45.} (문단 151) 확실히, 과거의 사회들도 인간의 행동에 영향을 줄 수 있는 수단을 가지고 있었지만, 지금 개발되고 있는 기술적인 수단에 비하면 원시적이고 비효율적이었다.  


\hypertarget{46}{46.} (문단 152)(2016년 보충)그러나, 어떤 심리학자들은 공식적으로 인간의 자유에 대한 그들의 경멸을  보여주고는 했다. 예를 들어, 제임스 맥코넬(James V. McConnell)은 \textlangle{}Behavior Control: Boon or  Bane?\textrangle{}(Chicago Sun Times, March 7, 1971)에서 다음과 같이 말했다. “감각 박탈과 약물, 최면,  보상체계 조작의 조합을 통해 개인의 행동을 거의 절대적으로 통제하는 것이 가능해질 날이 언젠가  오리라고 믿습니다. 우리는 사람들이 태어나서 죽을 때까지 사회가 원하는 행동만 하도록 우리 사회를  재구성해야합니다.” 그리고 수학자 클로드 섀넌(Claude Shannon)은 (1987년 8월)Omni에서 다음과  같이 말했다. "저는 인간과 로봇의 관계가 개와 인간의 관계와 같아질 미래를 꿈꾸고 있으며, 기계를  응원하고 있습니다." 


\hypertarget{47}{47.} (문단 153) 각주 21 참고. 


\hypertarget{48}{48.} (문단 154) 이것은 공상과학 소설이 아니다. 문단 154를 쓴 직후 우리는 \textlangle{}Scientific American\textrangle{} 지에서 과학자들이 미래의 잠재적 범죄자를 식별하고 생물학적, 심리학적인 수단의 조합을 통해 그들을  치료하는 기술을 열심히 개발하고 있다는 기사를 접하게 되었다. 어떤 과학자들은 가까운 미래에 그  치료법을 강제로 적용해야한고 주장한다. (Scientific American 1995년 3월 자에 실린 W. W. Gibbs가  쓴 "범죄 요인 찾기"를 볼 것.) 아마도 당신은 그 치료가 폭력적인 범죄자들에게 적용될 것이므로  괜찮다고 생각할지도 모른다. 그러나 물론 그것에서 그치지 않을 것이다. 다음 치료 대상은 (마찬가지로  인간의 생명을 위협하는)음주 운전자들일 것이다. 그 다음 치료 대상은 아마 자녀의 엉덩이를 때리는  사람들, 그 다음 치료 대상은 벌목장비를 망가뜨리는 환경주의자들이 될 것이다. 결국에는 체제를  불편하게 만드는 행동을 하는 모든 사람들이 치료 대상이 될 것이다. 


(2016년 추가) 위의 서술은 1995년에 쓰여졌다. 그러나 내가 아는 바에 따르면, 어린이들이 범죄자로  성장하는 것을 막는 강제 치료법은 아직 시작하지 않았다. 이 사실은 내가 1995년 완전히 이해하지는  못했던 두 가지 쟁점을 보여준다. 
첫째, 우리의 자유와 존엄성에 대한 침해는 예상치 못한 방향에서 다가오는 경향이 있다. 우리의 자유와  존엄성에 대한 명백한 위협은 대체로 실현되지 못하거나, 실현되는데 예상했던 것보다 훨씬 오래 걸린다.  우리의 자유와 존엄성에 대한 침해는 계속 진행되지만, 누구도 예측하지 못했던 방법으로 진행된다. 예를  
들어, 1970년에는 소유자에게 지나치게 강한 권력을 줄 (인간과 같은)“강”인공지능이 15년 후에 나올 것으로 예상되었다.(Darrach, p. 58) 하지만 지금까지 등장하지 않았다. 한편으로는, 오늘날 대부분의  사람들은 컴퓨터 미디어에 대단히 강하게 몰입, 의존하고 있고, 기술 산업은 1970년에 어느 누구도  꿈꾸지 못했던 방법으로 사람들에게 강한 권력을 행사하고 있다. 
둘째, 개인들을 면밀하게 관찰하거나 치료해야하는 인간 행동 통제 수단들은 전체 인구에 효과적으로  적용하기 어렵다. 따라서 우리의 사법제도에게 있어, 실제 또는 잠재적 범죄자 개인들을 통제하는 것은,  범죄 인구 전체의 행동을 통제하는 것에 비해 비싸고 비효율적이다. 우리의 교육제도가 개별 학생들을  통제하는 것도 마찬가지다. 따라서 오늘날의 사회는 개인들의 차이를 고려하기 보다는, 주로 대중 전반에게 적용될 수 있는 수단들에 의존한다. 예를 들어, 전자 감시 장치들을 이용해 잠재적 범죄자들을  위협하고, 물리적 강제력을 강화하거나, 교육제도를 동원해 프로파간다를 효과적으로 널리 퍼뜨린다.


\hypertarget{49}{49.} (문단 167)(2016년 추가) 지금의 나는 점진적인 붕괴가 불가능할 것으로 생각하기 때문에, 우리는  이 가능성을 고려해선 안된다.(Kaczynski, Anti-Tech Revolution 2장 참고.) 


\hypertarget{50}{50.} (문단 171${\sim}$178)(2016년 추가) 지금의 나는 이 부분의 고찰을 한참 뛰어넘었다. 더 그럴듯한 미래에 대해서는 Kaczynski, Anti-Tech Revolution 2장을 참고할 것. 


\hypertarget{51}{51.} (문단 184) 기술에 대한 대안적 이상으로서의 자연의 또 다른 장점은, 많은 사람들에게 자연은  일종의 종교적 경외심을 불러일으킨다는 것이다. 따라서 자연은 아마 종교적인 바탕에서 이상화될 수도  있을 것이다. 많은 사회에 있어서 종교가 기존의 질서를 지지하고 정당화하는 역할을 한 것은 사실이지만,  종교가 종종 반란의 토대를 제공했던 것 또한 사실이다. 따라서 기술에 대항한 반란에 종교적 요소를  도입하는 것도 유익할 것이다. 게다가 오늘날 서구 사회에는 종교적 기반이 없기 때문이다. 요즘의 종교는 좁고, 근시안적인 이기주의를 위한 쉽고 투명한 지원책으로 사용되거나 (몇몇 보수주의자들이 이 방법을 사용한다.) 또는 (많은 복음주의자들에 의해)쉽게 돈을 버는 수단으로 악용되거나, 또는 (개신교 근본주의, "사교" 같은)조잡한 비합리주의로 타락하거나, (천주교, 정통 청교도 처럼)단순히 침체되어있다. 최근에 서양이 목격한 강력하고 광범위하며 역동적인 종교에 가장 가까운 것은 좌익 이념이지만, 오늘날 좌파는 분열되어 있어 어떠한 명확하고 통합된 진취적인 목표도 가지고 있지 않다. 따라서 우리 사회에는 종교적 공백이 있고, 아마도 기술에 반대하며 자연에 초점을 맞춘 종교로 채워질 수도 있다.  그러나, 이 역할을 위한 종교를 인위적으로 설계하는 것은 실수일 것이다. 그런 식으로 발명된 종교는  아마도 실패할 것이다. "가이아" 종교를 예로 들어보자. 그 신도들은 “진심으로” 그것을 믿고 있는가? 아니면 그저 연극을 하고 있는 것인가? 만약 그들이 그저 연극을 하고 있는 것이라면, 그들의 종교는  결국에는 주저앉을 것이다. 만약 당신이 “진심으로” 그 종교를 믿고, 그 종교가 많은 사람들에게 깊고,  강력하고, 순수한 반응을 불러일으키지 않는 한, 자연과 기술의 충돌에 종교를 끌어들이지 않는 편이  최선일 것이다. 


\hypertarget{52}{52.} (문단 186${\sim}$188) 1995년에 이 문단들을 쓰고 난 후, 나보다 이 분야에 훨씬 권위있는 작가들이 이  두 쟁점에 동의했음을 알게되어서 기쁘다. \textlangle{}산업사회와 그 미래\textrangle{}의 이념의 두 수준에 대한 구분은  플레하노프와 레닌의 “프로파간다”와 “선동”의 구분과 대강 유사하다. NEB(2003), Vol. 26,  “Propaganda,” p. 171; Ulam, p. 34 & note 21; Lenin, What is to be Done?, Chapt. III, Part B; pp. 101-02 참고. Christman. Alinsky, pp. 27-28, 78, 133-34에서는 선동 과정에서 한편은 무조건 좋고, 다른 한편은 무조건 나쁘다는 흑백용어를 쓸 필요가 있음을 강조하고 있다. 


\hypertarget{53}{53.} (문단 189)(2016년 수정) 그러한 마지막 투쟁이 일어난다고 가정해보자. 산업 체제가 다소 점진적으로 제거될 것이라고 보기는 힘들다. (문단 4, 167, 그리고 각주 49를 참고.)  


\hypertarget{54}{54.} (문단 195) 평범한 사람들의 삶을 결정짓는데 있어서, 사회의 정치적 구조보다 경제적, 기술적  구조가 훨씬 더 중요하다. (문단 95, 119, 그리고 각주 26, 34를 볼 것.) 


\hypertarget{55}{55.} (문단 204)(2016년 추가) 이 문장은 철회하겠다. 1995년 당시에는 혁명가들이 자녀를 많이 가져야한다는 조언에 어느 정도의 장점이 있었을 것이다. 하지만 지금의 나는 체제를 향한 최후의, 그리고 결정적인 투쟁이 이미 태어난 사람들에 의해 진행되어야 한다고 믿는다. 물론 그들을 따르는 첫 세대로부터 약간의 도움을 얻을 수도 있다. 잠재적 혁명가들이 자녀를 갖게될 경우, 가족을 돌보느라 바빠 혁명가로써 쓸모가 없어지게 된다는 문제가 있다. 


\hypertarget{56}{56.} (문단 208)(2016년 추가)Último Reducto가 원시 사회에서 소규모 기술이 퇴화한 사례가 있음을  지적해 주었다. 그러나 만약 문명 사회에서 그런 사회가 발견된다 할지라도, 우리의 주장은 영향받지  않는다. 


\hypertarget{57}{57.} (문단 208)(2016년 추가) NEB(2003), Vol. 15, “Building Construction,” p. 317 참고.


\hypertarget{58}{58.} (문단 210)(2016년 추가) Bury, pp. 58${\sim}$60, 64${\sim}$65, 113 참고. 


\hypertarget{59}{59.} (문단 215) 이 말은 아나키즘 중에서 우리의 특정 분파를 가리키는 것이다. 다양한 사회적 태도들이 "아나키스트"라 불리어 왔고, 자신들을 아나키스트라 여기는 많은 사람들은 문단 215의 우리의 주장을  받아들이지 않을지도 모른다. 아마도 FC를 아나키스트로 받아들이지 않고, 분명히 FC의 폭력적인 방법을 인정하지 않을 그런 비폭력적인 아나키스트 운동도 존재한다는 사실은 언급되어야 한다. 
(2016년 추가) 1995년에 나는 어느 정도 알려진 정치적 정체성을 갖는 것이 유리할 것이라고 생각해  FC를 “아나키스트”라고 소개했다. 당시 나는 아나키즘에 대해 아는 바가 거의 없었다. 그 후 나는 미국과  영국의 아나키스트들이 절망적으로 무능한 아무짝에도 쓸모없는 몽상가들이라는 사실을 알게 되었다.  말할 것도 없이, 이제 나는 아나키스트라고 불리는 것을 거부한다. 


\hypertarget{60}{60.} (문단 219) 많은 좌파들은 적대감에 의해서도 고취되는데, 아마 그 적대감은 부분적으로 좌절된 권력에의 욕구로부터 생긴 것일 것이다. 


\hypertarget{61}{61.} (문단 222)(2016년 추가) 이 주장에는 반론의 여지가 있다.(Hoffer, 14 참고) Rothfels, p.63 에서는 1933년 히틀러가 집권했을 때 많은 공산주의자들이 나치로 전향했으며, 1945년 독일의 소련 점령지에서 많은 나치들이 공산주의로 전향했음을 지적한다. 하지만 그런 상황에서 전향한 사람들이  반드시 ‘참된 신자’는 아닐 것이다. 그보다는 언제든 우세해 보이는 정당에 달라붙는 기회주의자들일  것이다. 설령 20세기 전반의 참된 나치, 참된 공산주의자들이 동일한 심리적 성향을 갖고 있었다고  할지라도, 모든 ‘참된 신자’들이 어느 정도 동일한 심리적 요소를 갖고 있다고 할지라도, 해당 저자는  최근의 북미와 서유럽의 나치와 참된 좌파들이 심리적으로 유사하다고 볼 설득력 있는 근거는 없다고  말한다. 


\hypertarget{62}{62.} (문단 224)(2016년 추가) 내가 누구에게나 권력에의 욕구가 있음을 강조한 사실로 인해, “권력에  눈이 먼”이라는 표현이 약간의 혼동을 일으켰다. 이로인해 많은 독자들이 모든 사람을 “권력에 눈이 먼”  사람으로 간주해야한다고 받아들였다. 하지만, “권력에 눈이 먼”이라는 표현은 일반적으로 정치적,  경제적 권력 또는 명령권을 이용해 다른 사람들을 지배하려드는 사람들에게만 쓰이는 표현이다. 예를  들어, 지적, 예술적, 신체적 재능, 손재주를 통해, 혹은 기술 체제로부터 독립적인 삶을 통한 권력을 추구하는 사람들이 반드시 일반적인 의미에서의 권력에 눈이 먼 사람은 아닐 것이다. 


\hypertarget{63}{63.} (문단 229) 우리가 오늘날 우리 사회에 존재하는 이러한 운동들에 공감하는 사람들을 의미한다는  것을 이해해야 한다. 여성, 동성애자 등이 동등한 권리를 가져야 한다고 믿는 사람이 반드시 좌파인 것은  아니다. 우리 사회에 존재하는 페미니즘, 동성애자 권리 등의 운동들은 좌파로 규정되는 독특한 이념적  분위기를 갖고있다. 예를 들어, 어떤 사람이 여성이 동등한 권리를 가져야 한다고 믿는다고 해서, 오늘날  존재하는 페미니즘에 반드시 공감하는 것은 아니다. 
\end{multicols}

\end{document}
